\hypertarget{ux51fdux6570-function}{%
\subsubsection{函数 (function)}\label{ux51fdux6570-function}}

听到''function'', 第一反应可能是''方程'', 事实上函数才是 function,
方程或者等式应该叫做 equation. 不细想可能会觉得他们差别不大,
不过等式通常是来解未知的, 而函数通常是用来描述两个变量之间的关系.

非正式的, 初中阶段类似 \(y=ax+b\)
形式的一次函数可能是大家接触到的比较早的一个函数的例子. 有时候也会有类似
\(f(x)=ax+b\) 这种标记, 意为: 一个以 \(x\) 为变量的函数;
这种标记方便之处是可以很方便的标出当 \(x\) 等于某个值, 例如 \(42\)
时函数的取值, 即 \(f(42)=42a+b\). 另外 \(y=ax+b\) 也可以写成
\(y(x)=ax+b\) , \(y\) 作为函数, 其变量为 \(x\).

传统定义, 函数通常用来描述变化过程中两个变量的关系, 若有两个变量 \(x\)
和 \(y\), 如果对于任意一个 \(x\) 都有一个唯一确定 (unique) 的 \(y\)
与其对应, 那么就说 \(x\) 是自变量 (independent variable), \(y\) 是因变量
(dependent variable). \(x\) 的取值范围称为函数的\textbf{定义域}
(domain), 可能输出的 \(y\) 范围称为函数的\textbf{值域} (range).

自然科学和社会科学通常会用类似下左图的形式来可视化函数,
横轴竖轴分别标记两个变量的取值范围, 通过函数图像,
便可以找到当一个变量取某个特定值时另一变量对应的值.
下右图给出了一个经济学中的例子

\begin{quote}
随着某商品的市场价格 (price) 增高, 生产者的生产意愿自然是增高的,
因为不考虑成本和其他因素变化的情况下 (Ceteris Paribus),
多生产的利润会更高, 因此供给曲线 (supply) 上扬, 反映出价格和供给量
(supply quantity) 的正相关; 另一方面,
消费者的消费意愿随着价格上涨自然是下降的, 于是需求曲线 (demand) 下压,
反映出价格和需求量 (demand quantity) 的负相关.

两条曲线分别对应着供给量和需求量关于价格的函数,
两函数的焦点反映了理想的自由市场下的最终成交价格和供需量
(因为在这个点达到了供需平衡 - equilibrium) ,
焦点向下作竖直线与横轴的焦点便显示了最终成交价格,
焦点向左作水平线与竖轴的交点便显示了平衡点的供需量.
\end{quote}

在工程和计算机科学等思维里, 函数更像下左图所示, 给定一个输入 (input),
函数如同一台机器, 在加工后给出一个输出 (output); 这台机器非常可靠,
同样的输入能够稳定输出同样结果. 下右图给了一个例子

\begin{quote}
假想有这样一个叫做''首都'' (Captital) 的函数,
放入一个国家名便会稳定输出这个国家的首都,
数学上我们可以这么标记下右图的例子 \(\text{Capital(China)=Beijin}\).
\end{quote}

函数的近现代定义和工科思维里的图景就很像, 考虑集合 \(X\) 和 \(Y\),
且它们不是空的, 如果存在某种特定的对应关系 \(f\), 使得对于 \(X\)
中任意一个元素 \(x\), 在 \(Y\) 中都有唯一确定的元素 \(y\) 和 \(x\) 对应,
那么就称\textbf{映射} (mapping)\footnote{~Mapping 这个词用在这很贴切,
  map有地图的意思,
  ``映射''和地图上的每个点对应着实际区域上的一个个位置很相似.}
\(f: A\rightarrow B\) 为从 \(X\) 到 \(Y\) 的一个函数, 记作
\(y=f(x), x\in X\) 或者 \(f(X)=\{y|f(x)=y, y\in Y\}\);
第一种记法强调元素的映射, 第二种记法强调整个集合的映射, \(X\)
在这里便是这个映射的定义域, \(f(X)\) 是值域, \(Y\)
是这个映射的\textbf{陪域} (codomain, 也叫做上域, 到达域, 对应域),
大括号表示 \(X\) 被映射到的集合, 其元素 \(y\) 满足竖线后的条件, 即 \(y\)
是自 \(x\) 通过 \(f\) 这个映射得到, 并且 \(y\) 属于 \(Y\).
这样定义的直观感受类似下左图, 之前''首都''函数便类似下右图

\begin{quote}
Country这个集合里包含了很多国家, \{中国, 美国, 日本, \ldots\}; Capital
city这个集合里包含了很多城市, \{北京, 华盛顿, 东京, \ldots\};
Capital这个函数便描述了Country中的元素和Capital city中的元素的对应关系.
\end{quote}
