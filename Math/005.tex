\hypertarget{ux4e09ux89d2ux51fdux6570-trigonometry}{%
\subsubsection{三角函数
(Trigonometry)}\label{ux4e09ux89d2ux51fdux6570-trigonometry}}

三角函数最基本的使用应该是表示直角三角形的变长比. 如下图所示, 三角形
\[ABC\] 为直角三角形, 将 \[\angle BAC\] 记作 \[\theta\], 对于两条直角边
\[AB\] 和 \[BC\], 边 \[AB\] 在 \[\theta\] 边上, 称它为\textbf{邻边}
(adjacent), 边 \[BC\] 在 \[\theta\] 对面, 称它为\textbf{对边}
(opposite), 剩余的边 \[AC\] 被称为\textbf{斜边} (hypotenuse)。

易见, 各变长比仅和 \[\theta\] 相关\footnote{当然也可以说和除了直角外的另一个角
  \[(90^\circ-\theta)\] 相关; 边长比可以通过一个除直角外的角确定是因为,
  除直角外另一角相等的直角三角形都相似, 它们的边长比是一致的。},
三角函数便是用来表示各个比例的, 常用的三角函数有

\[\begin{align*}\cos\theta&=\frac{邻边}{斜边}=\frac{AB}{AC},\\
\sin\theta&=\frac{对边}{斜边}=\frac{BC}{AC},\\
\tan\theta&=\frac{对边}{邻边}=\frac{BC}{AB}.\end{align*}\]

不难看出\[\tan\theta=\frac{\sin\theta}{\cos\theta}\].

另外还有

\[\begin{align*}\sec\theta&\equiv\frac{1}{\sin\theta},\\
\csc\theta&\equiv\frac{1}{\cos\theta},\\
\cot\theta&\equiv\frac{1}{\tan\theta}.\end{align*}\]

\[\csc\] 很多时候也记作 \[\text{cosec}\].

一个非常实用的关系, 直角三角形中有\textbf{勾股定理} (Pythagorean
theorem): 斜边边长平方等于两直角边边长的平方之和, 即 \[AC^2=AB^2+BC^2\];
两边同时除以 \[AC^2\] 便有

\begin{itemize}
\tightlist
\item
  \[\boxed{1=\cos^2\theta+\sin^2\theta}\].\footnote{三角函数的平方:
    cos(x)\textsuperscript{2} 通常理解为 cos((x)\textsuperscript{2});
    cos\textsuperscript{2}x 约定俗成表示 (cos(x))\textsuperscript{2}.}
\end{itemize}

\textbf{正弦定律 Law of sine}

将三角形三个角分别记作 \[\alpha\], \[\beta\], 和 \[\gamma\],
将它们的对边分别记作 \[A\], \[B\], 和 \[C\]. 先是结论:

\begin{itemize}
\tightlist
\item
  \[\boxed{\frac{A}{\sin\alpha}=\frac{B}{\sin\beta}=\frac{C}{\sin\gamma}}\].
\end{itemize}

推导如下:

如上图所示, 以 \[C\] 为底做高, 将原本的三角形分为左右两个直角三角形,
这条高利用左边的直角三角形可以表示为 \[A\sin\beta\],
利用右边的直角三角形则是 \[B\sin\alpha\], 于是有
\[A\sin\beta=B\sin\alpha\], 整理可得
\[\frac{A}{\sin\alpha}=\frac{B}{\sin\beta}\];
再做另一条高重复前面的操作, 便可得到完整的结论.

\textbf{余弦定律 Law of cosine}

还是先上结论:

\begin{itemize}
\tightlist
\item
  \[\boxed{B^2=A^2+C^2-2AC\cos\beta}\],
\end{itemize}

即, 【一条边的边长平方】等于【另两条边的边长平方之】和加上【两倍的
(另两条边边长的乘积) 乘以 (另两条边的夹角的余弦)】.

推导如下:

如下图所示, 依旧利用底边 \[C\] 上的高将其分为左右两个直角三角形;
左边的直角三角形, 利用斜边 \[A\] 和角 \[\beta\], 两直角边分别可以表示为
\[A\cos\beta\] 和 \[A\sin\beta\], 于是右边的直角三角形边长便可表述为
\[A\sin\beta\] 和 \[(C-A\cos\beta)\]; 对右边的直角三角形使用勾股定理

\[\begin{align*}B^2&=A^2\sin^2\beta+(C-A\cos\beta)^2\\
&=A^2\sin^2\beta+C^2+A^2\cos^2\beta-2AC\cos\beta\\
&=A^2+C^2-2AC\cos\beta.\end{align*}\]

其中等式的后两行用到了之前得出的 \[1=\cos^2\theta+\sin^2\theta\].

\hypertarget{ux4efbux610fux89d2ux5ea6ux7684ux4e09ux89d2ux51fdux6570}{%
\subsubsection{任意角度的三角函数}\label{ux4efbux610fux89d2ux5ea6ux7684ux4e09ux89d2ux51fdux6570}}

不难发现, 前面讨论的情况似乎都是锐角的情况 (主要是因为插图\ldots),
钝角的三角函数似乎没那么直观了, 因为做不成一个含有钝角的直角三角形,
没法简单地用边长比来表示 \[\sin\] 和 \[\cos\] 等. 于是,
我们需要想办法将前面的情形推广.

如下左图所示, 建立直角坐标系, 做一圆心位于原点的单位圆, 即半径为 \[1\]
的圆, 考虑在第一象限的圆上的一点, 将其与原点做连线, 将从
\[x\]-轴正方向与这条连线\textbf{顺时针}方向形成的夹角记作 \[\theta\],
不难看出这个点的坐标 \[(x,y)\] 满足

\[\begin{cases}x=\cos\theta,\\y=\sin\theta.\end{cases}\]

于是不妨将其他象限的情况也按此定义,
于是如上右图所示的钝角甚至更大角度的三角函数便可以被定义了.

\hypertarget{ux5f27ux5ea6ux5236-radian}{%
\subsubsection{弧度制 (Radian)}\label{ux5f27ux5ea6ux5236-radian}}

为什么一个周角是 \[360^\circ\] 呢, 听说过一个不可考的说法: \[360\]
是一个有很多因数的数字 (1, 2, 3, 4, 5, 6, 8, 9, 10, 12\ldots),
等分起来的时候数字会比较友好, 所以 \[360^\circ\] 其实是非常随意地规定的.
那么有没有更好的用来描述角度方法呢? 答案是弧度.

一个半径为 \[r\] 的圆的周长是 \[2\pi r\], 一个圆心角为 \[n^\circ\]
的扇形的弧长是 \[2\pi r\frac{n}{360}\]. 可见圆心角越大弧越长,
且圆心角和弧长成正比. 既然如此,
不如重新将角度定义为圆心角与弧长的比值以方便计算, 于是便有了,
在新的这套单位系统中, 若圆心角大小为 \[\theta\], 其对应弧长应为
\[r\theta\]; 当圆心角是一个周角时, 对应弧长便成了圆的周长 \[r(2\pi)\].
所以角度和这个新的单位的换算有 \[360^\circ\equiv 2\pi\ \text{rad}\],
因为这个单位把圆心角和对应的弧长联系起来了, 因此称之为\textbf{弧度}
(radian).

扇形面积在这套单位制, 即弧度制下, 便也成了 \[\frac{1}{2}r^2\theta\].

\hypertarget{ux4e09ux89d2ux51fdux6570ux7684ux56feux50cf}{%
\subsubsection{三角函数的图像}\label{ux4e09ux89d2ux51fdux6570ux7684ux56feux50cf}}

现在这个时代, 大家都或多或少能接触到科学计算器,
再不济在bing.com上搜索''solver''用微软的 Microsoft Solver
也可以计算某个特定角度的三角函数值, 自然也可以绘制函数图像.
下图分别展示了 \[\sin(x)\] 和 \[\cos(x)\] 的图像,

一些值得关注的点是它们都是\textbf{周期函数} (periodic function),
随着自变量-角度的变化, 因变量-函数值的变化是周期性的, 它们的周期都是
\[2\pi\], 这一点从上文的单位圆里便可看出些许原因,
当角度变化超过一个周角时, 和角度刚从 \[0\] 开始的情况是一样的.

一个个人很喜欢的可视化如下:

右下显示的是角度 \[\theta\] 不断增加, 左下的图可以看作右下的点的 \[y\]
坐标也就是 \[\sin\theta\] 的值的变化, 左上则是 \[x\] 坐标也就是
\[\cos\theta\] 的值的变化.
