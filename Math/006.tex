一点点绕行 (detour).

\hypertarget{ux865aux6570ux548cux590dux6570}{%
\subsubsection{虚数和复数}\label{ux865aux6570ux548cux590dux6570}}

考虑一个一元二次方程 \[ax^2+bx+c=0\], 它的解有 \[
\begin{align*}
0&=a\left(x^2+\frac{b}{a}x+\frac{c}{a}\right)\\
0&=x^2+\frac{b}{a}x+\frac{c}{a}\\
0&=x^2+\frac{b}{a}x+\left(\frac{b}{2a}\right)^2-\left(\frac{b}{2a}\right)^2+\frac{c}{a}\\
0&=\left(x-\frac{b}{2a}\right)^2-\left(\frac{b}{2a}\right)^2+\frac{c}{a}\\
\left(x-\frac{b}{2a}\right)^2&=\left(\frac{b}{2a}\right)^2-\frac{c}{a}\\
&...\\
x&=\boxed{\frac{-b\pm\sqrt{b^2-4ac}}{2a}}.
\end{align*}
\]

上式最后的结论便是求根公式, 不难看出整个推导过程实际上就是配方,
其中根号前面的''加减''是因为对等式两边同时开方时,
正负两种情况都是正确的.

我们在此之前接触到的数字都还限于实数范围内, 因此会要求
\[\left(b^2-4ac\right)\] 是正的, 以保证开方之后的结果是''有意义的'',
然而''从来如此, 便对么?''

之前也出现了, 不能被表示成分数形式的数字,
我们的研究范围从有理数扩充到了实数; 现在, 若 \[\left(b^2-4ac\right)\]
是负的, 按照当前的理解, 它不能被开方, 那是不是又到了这样一个神圣的时刻,
我们需要拓展我们研究的数字的范围?

既然如此, 不如规定 \[\sqrt{-1}\equiv i\], 作为新的一类数字的单位,
因为之前的数字叫''实数'', 那么这一类新的数字就不妨叫做''\textbf{虚数}''
(imaginary number) 吧. 一个既包含实数部分, 又包含虚数的部分的数字,
我们就叫它''\textbf{复数}'' (complex number), 记作 \[\mathbb{C}\].

\textbf{运算规律}

考虑若干个复数, \[z_1=a+bi\], \[z_2=c+di\], \[z_3=e+fi\]\ldots{}

\begin{itemize}
\tightlist
\item
  \textbf{加法}: \[z_1+z_2=(a+c)+(b+d)i\].
  实数部分和虚数部分可以分开计算,
  应该不难看出复数和加法是构成\textbf{阿贝尔群}的
  (即它具有封闭性和结合律, 有单位元和逆元, 并且有交换律, 详细参见001).
\item
  \textbf{乘法}:
  \[z_1\times z_2=(a+bi)\times(c+di)\\=ac+adi+bci+bdi^2=(ac-bd)+(ad+bc)i.\]
  不难看出, 复数和乘法也构成阿贝尔群.
\item
  乘法对于加法满足\textbf{分配律}, 即,
  \[(z_1+z_2)\times z_3=z_1\times z_3+z_2\times z_3\],
  证明留作练习\footnote{事实上, 很多情况下, 之前提到的很多知识点,
    例如单位元, 零元, 逆元都分左右, 分配律也有左分配律和右分配律,
    但是目前讨论的情况都是满足交换律的, 所以可以不区分左右.}.
\end{itemize}

以上三条已经足够使得复数与加法和乘法构成一个\textbf{环} (ring), 事实上,
环只需要乘法是半群 (semi-group, 满足结合律和有单位元) 即可.

\begin{itemize}
\tightlist
\item
  \textbf{减法}: 因为加法存在逆元, 所以减去一个数,
  可以视作加上这个数的加法逆元, 即:
  \[z_1-z_2=(a+bi)-(c+di)\\\Rightarrow z_1+(-z_1)=(a+bi)+(-(c+di))=(a-c)+(b-d)i\]
\item
  \textbf{除法}: 不难发现每个非零的元素都有乘法逆元,
  因此除以一个数可以视作乘上这个数的乘法逆元, 即: 因为
  \[z_2\times\frac{1}{z_2}=\frac{c+di}{c+di}=1\], 于是
  \[z_1\div z_2=z_1\times\frac{1}{z_2}=\frac{a+bi}{c+di}\].
\end{itemize}

\begin{quote}
一点小插曲, \[\frac{a+bi}{c+di}\] 应该怎么化简呢,
怎么写成简单的实数部分加上虚数部分的形式呢?
回顾一下无理数的''分母有理化'', 例如有
\[\frac{a+\sqrt{b}}{c+\sqrt{d}}\], 我们会将分子分母同时乘以
\[c-\sqrt{d}\] 讲分母变为有理数, 便有
\[\frac{(a+\sqrt{b})(c-\sqrt{d})}{c^2-d}\]. 类似的, 当我们尝试化简
\[\frac{a+bi}{c+di}\]时, 我们也不妨对分子分母同时乘以 \[c-di\], 于是有
\[\frac{(a+bi)(c-di)}{c^2+d^2}\], 分母便变为了实数,
再稍加化简便可转化为一个实数加上一个虚数的形式. 我们称 \[c-di\] 是
\[c+di\] 的\textbf{复共轭} (complex conjugate)\footnote{两头牛背上的架子称为轭,
  轭使两头牛同步行走. 共轭就描述了两个对象这样一种相生相随的关系.}.
\end{quote}

像上述这样可以进行加减乘和除零外除法,
并满具足一些特定的阿贝尔群的特点和分配律的代数结构,
换言之一个满足交换律的环 (交换环 commutative ring)
附加上除零外元素的除法运算, 构成一个\textbf{域} (field), 可以记作
\[\mathbb{F}\], 常见的例子有有理数域, 实数域, 复数域.

环, 交换环, 域的关系
\href{https://www.zhihu.com/people/SparkandShine}{SparkAndShine - 知乎
(zhihu.com)} http://sparkandshine.net/
