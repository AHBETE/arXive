\begin{quote}
万物伊始 天地初开
\end{quote}

\hypertarget{ux52a0ux6cd5-addition}{%
\subsubsection{加法 (addition)}\label{ux52a0ux6cd5-addition}}

\(1+1=2\) , 好了, 这一节结束 (玩笑). 加法的运算, 九九加法表,
这里就不赘述.

我们首先考虑\textbf{整数} (integer), 记作 \(\mathbb{Z}\),
当我们考虑一类数字或者一类符合某种性质的对象时,
我们称这一类对象组成的东西叫做\textbf{集合} (set),
集合中的东西便叫做\textbf{元素} (element).

\begin{itemize}
\tightlist
\item
  不难发现, 从整数里取出两个元素, 例如 \(1\) 和 \(2\) , 将他们相加,
  得到的结果 \(1+2=3\) 依旧是整数.
  这样的性质叫做\textbf{封闭性}或者\textbf{闭包性} (closed, closure).
\item
  再考虑三个元素在加法下的运算. 三个元素的运算可以看作,
  两个元素的运算结果与第三个元素再次运算, 还是不难发现,
  任意从整数取出三个元素, 例如 \(1\) , \(2\) 和 \(3\) , 有
  \((1+2)+3=6=1+(2+3)\) ,
  即前两个元素先进行运算和后两个元素先进行运算的结论时一致的.
  这样的性质叫做\textbf{结合律} (assosiative).
\item
  整数里存在一个特殊的元素,
  使得加法这个运算不对其他任何元素''产生效果'', 这个特殊的元素是 \(0\) ,
  \(0+n=n+0=n\) , 这里 \(n\) 是任意整数, 我们可以这么标记:
  \(n\in\mathbb{Z}\) , 中间这个符号表示属于 (belong to).
  这个特殊的元素叫做\textbf{单位元}或\textbf{幺元} (identity
  element)\footnote{单位和幺都有一的含义, 因为在乘法中单位元是1,
    这可能是名字来源.}.
\item
  整数的加法中, 对于任何一个元素, 都能找到另一个元素,
  使得它们运算结果为单位元- \(0\) , 比如 \(1+(−1)=(−1)+1=0\) , 我们便叫
  \((−1)\) 是 \(1\) 的\textbf{逆元} (inverse element).
\end{itemize}

一个集合, 再附加一个二元运算(像加法这样输入两个元素输出一个元素的运算),
并且拥有上述性质和元素的, 我们便把它叫做\textbf{群} (group), 整数和加法,
便是这样构成了一个群 \((\mathbb{Z},+)\) .

好的,我们在学习加法的过程中顺便体验了以下群论. 要注意的是,
上文并没有强调\textbf{交换性} (commutative), 因为往后看我们会发现,
很多运算其实并不满足交换律,
满足交换律的群我们可以称它为\textbf{交换群}或\textbf{阿贝尔群} (Abelian
group).

\begin{quote}
有人问一个小朋友, ``3+4 等于几啊?'' 小朋友说: ``不知道, 但我知道 3+4
等于 4+3.'' 那人接着问: ``为什么呀?''
答曰:``因为整数与整数加法构成了阿贝尔群.''
这个笑话讽刺了某次法国一场幼儿园从抽象数学教起的实验,
不过最后实验的结果是以失败告终.
\end{quote}

\hypertarget{ux51cfux6cd5-subtraction}{%
\subsubsection{减法 (subtraction)}\label{ux51cfux6cd5-subtraction}}

思路要打开, 减法可以看作是加法的逆运算; 又或者, 减去一个数,
可看作加上这个数字的逆元.

减法的一些性质:

\begin{itemize}
\tightlist
\item
  \textbf{反交换律} (anti-commutativity), 例如 \(4−3=−(3−4)\) ,
  交换两个元素的顺序会导致结果变为之前结果的逆.
\item
  \textbf{非结合律} (non-associativity), 例如 \((6−3)−2\neq 6−(3−2)\) .
\end{itemize}

因为整数减法不满足结合律, 所以整数和减法不构成群, 只构成\textbf{拟群}
(quasi-group), 字面上可以理解成, 像群, 但是不是群, 这里不做展开.

\hypertarget{ux4e58ux6cd5-multiplication}{%
\subsubsection{乘法
(multiplication)}\label{ux4e58ux6cd5-multiplication}}

还是九九乘法表, 结束了 (玩笑). 乘法可以视作是多个同样加法的标记, 例如:
\(2×3=2+2+2=3+3=3×2\) .

来看看乘法的一些性质:

\begin{itemize}
\tightlist
\item
  易见\textbf{封闭性}或者\textbf{闭包性}是满足的.
\item
  也不难看出乘法具有\textbf{结合律}.
\item
  乘法的\textbf{幺元}是 1 .
\item
  再\textbf{逆元}上似乎出了问题, 到目前为止, 我们讨论的都还是整数
  \(\mathbb{Z}\) , 这个范围内, 似乎找不到逆元, 但是没有关系,
  我们把范围扩大到非零\textbf{有理数} (rational number)\footnote{有理数其实是谬译,
    rational在这里其实意为可约的, 而不是有理的.}, 记作
  \(\{\mathbb{Q}/\{0\}\}\) , 这样每一个元素 \(n\in\{\mathbb{Q}/\{0\}\}\)
  都有逆元 \(\frac{1}{n}\) . 将 \(0\) 剔除是因为它没有逆元, 我们应该知道
  0 不能作为分母.
\end{itemize}

以上性质已经决定了非零有理数和乘法构成群, \((\mathbb{Q}/\{0\}\,×)\) .
乘法另外还有特性:

\begin{itemize}
\tightlist
\item
  乘法与加法的混合运算, 会有\textbf{分配律} distributive property, 例如
  \(2×(3+4)=2×3+2×4\) .
\item
  任何数乘上 \(0\) 得到 \(0\) , \(0\) 可以称作乘法的\textbf{零元} (zero
  element), 零元没有逆.
\end{itemize}

\hypertarget{ux9664ux6cd5-division}{%
\subsubsection{除法 (division)}\label{ux9664ux6cd5-division}}

乘法和除法的关系类似加法和减法的关系. 除以零在大多数场景下是不被定义的.
