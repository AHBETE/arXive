\begin{quote}
\textbf{虚}中有\textbf{实}者, 或山穷水尽处, 一折而豁然开朗. - 沈复
『浮生六记』 Lesson 5: 最短的捷径就是绕远路,
绕远路才是我的最短捷径.\footnote{\emph{ジョジョの奇妙な冒険 Part7
  スティール・ボール・ラン}.}
\end{quote}

\hypertarget{ux6781ux5750ux6807-polar-coordinate}{%
\subsubsection{极坐标 (polar
coordinate)}\label{ux6781ux5750ux6807-polar-coordinate}}

其实在【005】中已经借用了一点极坐标的概念, 这里再正式地介绍一遍.
在平面直角坐标系里, 一个点所在的位置具有两个自由度 (degree of freedom),
因此一般不多不少需要两个独立变量来锚定这个点, 通常我们会有一个点的
\(x\)-轴坐标和 \(y\)-轴坐标来表示这个点的位置 \((x,y)\). 当然,
我们也可以在建立坐标系后, 利用和原点的距离 \(r\), 和一个方向,
例如【\(x\)-轴正方向】至【这个点和原点的连线】顺时针方向形成的夹角
\(\theta\), 来表示一个点的位置 \((r,\theta)\).

不难看出, 存在以下的转换

\(\begin{cases}x=r\cos\theta\\y=r\sin\theta\end{cases};\ \begin{cases}r=\sqrt{x^2+y^2}\\\theta=\arctan (y/x)\end{cases}.\)

上式中 \(\arctan\) 是 \(\tan\) 的逆运算, 即
\(\tan\theta=y/x\Rightarrow \theta=\arctan (y/x)\), 有时 \(\arctan\)
也记作 \(\tan^{-1}\).

\hypertarget{ux590dux5e73ux9762-complex-plane}{%
\subsubsection{复平面 (complex
plane)}\label{ux590dux5e73ux9762-complex-plane}}

考虑实数的时候, 有时我们会想象有一条数轴, 在【001】和【002】中,
我们将这条数轴添上了最初离散分布的整数,
然后又补上了似乎没有空隙的有理数, 最后才用全体实数彻底''填满''了.
在【006】中, 我们研究的对象再次扩展到了复数,
于是一条实轴似乎''放不下''这些复数的存在了, 我们便添加一条额外的,
与之前实数轴垂直的虚数轴, 如此一来便构成了一个复平面.

那么考虑一个复数 \((a+bi)\), 它的实部大小便是 \(a\), 虚部大小便是 \(b\),
仿照着实二维平面直角坐标系, 我们便可以在复平面上标出 \((a+bi)\)
对应的一个点. 既然如此, 我们可以继续仿照着极坐标, 表示出至原点的距离,
以及 \(x\)-轴正方向到这个点和原点的连线顺时针方向形成的夹角,
这两个量在复平面中分别被称作\textbf{模} (modulus, 合理怀疑有音译成分)
和\textbf{辐角} (argument), 记作

\$ \textbar a+bi\textbar=\sqrt{a^2+b^2},~\arg(a+bi)=\arctan (b/a).\$

\begin{quote}
像, 太像了.\footnote{\emph{让子弹飞}.}
\end{quote}

若是把模和辐角分别记作 \(r\) 和 \(\theta\), 则这个复数 \((a+bi)\)
便可以表示为

\((a+bi)=r\cos\theta+ir\sin\theta.\)

\hypertarget{ux590dux6570ux7684ux6307ux6570ux5f62ux5f0f}{%
\subsubsection{复数的指数形式}\label{ux590dux6570ux7684ux6307ux6570ux5f62ux5f0f}}

有那么一点跳脱, 怎么突然扯到指数了呢? 在【002】中,
其实我们只讨论过指数为有理数的情况, 当然指数是任意实数的情况, 例如
\(\sqrt{2}\) , 我们也可以利用 \(\sqrt{2}=1.414213...\) 去估算,
或者丢给一个科学计算器; 但是当指数是虚数或者复数时,
似乎情况就不大一样了\ldots 考虑自然常数为底数, 我们从自然常数的定义出发,
【008】中我们有

\(\lim_{n\rightarrow\infty}\left(1+\frac{1}{n}\right)^n\equiv\mathrm{e}.\)

不难推出, 对于任意实数 \(x\), 有

\(\lim_{n\rightarrow\infty}\left(1+\frac{x}{n}\right)^n=\mathrm{e}^x.\)

把上面结论推广到虚数,

\(\lim_{n\rightarrow\infty}\left(1+\frac{i}{n}\right)^n=\mathrm{e}^i.\)

如果上式让您感到不适 (uncomfortable), 您大可用二项式展开 (参见【007】)
来确认一下上述结果的正当性 (validity). 怎么理解这个 \(\mathrm{e}^i\) 呢,
不妨来看看它的模和辐角.

在此之前, 我们还需要一些小\textbf{引理} (lemma \footnote{\textbf{公理/假定}
  (axiom/postulate): 默认为真无需证明的陈述. \textbf{定义} (definition):
  准确无歧义的对一个术语的描述. \textbf{定理} (theorem):
  证明为真的大结论. \textbf{引理} (lemma): 为了证明定理的小结论.
  \textbf{推论} (Corollary): 借助定理可简短地证明的结论.
  一本严格的数学书经常会出现前面这些令人畏惧的词,
  类似的还有\textbf{命题} (proposition), \textbf{推测/猜想}
  (conjecture), \textbf{断言} (claim)等等.}, 区别于 llama - 大羊驼,
好冷), 这里没有严谨证明, 仅提供一个思路:

\textbf{引理1}: \(\boxed{|(a+bi)^n|=|a+bi|^n}\).

\begin{quote}
当 \(n=2\) 时, \((a+bi)^2=a^2-b^2+2abi\). 于是

\(\begin{aligned}|(a+bi)^2|=&\sqrt{(a^2-b^2)^2+(2ab)^2}\\=&\sqrt{(a^2+b^2)^2}\\=&\sqrt{(a^2+b^2)}^2\\=&|a+bi|^2.\end{aligned}\)

用数学归纳法 (关于数学归纳法, 可以参见【007】中的一个实例)
便不难得出结论.
\end{quote}

\textbf{引理2}: \(\boxed{\arg((a+bi)^n)=n\cdot\arg(a+bi)}\).

\begin{quote}
这里需要引入一个新的视角, 我们把复数看作一个''作用'',
将一个复数作用在某一个数 \(z\) 上, 在复平面上事实上是将 \(z\)
对应的点关于原点旋转了. 考虑 \(i\times1\), 即将 \(i\) 作用到 \(1\) 上,
得到了 \(i\), 相当于逆时针旋转了 \(90^\circ\); 继续作用 \(i\), 得到了
\(-1\), 相当于又逆时针旋转了 \(90^\circ\)\ldots{} 于是用 \(i\) 作用
\(n\) 次, 即作用了 \(i^n\), 相当于将其作用对象, 关于原点, 旋转了 \(i\)
的辐角 \(90^\circ\) \(n\) 次, 便是旋转了 \(n90^\circ\). ( \(90^\circ\)
用 \(\pi/2\) 弧度表述其实会更好, 关于弧度可以参考【005】,
下文若无额外说明, 角度皆用弧度制).

这个结论可以推广到其他任意复数, 便有了上述结论.
\end{quote}

事实上, 利用辐角和模来表述一个复数, 上述两则引理可以总结成一条:

\textbf{引理1+2}:
\(\boxed{\left[r(\cos\theta+i\cos\theta)\right]^n=r^n(\cos n\theta+i\sin n\theta)}\).

那么来看 \(\mathrm{e}^i\) , 它的模

\(\begin{aligned} |\mathrm{e}^i|=&\left|\lim_{n\rightarrow\infty}\left(1+\frac{i}{n}\right)^n\right|\\ =&\lim_{n\rightarrow\infty}\left|\left(1+\frac{i}{n}\right)^n\right|\\ =&\lim_{n\rightarrow\infty}\sqrt{1+\frac{1}{n^2}}^n\\ =&\lim_{n\rightarrow\infty}\left(1+\frac{1}{n^2}\right)^{n/2}\\ =&\lim_{n\rightarrow\infty}\left[\left(1+\frac{1}{n^2}\right)^{n^2}\right]^{1/2n}=\lim_{n\rightarrow\infty}\mathrm{e}^{1/2n}=1. \end{aligned}\)

上式第二行到第三行利用了引理1, 最后一行先是利用了自然常数的定义, 然后当
\(n\rightarrow\infty\) 便有 \(1/2n\rightarrow0\), 于是
\(\mathrm{e}^0=1\). 再来看辐角

\(\begin{aligned} \arg(\mathrm{e}^i)=&\lim_{n\rightarrow\infty}n\arg\left(1+\frac{i}{n}\right)\\ =&\lim_{n\rightarrow\infty}n\arctan\frac{1}{n}=\lim_{n\rightarrow\infty}n\frac{1}{n}=1.\\ \end{aligned}\)

上式先是利用了引理2, 然后当 \(n\rightarrow\infty\) 时
\(1/n\rightarrow0\), 然后在这个极限下 (啊, 还是,
极限我们晚点再稍严格地讨论) 有
\(\arctan\frac{1}{n}\rightarrow\frac{1}{n}\).

\begin{quote}
当然也可以用\textbf{小角度近似} (small angle approximation) 来理解
(注意, 是理解不是证明) 这个过程. 小角度近似即, 使用弧度制时, 当
\(\theta\ll1\), 有 \(\theta\approx\sin\theta\approx\tan\theta\).
\end{quote}

这样利用模和辐角, 我们有

\(\mathrm{e}^i=\cos1+i\sin1\).

所以 \(\mathrm{e}^i\) 在复平面上对应一个距离原点 \(1\), 与原点连线和
\(x\)-轴夹角是 \(1\) 的点, 或者说 \(\mathrm{e}^i\) 是一个实部是
\(\cos1\), 虚部是 \(\sin1\) 的复数. 同理可得 \(|\mathrm{e}^{ib}|=1\),
\(\arg(\mathrm{e}^{ib})=b\), 进而

\(\mathrm{e}^{ib}=\cos b+i\sin b\).

早一些地结论
\(\lim_{n\rightarrow\infty}\left(1+\frac{i}{n}\right)^n=\mathrm{e}^i\)
其实可以推广到任意复数 \((a+ib)\), 即
\(\lim_{n\rightarrow\infty}\left(1+\frac{a+ib}{n}\right)^n=\mathrm{e}^{a+ib}=\mathrm{e}^{a}\mathrm{e}^{ib}\),
那么

\(\boxed{\mathrm{e}^{a+ib}=\mathrm{e}^a(\cos b+i\sin b)}\)\footnote{a+ib
  取 iπ 便可以得到著名的欧拉公式, 确实非常美丽, 虚实正负在此交汇,
  圆周率和自然常数也藏于其中.}.
