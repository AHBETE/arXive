\begin{quote}
天 工 开 物 !

「Made in Heaven!」要开始加速了\footnote{ジョジョの奇妙な冒険 Part 6
  ストーンオーシャン}
\end{quote}

跳过一些内容: 集合 (set), 点集拓扑 (point set topology), 数列与级数
(sequence and series); 如果您希望习得更加严谨的数学语言, 那么可以移步
Walter Rudin - \emph{Principles of Mathematical Analysis} (俗称 baby
Rudin\footnote{Baby = 数学分析原理 \emph{Principles of Mathematical
  Analysis}; Papa/Big = 实分析与复分析 \emph{Real and Complex Analysis};
  Grandpa = 泛函分析 \emph{Functional Analysis}; 好好地学严格的数学,
  逃不掉这三本分析, 这个系列也就看个乐子.}).

\hypertarget{ux6781ux9650-limit}{%
\subsubsection{极限 (limit)}\label{ux6781ux9650-limit}}

\textbf{定义}: 当 \[x\] 趋向于 \[p\] 时, \[x\rightarrow p\] , \[f(x)\]
趋向于 \[q\], \[f(x)\rightarrow q\] , 记作
\[\lim_{x\rightarrow p}f(x)=q.\]

用 \[\epsilon - \delta\] 语言 (出现了!) 来说,
\[\lim_{x\rightarrow p}f(x)=q\] 便是: 对于任意的 \[\epsilon>0\] , 存在
\[\delta>0\] 使得【若 \[0<|x-p|<\delta\], 便有 \[|f(x)-q|<\epsilon\]
】\footnote{其实这里的定义还是很不严谨, 比如没有说明定义域和值域.
  为了省笔墨下文都默认取值范围在合适的区间内.}.

为什么要用 \[\epsilon - \delta\] 语言? 原本的''趋向于''其实很不严格,
什么叫趋向于呢, 于是 \[\epsilon - \delta\] 语言如是说道:

\begin{itemize}
\tightlist
\item
  我先任意选定一个 \[\epsilon\],
\item
  然后我要试着找到一个 \[\delta\],
\item
  使得 \[x\] 与 \[p\] 足够接近时 - 有多接近呢? 它们差的绝对值
  (或者说''距离'', 不过这边还没定义距离, 233) 小于 \[\delta\] - 便有
  \[f(x)\] 与 \[q\] 足够接近 - 多近呢? 他们差的绝对值小于 \[\epsilon\].
\item
  若对于任意小的 \[\epsilon\], 总能找到一个这样的 \[\delta\],
  那么便可以放心地说, 确有\[\lim_{x\rightarrow p}f(x)=q\].
\end{itemize}

\begin{quote}
\textbf{例子}: 一个平凡的情况 (a trivial case), \[f(x)=ax\], 证明
\[\lim_{x\rightarrow 1}f(x)=a\].

\textbf{思路} (草稿纸上或者脑子里的部分): 对于任意 \[\epsilon>0\]
我们需要找到 \[\delta>0\], 满足当 \[0<|x-1|<\delta\] 时,
\[|f(x)-a|<\epsilon\] 成立.

\begin{itemize}
\item
  \[|f(x)-a|=|ax-a|=a|x-1|\]
\item
  令上式小于 \[\epsilon\], 发现有 \[|x-1|<\epsilon/a\]
\item
  令 \[\delta=\epsilon/a\] , 即可出锅食用 (bushi).
\end{itemize}

\textbf{证明} (写下来的正式的书面的部分): 对任意 \[\epsilon>0\] , 令
\[\delta=\epsilon/a\], 则当 \[0<|x-1|<\delta\] 时, 有

\begin{itemize}
\tightlist
\item
  \[|f(x)-a|=|ax-a|=a|x-1|<a\delta<\epsilon\].
\item
  于是根据极限的定义, \[\lim_{x\rightarrow 1}f(x)=a\]
\end{itemize}

Q.E.D\footnote{Quod erat demonstrandum - 这被证明了.}

\textbf{吐槽}: 鄙人学分析的时候学得就很不到位,
写证明的时候常常向同学''借鉴'', 时常觉得, 证明本身并不难写,
难得是想到并''构造''出一些证明需要的东西, 就如在上面的例子中构造一个
\[\delta=\epsilon/a\];
殊不知''借鉴''的那些作业其实只有上面例子中【证明】的部分,
而【思路】部分被写在草稿纸上丢掉了.
这种狡猾如雪地上的狐狸一般用尾巴扫去自己的踪迹的行为\ldots{}
于是有这样的说法:

一位菲尔兹得主告诉我, 顶级的数学家们会秘密地像物理学家一样思考,
等他们得到证明的一个大框架之后, 他们再用 epsilon 和 delta
的语言把证明过程包装起来.
\end{quote}

\begin{itemize}
\tightlist
\item
  若 \[f(x)\] 在 \[x\rightarrow p\] 处存在极限,
  这个极限是\textbf{唯一}的 (unique).
\end{itemize}

考虑 \[\lim_{x\rightarrow p}f(x)=a\], \[\lim_{x\rightarrow p}g(x)=b\],
极限还存在以下规律 (啊, 美好的线性) :

\begin{itemize}
\tightlist
\item
  \[\lim_{x\rightarrow p}(f\pm g)(x)=a\pm b\];
\item
  \[\lim_{x\rightarrow p}(fg)(x)=ab\];
\item
  \[\lim_{x\rightarrow p}\frac{f}{g}(x)=\frac{a}{b}\], 若 \[b\neq 0\].
\end{itemize}

\begin{quote}
回收一个坑. 【002】中提到过 \[\mathrm{0}^0\] 是未被定义的问题,
一种解释便是, 若希望用极限来定义它的取值, 那么应该从
\[\lim_{x\rightarrow0}x^0\] 出发, 得到 \[1\], 还是应该从
\[\lim_{x\rightarrow0}0^x\] 出发, 得到 \[0\]?
\end{quote}

\hypertarget{ux8fdeux7eedux6027-continuity}{%
\subsubsection{连续性
(continuity)}\label{ux8fdeux7eedux6027-continuity}}

\textbf{定义}: 对于一函数 \[f(x)\], 对于任意的 \[\epsilon>0\] , 存在
\[\delta>0\] 使得【对于某个特定的 \[x_0\], 若有 \[x\] 满足
\[|x-x_0|<\delta\], 便有 \[|f(x)-f(x_0)|<\epsilon\]】, 那么我们便可以说,
\[f(x)\] 在 \[x_0\] 处连续.

\begin{itemize}
\tightlist
\item
  若 \[f(x)\] 和 \[g(x)\] 连续, 那么 \[f(g(x))\] 也连续.
\item
  若 \[f(x)\] 和 \[g(x)\] 连续, 那么 \[(f\pm g)(x)\], \[fg(x)\],
  \[\frac{f}{g}(x)\] 都连续, 最后一条要求 \[g(x)\] 对于任意 \[x\] 不为
  \[0\].
\end{itemize}

\textbf{极值定理 (extreme value theorem)}

若函数 \[f(x)\] 在区间 \[[a,b]\] 连续, 则 \[f(x)\] 必然在区间 \[[a,b]\]
存在最大值和最小值.

\textbf{介值定理 (intermediate value theorem)}

若函数 \[f(x)\] 在区间 \[[a,b]\] 连续, 且有 \[f(a)<C<f(b)\] 或
\[f(a)>C>f(b)\], 那么总是存在 \[c\in(a,b)\] 或者说 \[a\le c\le b\] 使得
\[f(c)=C\].

\textbf{零点定理 (zero theorem)}

若函数 \[f(x)\] 在区间 \[[a,b]\] 连续, 且\[f(a)f(b)<0\], 则存在
\[x_0\in(a,b)\] 使得 \[f(x_0)=0\].
