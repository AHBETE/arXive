\hypertarget{ux5e38ux89c1ux51fdux6570ux7684ux5bfcux6570}{%
\subsubsection{常见函数的导数}\label{ux5e38ux89c1ux51fdux6570ux7684ux5bfcux6570}}

\textbf{指数函数}

从指数函数开始, 首先考虑自然常数作为底数的情况, 代入导数的定义: \$\$

\begin{aligned}
\frac{\mathrm{d}}{\mathrm{d}x}(\mathrm{e}^x)=&\lim_{h\rightarrow0}\frac{\mathrm{e}^{x+h}-\mathrm{e}^x}{h}\\

=&\lim_{h\rightarrow0}\mathrm{e}^x\left(\frac{\mathrm{e}^h-1}{h}\right)\\
\\
&\text{令 }n:=1/h\\
\\
=&\lim_{n\rightarrow\infty}\mathrm{e}^x\left(\frac{\mathrm{e}^{1/n}-1}{1/n}\right)\\
\\
&\text{根据定义 }\mathrm{e}\equiv\lim_{n\rightarrow\infty}\left(1+\frac{1}{n}\right)^n\\
\\
=&\lim_{n\rightarrow\infty}\mathrm{e}^x\left(\frac{{\left(\left(1+1/n\right)^n\right)}^{1/n}-1}{1/n}\right)\\

=&\lim_{n\rightarrow\infty}\mathrm{e}^x\left(\frac{1+1/n-1}{1/n}\right)\\

=&\mathrm{e}^x.
\end{aligned}

\$\$

于是 \[
\boxed{\frac{\mathrm{d}}{\mathrm{d}x}\mathrm{e}^x=\mathrm{e}^x}.
\]

对于其他的底数, 例如 \(a^x\), 我们需要一些额外的知识了:

\textbf{链式法则 chain rule}

结论上有: \[
\boxed{\frac{\mathrm{d}y}{\mathrm{d}x}=\frac{\mathrm{d}y}{\mathrm{d}u}\frac{\mathrm{d}u}{\mathrm{d}x}}.
\]

\begin{quote}
即原先有一个函数 \(y=h(x)\), 将它改写为一个符合函数 \(y=f(g(x))\), 并令
\(u:=g(x)\).

不严格的直觉上的证明:

\(\Delta u=g(x+\Delta x)-g(x)\), \(\Delta y=f(u+\Delta u)-f(u)\). 于是
\(\frac{\Delta y}{\Delta x}=\frac{\Delta y}{\Delta u}\frac{\Delta u}{\Delta x}\).
再取 \(\Delta x\rightarrow0\) 的极限, 若 \(g(x)\) 是连续的, 在
\(\Delta x\rightarrow0\) 时便有 \(\Delta u\rightarrow 0\),
于是便得到了上述结论.
\end{quote}

现在我们再来看 \(a^x\), 我们可以先利用【002】的知识换个底数,
\(a^x=\mathrm{e}^{\ln(a)x}\), 再利用链式法则, 令 \(y:=\mathrm{e}^u\)
\(u:=\ln{(a)}x\), 于是 \$\$

\begin{aligned}
\frac{\mathrm{d}}{\mathrm{d}x}(a^x)=&\frac{\mathrm{d}y}{\mathrm{d}u}\frac{\mathrm{d}u}{\mathrm{d}x}\\

=&(\mathrm{e}^u)(\ln{a})\\

=&(\ln{a})\mathrm{e}^{\ln{a}}=(\ln{a})a^x.
\end{aligned}

\[
更加通常 (generalize) 一点, 我们可以有
\]
\boxed{\frac{\mathrm{d}}{\mathrm{d}x}a^{u(x)}=(\ln{a})a^u\frac{\mathrm{d}u}{\mathrm{d}x}}.
\$\$ \textbf{三角函数}

先考虑 \(\sin\), \$\$

\begin{aligned}
\frac{\mathrm{d}}{\mathrm{d}x}(\sin(x))=&\lim_{h\rightarrow0}\frac{\sin(x+h)-\sin(x)}{h}\\
\\
& \text{ 参见【010】, }\sin(a\pm b)=\sin a\cos b\pm \cos a\sin b\\
\\
=&\lim_{h\rightarrow0}\frac{\sin(x)\cos(h)+\cos(x)\sin(h)-\sin(x)}{h}\\

=&\lim_{h\rightarrow0}\left(\frac{\sin(x)(\cos(h)-1)}{h}+\frac{\cos(x)\sin(h)}{h}\right),
\end{aligned}

\$\$

到这一步似乎就不很直观了, 若是直接取 \(h=0\), 则有 \(\cos(h)-1=0\) 和
\(\sin(h)=0\), 两项的分子分母都同时为零了, 类似这种出现了
\(\frac{0}{0}\) 或者 \(\frac{\infty}{\infty}\) 的情况称为不定式/未定型
(indeterminate forms), 后面我们会看到将有一种更便 (简单) 捷 (粗暴)
的方式来解决这样的问题 (剧透: 洛必达法则 - L'Hôpital's rule),
目前我们先老老实实地来解决. 对于上面的形式, 我们可以用几何方法入手:

如上图所示, 考虑一个单位圆, 三角形 \(OAB\), 扇形 \(OAD\), 和三角形
\(OCD\) 的面积关系是 \(\sin(\theta)/2<\theta/2<\tan(\theta)/2\);
将每个式子都除以 \(\sin(\theta)/2\), 有
\(1<\theta/\sin(\theta)<1/\cos(\theta)\); 再取倒数, 得到
\(1>\sin(\theta)/\theta>\cos\theta\).
这个结论至少在第一象限应该是成立的, 于是当我们取 \(\theta=0\) 时,
我们发现不等式的最右边也是 \(1\) 了, 直觉上, 中间一项既要比 \(1\) 小,
又要比 \(1\) 大, 那么它只能是等于 \(1\) 了. 这个朴素的想法被叫做
``三明治定理'' (Sandwich Theorem, 也叫夹逼定理\ldots).

这样一来, 便有当 \(h\rightarrow0\), \(\sin(h)/h\rightarrow 1\),
另一项的话, 也不难利用三角函数的恒等式变形: \[
\begin{aligned}
&\frac{\cos(h)-1}{h}\\
\\
&\text{参见【010】}\cos2a=1-2\sin^2a\\
\\
=&-\frac{2\sin^2(h/2)}{h}\\
\\
&\text{令 }\theta:=h/2
\\
=&-\frac{\sin(\theta)}{\theta}\cdot\sin(\theta),
\end{aligned}
\]

于是当 \(h\rightarrow0\), \(\theta\rightarrow0\), \[
\begin{aligned}
&\frac{\cos(h)-1}{h}\\
=&-\frac{\sin(\theta)}{\theta}\cdot\sin(\theta)\\
=&-(1)\cdot(0)\\=&0.
\end{aligned}
\] 综上, \[
\boxed{\frac{\mathrm{d}}{\mathrm{d}x}(\sin(x))=\cos(x)}.
\] 类似的, 不难推出 \[
\boxed{\frac{\mathrm{d}}{\mathrm{d}x}(\cos(x))=-\sin(x)}.
\]
