\section{偏微分}\label{023}

\subsection{多变量函数}

先举一个偏日常的例子: 当我们看气象图的时候,
我们可以在地图上建立一个直角坐标系, 用类似 \((x,y)\)
这样的坐标来描述某地的位置, 每一个位置会对应一个温度 \(T\),
于是我们将两个实数映射到了一个实数
\(\mathbb{R}^2\rightarrow\mathbb{R}\), 这样一来, \(T(x,y)\) 是一个关于
\(\{x,y\}\) 这两个变量的函数.

\begin{newquote}
物理上, 我们也可以把这样一个函数叫做\textbf{标量场} (scalar field),
空间里的每一个点对应了一个标量.
\end{newquote}

稍微抽象一点的例子: 一个多变量函数也可以表示一个平面, 比如 \(z(x,y)\)
就可以描述一个在三维直角坐标系中的一个平面在各 \((x,y)\) 坐标的高度
\(z\); 这一层理解也可以推广到其他坐标系以及更高的维度.

\subsection{高维的极限与连续性}

考虑二维的\textbf{欧几里得空间} (Euclidean space,
后面可能会简称欧氏空间),

\begin{newquote}
这里强调了一下欧氏空间, 因为一个空间里的距离事实上并不只有我们最习惯的,
利用勾股定理计算``直线距离''这样一种定义, 在前面【\ref{011}\nameref{011}】定义一元函数时,
为了方便起见事实上我们忽略了距离的讨论; 在这里我们还是类似的,
先从最熟悉且的欧式的情况出发\ldots{}
\end{newquote}

\textbf{定义}: 如果对于任意 \(\epsilon>0\), 都存在 \(\delta>0\),
使得若有 \[
\sqrt{(x-x_0)^2+(y-y_0)^2}<\delta,
\] 便有 \[
|f(x,y)-L|<\epsilon,
\] 那么 \(L\) 就是 \(f(x,y)\) 在 \((x_0,y_0)\) 处的极限, 记作 \[
\lim_{(x,y)\rightarrow(x_0,y_0)}f(x,y).
\] \textbf{定义}: 如果 \(f(x,y)\) 在 \((x_0,y_0)\) 处是被定义的 (即
\(f(x_0,y_0)\) 存在), 且此处的极限也存在, 并且
\(\lim_{(x,y)\rightarrow(x_0,y_0)}f(x,y)=f(x_0,y_0)\), 那么可以说
\(f(x,y)\) 在 \((x_0,y_0)\) 是连续的.

\subsection{偏微分}

考虑函数 \(f(x,y)\), 它关于 \(x\) 的偏导定义为 \[
\frac{\partial}{\partial x}f(x,y)=\lim_{h\rightarrow0}\frac{f(x+h,y)-f(x,y)}{h},
\] 其中 \(\partial\) 是偏导的记号, 观察不难发现,
它作用的``效果''类似于导数的 \(\mathrm{d}\),
为了和导数区分所以使用了不同的记号.
\(\frac{\partial}{\partial y}f(x,y)\) 的定义也是类似的;
当函数拥有更多变量时, 它关于其他变量的偏导也是类似地去定义.

\begin{newquote}
实际计算中, 例如当我们计算 \(f(x,y)\) 关于 \(x\) 的的偏导时, 我们可以将
\(y\) ``视作''常数, 在计算它关于 \(y\) 的偏导时, 则可以将 \(x\)
``视作''常数.

例: 考虑 \(f(x,y)=xy+x/y\), 有 \(\partial f/\partial x=y+1/y\),
\(\partial f/\partial y=x+x\ln y\).
\end{newquote}

\subsection{二次偏导}

还是考虑函数 \(f(x,y)\), 对其关于 \(x\) 进行两次求导或关于 \(y\)
进行二次求导, 即 \(\partial^2f/\partial x^2\) 和
\(\partial^2f/\partial y^2\), 和一元函数的二次导没有太大区别;
但是二次偏导还存在``混合''的情况: 先关于 \(x\) 求导再关于 \(y\)
求导和先关于 \(y\) 求导再关于 \(x\) 求导, 即
\(\partial^2f/\partial x\partial y\) 和
\(\partial^2f/\partial y\partial x\),
那么这两个``混合''的偏导满足什么样的关系呢?

很多教授总说对于足够``好''的函数,
对不同变量求偏导的顺序应该是不影响结果的, 即
\(\partial^2f/\partial x\partial y=\partial^2f/\partial y\partial x\);
对于这样的``好''更严格的表述是:

若一个含有 \((x_0,y_0)\) 的开区间里, 或者说 \((x_0,y_0)\) 的一个领域里,
\(\partial f/\partial x\), \(\partial f/\partial y\),
\(\partial^2f/\partial x\partial y\) 和
\(\partial^2f/\partial y\partial x\) 都是被定义的, 那么有 \[
\left.\frac{\partial^2f}{\partial x\partial y}\right|_{(x,y)=(x_0,y_0)}=\left.\frac{\partial^2f}{\partial y\partial x}\right|_{(x,y)=(x_0,y_0)}.
\] 上面的关系也被叫做克莱罗定理 (Clairaut's theorem),
这样的关系也可以推广到更高阶的导数和含更多变量的函数.

\subsection{链式法则和与全微分的联系}

依旧考虑函数 \(f(x,y)\), 现在, 若 \(\{x,y\}\) 本身也是关于一个参数 \(t\)
的函数 \(\{x(t),y(t)\}\), 那么函数 \(f\) 关于 \(t\) 的导数
\(\mathrm{d}f/\mathrm{d}t\) 利用链式法则不难看出应该是 \[
\boxed{\frac{\mathrm{d}f}{\mathrm{d}t}=\frac{\partial f}{\partial x}\frac{\mathrm{d}x}{\mathrm{d}t}+\frac{\partial f}{\partial y}\frac{\mathrm{d}y}{\mathrm{d}t}}.
\] 上式反映了全微分和偏微分的关系, 热力学中会用到的麦克斯韦关系式
(Maxwell relation) 的底层逻辑便是它以及前面的克莱罗定理.
