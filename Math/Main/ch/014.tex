\section{常见函数的导数-下}\label{014}

\begin{tcolorbox}[size=fbox, breakable, enhanced jigsaw, title={逆函数和导数}]

有的时候我们会有一个函数的反函数 (关于反函数可以参见【\ref{004}\nameref{004}】),
而不是这个函数本身, 求这个函数的导数时, instead of
先求反函数的反函数得到原函数, 我们可以利用反函数本身来求这个导数,
\begin{align*}
f(f^{-1}(x))=&x\\
\text{(对两边求导)}\\
\frac{\mathrm{d}}{\mathrm{d}x}f(f^{-1}(x))=&1\\
\text{(利用链式法则)}\\
f'(f^{-1}(x))\cdot\frac{\mathrm{d}}{\mathrm{d}x}f^{-1}(x)=&1\\
\text{(rearrange)}\\
\boxed{\frac{\mathrm{d}}{\mathrm{d}x}f^{-1}(x)=\frac{1}{f'(f^{-1}(x))}}.
\end{align*}

\end{tcolorbox}

\begin{tcolorbox}[size=fbox, breakable, enhanced jigsaw, title={对数函数}]

这样一来, 因为我们已经知道对于指数函数有
$\frac{\mathrm{d}}{\mathrm{d}x}\mathrm{e}^x=\mathrm{e}^x$,
便可借助上式得到对数函数的导数了,
\begin{align*}
\frac{\mathrm{d}}{\mathrm{d}x}\ln(x)=&\frac{\mathrm{d}}{\mathrm{d}x}\exp^{-1}(x)\\
&\text{(利用上式结论, 令 }f^{-1}(x):=\ln(x),\\
&\text{ 于是 }f(x)=\exp(x), f'(x)=\exp(x)\text{ )}\\
=&\frac{1}{\exp(\ln(x))}\\
=&\frac{1}{x},
\end{align*}
因此 \begin{equation*}
\boxed{\frac{\mathrm{d}}{\mathrm{d}x}\ln(x)=\frac{1}{x}}.
\end{equation*} 当然, 也可以直接利用换元, 令 $y:=\ln(x)$, 于是有 $\exp(y)=x$,
于是
\begin{align*}
\frac{\mathrm{d}}{\mathrm{d}x}\exp(y)=&\frac{\mathrm{d}}{\mathrm{d}x}(x)\\
\exp(y)\frac{\mathrm{d}y}{\mathrm{d}x}=&1\\
\frac{\mathrm{d}y}{\mathrm{d}x}=&\frac{1}{\exp(y)}=\frac{1}{\exp(\ln(x))}=\frac{1}{x}.
\end{align*}
不难得出, 更通常而言, 有 \begin{equation*}
\boxed{\frac{\mathrm{d}}{\mathrm{d}x}\log_au=\frac{1}{\ln(a)\cdot u}\frac{\mathrm{d}u}{\mathrm{d}x}}.
\end{equation*}

\end{tcolorbox}

\begin{tcolorbox}[size=fbox, breakable, enhanced jigsaw, title={反三角函数}]

利用逆函数和导数的关系, 类似的, 不难推出,
\begin{align*}
\frac{\mathrm{d}}{\mathrm{d}x}\sin^{-1}(x)=&\frac{1}{\cos(\sin^{-1}(x))}\\
=&\frac{1}{\sqrt{1-\sin^2(\sin^{-1}(x))}}\\
=&\frac{1}{\sqrt{1-x^2}}.
\end{align*}
更通常的, \begin{equation*}
\boxed{\sin^{-1}(u)=\frac{1}{\sqrt{1-u^2}}\frac{\mathrm{d}u}{\mathrm{d}x}}.
\end{equation*} 和前面类似的, 也可以利用换元得到同样的结果, 考虑一个单位圆, 有
$\theta=\sin^{-1}(y)$, 于是 $\sin(\theta)=y$, 对两边同时求导, 有
$\cos(\theta)\frac{\mathrm{d}\theta}{\mathrm{d}y}=1$, 继而有
\begin{align*}
\frac{\mathrm{d}\theta}{\mathrm{d}y}=&\frac{1}{\cos(\theta)}\\
&\text{(在单位圆里有 }\cos(\theta)= x\text{)}\\
=&\frac{1}{x}\\
=&\frac{1}{\sqrt{1-y^2}}.
\end{align*}
这样的方法其实有利用隐函数 $x^2+y^2=1$,
之后我们还会看到隐函数的导数和微分方程有一些联系.

另一个常用的反三角函数的导数
$$\boxed{\begin{aligned}&\frac{\mathrm{d}}{\mathrm{d}x}\cos^{-1}(u)=-\frac{1}{\sqrt{1-u^2}}\frac{\mathrm{d}u}{\mathrm{d}x},\\
&\frac{\mathrm{d}}{\mathrm{d}x}\tan^{-1}(u)=\frac{1}{1+u^2}\frac{\mathrm{d}u}{\mathrm{d}x},\\
&\frac{\mathrm{d}}{\mathrm{d}x}\cot^{-1}(u)=-\frac{1}{1+u^2}\frac{\mathrm{d}u}{\mathrm{d}x},\\
&\frac{\mathrm{d}}{\mathrm{d}x}\sec^{-1}(u)=\frac{1}{|u|\sqrt{u^2-1}}\frac{\mathrm{d}u}{\mathrm{d}x},\\
&\frac{\mathrm{d}}{\mathrm{d}x}\mathrm{cosec}^{-1}(u)=-\frac{1}{|u|\sqrt{u^2-1}}\frac{\mathrm{d}u}{\mathrm{d}x}.\end{aligned}}$$

推导过程可以留作练习.

\end{tcolorbox}