\section{导数算符的乘法法则}\label{029}

\begin{newquote}
注: 这一节的讨论限制在欧氏三维空间搭配直角坐标系的情况.
\end{newquote}

我们知道一元函数的导数是线性的, 不难发现, 导数算符 $\nabla$
也应该是线性的, 即有 \[
\begin{aligned}
&\nabla(af+bg)=a\nabla f+b\nabla g,\\
&\nabla\cdot(a\boldsymbol{u}+b\boldsymbol{v})=a\nabla\cdot\boldsymbol{u}+b\nabla\cdot\boldsymbol{v},\\
&\nabla\times(a\boldsymbol{u}+b\boldsymbol{v})=a\nabla\times\boldsymbol{u}+b\nabla\times \boldsymbol{v};
\end{aligned}
\] 这里 $a,b$ 是两个常数; $f,g$ 是两个标量场;
$\boldsymbol{u},\boldsymbol{v}$ 是两个向量场.
可以利用分量运算来验证上述规则.

乘法法则则相对复杂, 我们可以有四种情况:

\begin{enumerate}
\def\labelenumi{\arabic{enumi}.}

\item
  两个标量场相乘得到一个标量场;
\item
  两个向量场点乘得到一个标量场;
\item
  一个标量场乘一个向量场得到一个向量场;
\item
  两个向量场叉乘得到一个向量场.
\end{enumerate}

若结果是标量场, 我们可以求它的梯度, 而对于向量场, 我们可以求旋度和散度;
省略掉一些计算上的细节 (还是, 可以考虑分量来验证), 有结论如下: \[
\boxed{\begin{aligned}
&\nabla(fg)=f\nabla g+g\nabla f,\\
&\nabla(\boldsymbol{u}\cdot\boldsymbol{v})=\boldsymbol{u}\times(\nabla\times\boldsymbol{v})+\boldsymbol{v}\times(\nabla\times\boldsymbol{u})+(\boldsymbol{u}\cdot \nabla)\boldsymbol{v}+(\boldsymbol{v}\cdot \nabla)\boldsymbol{u},\\
&\nabla\cdot(f\boldsymbol{v})=f(\nabla\cdot\boldsymbol{v})+\boldsymbol{v}\cdot(\nabla f),\\
&\nabla\cdot(\boldsymbol{u}\times\boldsymbol{v})=\boldsymbol{u}\cdot(\nabla\times\boldsymbol{v})-\boldsymbol{v}\cdot(\nabla\times\boldsymbol{u}),\\
&\nabla\times(f\boldsymbol{v})=f(\nabla\times\boldsymbol{v})-\boldsymbol{v}\times(\nabla f);\\
&\nabla\times(\boldsymbol{u}\times\boldsymbol{v})=(\boldsymbol{v}\cdot\nabla)\boldsymbol{u}-(\boldsymbol{u}\cdot\nabla)\boldsymbol{v}+\boldsymbol{u}(
\nabla\cdot\boldsymbol{v})-\boldsymbol{v}(
\nabla\cdot\boldsymbol{u}).
\end{aligned}}
\] 上面, 例如第二个和第五个式子, 都出现了导数算符 $\nabla$
没有\textbf{直接}作用在函数 (either 标量场 or 向量场) 上的情况,
初看可能会较为费解, 但是把导数算符写成
$\nabla=\partial_x\hat{\imath}+\partial_y\hat{\jmath}+\partial_z\hat{k}$
这样一个类似``向量''的东西再运算就明确很多了.

\subsection{二次导}

\textbf{梯度的散度} \[
\begin{aligned}
\boxed{\nabla\cdot(\nabla f)}&=(\partial_x\hat{\imath}+\partial_y\hat{\jmath}+\partial_z\hat{k})\cdot\left(\frac{\partial f}{\partial x}\hat{\imath}+\frac{\partial f}{\partial y}\hat{\jmath}+\frac{\partial f}{\partial z}\hat{k}\right)\\
&\boxed{=\frac{\partial^2 f}{\partial x^2}+\frac{\partial^2 f}{\partial y^2}+\frac{\partial^2 f}{\partial z^2}}.
\end{aligned}
\] 梯度的散度常常简写为 $\nabla^2$ 或者 $\Delta$,
称作拉\textbf{普拉斯算子} (Laplacian). 拉普拉斯算子也可以作用在向量场上,
\[
\boxed{(\nabla\cdot\nabla)\cdot\boldsymbol{v}:=\nabla^2\boldsymbol{v}=(\nabla^2v_x)\hat{\imath}+(\nabla^2v_y)\hat{\jmath}+(\nabla^2v_z)\hat{k}}.
\]

\textbf{梯度的旋度} \[
\boxed{\nabla\times(\nabla f)\equiv0}.
\] 梯度的散度恒等于零. 有一个不正确的``理解'' (但是可以当作记忆法) 是,
如果把导数算符看作向量, 那么梯度的旋度便是导数算符的叉乘,
一个向量和自己的叉乘总是为零. 事实上, 若要得到这个结论,
需要用到导数算符\textbf{无挠} (torsion-free) 的性质,
这个性质并不是导数算符自带的,
而是欧氏三维空间搭配直角坐标系带来的一个好的性质, to be specific,
我们有类似 \[
\frac{\partial}{\partial x}\left(\frac{\partial f}{\partial y}\right)=\frac{\partial}{\partial y}\left(\frac{\partial f}{\partial x}\right),
\] 然后利用分量运算便很好验证.

\textbf{散度的梯度} 很少在理工科的语境中被使用到,
也没有太值得关注的规律, 甚至没有专有的称谓, 这里就不作展开.
唯一要注意的是它不等于拉普拉斯算子作用于向量场.

\textbf{旋度的散度} \[
\boxed{\nabla\cdot(\nabla\times\boldsymbol{v})=0}.
\] 旋度的散度恒为零. Again, 利用分量加以导数算符的无挠性易证. 记忆时,
可类比向量的
$\boldsymbol{a}\cdot(\boldsymbol{b}\times\boldsymbol{c})=(\boldsymbol{a}\times\boldsymbol{b})\cdot\boldsymbol{c}$
(可以自行验证), 于是又会有 $\nabla\times\nabla$ 项出现 which
可以视作一个向量和其自身的叉乘于是总是为零.

\textbf{旋度的旋度} \[
\boxed{\nabla\times(\nabla\times\boldsymbol{v})=\nabla(\nabla\cdot\boldsymbol{v})-\nabla^2\boldsymbol{v}}.
\] 记忆时, 可以类比向量三重积的 bac-cab rule, 即
$\boldsymbol{a}\times\boldsymbol{b}\times\boldsymbol{c}=\boldsymbol{b}(\boldsymbol{a}\cdot\boldsymbol{c})-\boldsymbol{c}(\boldsymbol{a}\cdot\boldsymbol{b})$
(可以自行验证). 当然, 严格一点应该依旧利用分量运算.
