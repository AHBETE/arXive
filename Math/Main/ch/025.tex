\section{点乘}\label{025}

前面我们看到向量有数乘这个运算, 那么向量之间有类似乘法得运算吗?

Well, 还是先从熟悉的三维欧氏空间 $\mathbb{R}^3$ 出发, 遂有单位向量
$\hat{\imath}$, $\hat{\jmath}$, 和 $\hat{k}$. 不难看出,
三维欧式空间里任意的一个向量, 都可以用这三个单位向量的倍数之和来表示,
于是在这样的上下文下 (under this context),
这三个单位向量可以被称为\textbf{基}或\textbf{基底} (basis),
它们\textbf{张成} (span) 了 $\mathbb{R}^3$ 中的全体向量的集合.

\begin{newquote}
当然这个基底的选择并不是唯一的. 一个``好的", 合理的选择应该满足:

首先是\textbf{基向量} (basis vector) 的数量,
基向量的数量应该和维度是一致的, 否则自由度会不够; 就 $\mathbb{R}^3$
为例, 如果只选取两个向量作为基向量, 它们所有的\textbf{线性组合} (linear
combination), 即它们两个的倍数之和, 被限制在了一个平面里,
而无法``覆盖''整个三维立体的空间.

还有是, 基向量之间应该是\textbf{线性不相关} (linearly
independent) 的,
即其任意一个基向量不应该能被剩余的其他基向量的线性组合表示; 还是考虑
$\mathbb{R}^3$ 中的情况, 两个基向量的线性组合已经``覆盖''了一个平面,
若第三个基向量的选取是前两个基向量的线性组合, 那么它依旧在那个平面内,
从而使得这三个向量的线性组合无法覆盖整个三维立体的空间.

举一个更具体地例子,
$\{\frac{\hat{\imath}+\hat{\jmath}}{\sqrt{2}},\frac{-\hat{\imath}+\hat{\jmath}}{\sqrt{2}},\hat{k}\}$,
也就是原先的三个基向量, 以 $z-$轴为旋转轴,
顺时针旋转45°得到的三个新的向量, 也可以是一组基底.

当然, 基向量的数量也可以多于维数, 基向量间也可以是线性相关的, 这样就有一定的冗余度, 一个向量在这样的基底展开下可以有多种形式, 没有唯一的展开形式在很多场合下就不够``好".
\end{newquote}

在 $\{\hat{\imath},\hat{\jmath},\hat{k}\}$ 这个基底选择下, 我们定义``点乘'' ((dot product)) 这个向量乘法, 点乘规律可以总结为: 对于向量
$\boldsymbol{a}=a_x\hat{\imath}+a_y\hat{\jmath}+a_z\hat{k}$ 和
$\boldsymbol{b}=b_x\hat{\imath}+b_y\hat{\jmath}+b_z\hat{k}$,
它们的点乘是 \[
\boxed{\boldsymbol{a}\cdot\boldsymbol{b}=a_xb_x+a_yb_y+a_zb_z}.
\] 这是因为我们规定, \[
\begin{cases}
\hat{\imath}\cdot\hat{\imath}=1\\
\hat{\jmath}\cdot\hat{\jmath}=1\\
\hat{k}\cdot\hat{k}=1
\end{cases},
\begin{cases}
\hat{\imath}\cdot\hat{\jmath}=\hat{\jmath}\cdot\hat{\imath}=0\\
\hat{\imath}\cdot\hat{k}=\hat{k}\cdot\hat{\imath}=0\\
\hat{\jmath}\cdot\hat{k}=\hat{k}\cdot\hat{\jmath}=0
\end{cases}.
\] 于是某种意义上, 点乘可以看作正常的乘法, 即: 
\begin{align*}
\boldsymbol{a}\cdot\boldsymbol{b}=&(a_x\hat{\imath}+a_y\hat{\jmath}+a_z\hat{k})(b_x\hat{\imath}+b_y\hat{\jmath}+b_z\hat{k})\\
=&a_x\hat{\imath}(b_x\hat{\imath}+{b_y\hat{\jmath}+b_z\hat{k}})\\
&+a_y\hat{\jmath}({b_x\hat{\imath}}+b_y\hat{\jmath}+{b_z\hat{k}})\\
&+a_z\hat{k}({b_x\hat{\imath}+b_y\hat{\jmath}}+b_z\hat{k}).
\end{align*}

写成分量形式后可以看出, 点乘这个运算是有交换律和结合律的.

\begin{newquote}
\textbf{正交归一基}/\textbf{正交规范基} (orthonormal basis)
$\{\hat{\imath},\hat{\jmath},\hat{k}\}$ 这个基底选择带来的一个好处是,
基向量之间的乘法非常容易总结: 不同的基向量相乘得到 $0$ - 正交,
同样的基向量相乘得到 $1$ - 归一 ; 更通常的, 如果有一组正交归一基,
$\{\hat{e}_1,\hat{e}_2,\hat{e}_3,…\}$, 基向量之间的乘法可以利用
Kronecker Delta表述为 \[
\hat{e}_i\hat{e}_j=\delta_{ij}=\begin{cases}1\ \text{if }i=j\\0\ \text{if }i\neq j\end{cases}.
\] 在 $\mathbb{R}^3$ 中, 正交可以比较肤浅的理解成两向量是垂直的,
而归一则可以理解为向量的``长度''是 $1$ (again
事实上我们还没有定义距离, but soon enough
我们就可以有``距离''这个概念了).

虽然正交归一是一个很好的性质, 但是基底的选择未必需要正交归一性; 当我们有一组比较任意的线性无关向量时, 若我们需要一组正交归一的基向量, 我们可以通过施密特正交化 (Schmidt orthogonalization) 来得到, 具体的过程这里就不做展开了.
\end{newquote}

\subsubsection{内积 (inner product) - 选读}

严格点来说, 点乘是一种内积, 或者也可以说, 内积是点乘的推广.

在【\ref{024}\nameref{024}】中, 我们看到, 向量这个概念事实上比我们想象得要更宽泛一些,
例如所有的光滑函数 $C^\infty$ 可以视作一个线性空间,
不难发现函数加法和函数数乘这两个运算是满足封闭性的,
同时余下的八条运算规律也是满足的.

至于基底的选择, 可以是多项式, 在【\ref{016}\nameref{016}】中,
【对一个函数做泰勒展开】就好比【向量写作分量形式】, \[
f(x)=\sum_{n=0}^\infty\frac{f^{(n)}(x_0)}{n!}(x-x_0)^n.
\] $(x-x_0)^n$ 和 $\frac{f^{(n)}(x_0)}{n!} $
相当于基向量和各基向量对应的分量, 值得注意的一点是, 在这个例子中,
基向量的数量是无限多的, 即所有连续函数构成的线性空间是无限维的.

除了多项式以为, 三角函数也是一个很好的选择, 忽略亿点点细节, 在一个区间
$[-\pi,\pi]$ 里一个函数的\textbf{傅立叶级数} (Fourier series) 可以写作
\[
f(x)=\frac{a_0}{2}+\sum_{n=1}^\infty\left(a_1\cos(nx)+b_n\sin(nx)\right),
\] 其中 \[
\begin{aligned}
a_n=\frac{1}{\pi}\int_{-\pi}^\pi f(x)\cos(nx)\mathrm{d}x,\\
b_n=\frac{1}{\pi}\int_{-\pi}^\pi f(x)\sin(nx)\mathrm{d}x.
\end{aligned}
\]

并且 $n$ 为正整数. 注意到 \[
\begin{aligned}
&\frac{1}{\pi}\int_{-\pi}^{\pi}\cos(nx)\cos(kx)\mathrm{d}x=\delta_{kn},\\
&\frac{1}{\pi}\int_{-\pi}^{\pi}\sin(nx)\sin(kx)\mathrm{d}x=\delta_{kn},\\
&\frac{1}{\pi}\int _{-\pi}^{\pi}\cos(nx)\sin(kx)\mathrm{d}x=0.
\end{aligned}
\] 也就是说, 如果我们把积分 $\int f(x)g(x)\mathrm{d}x$, 视作函数
$f(x)$ 和 $g(x)$ 的内积 (在这个例子里, 积分前还需要加一个归一化系数
$\frac{1}{\pi}$), 那么基底 $\{\cos(nx),\sin(nx)\}$ 便是正交归一的.

\begin{newquote}
上述的情况都是对于变量是实数的函数, 即实变函数 (functions of a real
variable), 将内积的概念推广到复变函数 (functions of a complex variable)
便成了 Hermitian 内积, $f(z)$ 和 $g(z)$ 的内积会是形如
$\int\bar{f}(z)g(z)\mathrm{d}z$, 上加一杠表示求复共轭 (参见【\ref{006}\nameref{006}】).

类似的, 对于分量含有复数的``通常''的向量, Hermitian 内积也应形如
$\bar{\boldsymbol{a}}\boldsymbol{b}$.
\end{newquote}

另外注意到, 在求每一个基向量前的系数, 也就是求各分量 $a_n$ 和 $b_n$
的方法, 和 $\mathbb{R}^3$
中求``通常``的向量的分量的方法是几乎``一致''的: \[
a_x=\boldsymbol{a}\cdot\hat{\imath}\Leftrightarrow a_n=\frac{1}{\pi}\int_{-\pi}^\pi f(x)\cos(nx)\mathrm{d}x.
\]

\begin{newquote}
抽象化一点, 借用物理的狄拉克记号 (Dirac notation) 或者说 bra-ket
notation: 如果有一系列的向量 $\left|\psi\right>$,
$\left|\phi\right>$, \ldots; $\left|\psi\right>$ 和
$\left|\phi\right>$ 内积记作 $\left<\psi\right.\left|\phi\right>$,
若有\textbf{完备} (complete) 正交归一基
$\{\left|\hat{e}_1\right>,\left|\hat{e}_2\right>,\left|\hat{e}_3\right>,...\}$,
那么向量 $\left|\psi\right>$ 可以展开成
$\left|\psi\right>=a_1\left|\hat{e}_1\right>+a_2\left|\hat{e}_2\right>+a_3\left|\hat{e}_3\right>+...$
; 在某个基向量 $\left|\hat{e}_i\right>$ ``方向''上的分量 $a_i$
一般都可以通过 $\left<\hat{e}_i\right.\left|\psi\right>$ 来计算.

(超纲) 证明: 这里需要用到完备正交归一基的特性
$\sum_i\left|\hat{e_i}\right>\left<\hat{e}_i\right|=\hat{I}$, 这里
$\hat{I}$ 是一个单位算符 (identity operator), 相当于一,
即``什么都不做'', 于是 \[
\left|\psi\right>=\sum_i\left|\hat{e_i}\right>\left<\hat{e}_i\right.\left|\psi\right>.
\] 利用运算的结合律, 后两项的运算相当于一个内积, 运算的结果是一个标量,
于是可以向前挪, 便有 \[
\left|\psi\right>=\sum_i\left<\hat{e}_i\right.\left|\psi\right>\left|\hat{e_i}\right>,
\] 对比 \[
\left|\psi\right>=\sum_ia_i\left|\hat{e_i}\right>,
\] 便有 $a_i=\left<\hat{e}_i\right.\left|\psi\right>$.
\end{newquote}