\section{导数的应用}\label{015}

先前我们有利用导数求函数的切线过, 这里我们继续来看导数还有那些应用.

\begin{tcolorbox}[size=fbox, breakable, enhanced jigsaw, title={一次导和极值}]

导数可以视作切线的斜率, 所以可以利用导数的正负判断函数的增减性,
当导数大于$0$ 时, 函数是递增的, 当导数小于$0$ 时, 函数是递减的.

若函数 $f(x)$ 有局域的最大或最小值 (统称极值) 在 $x=c$ 处, 且
$f'(x)$ 在 $x=c$ 处是被定义的, 那么 $f'(c)=0$.

\begin{newquote}
不太严谨的证明:

\begin{itemize}

\item
  考虑 $x=c$ 附近足够小的邻域 (neighborhood) , 使得在此领域内
  $f'(x)$ 是连续的;
\item
  因为 $f(c)$ 是极值: 则在 $x=c$ 的一侧, 有 $f'(x)>0$
  (切线斜率为正, 函数递增); 另一侧, 有 $f'(x)<0$ (切线斜率为负,
  函数递减);
\item
  利用零点定理 (参见【\ref{011}\nameref{011}】) , 这个领域中必然有一点使得 $f'(x)=0$,
  利用极限可得这个点恰好在 $x=c$ 上.
\end{itemize}
\end{newquote}

\begin{newquote}
一个 trivial 的例子:

利用上述性质, 我们来找一下二次函数的顶点.

若有函数 $f(x)=y=ax^2+bx+c$, 于是 $f'(x)=2ax+b$, 当 $f'(x)=0$,
$2ax+b=0$, 于是 $x=\frac{b}{2a}$. 将 $x=\frac{b}{2a}$ 代入原函数,
得到 $y=f\left(\frac{b}{2a}\right)=\frac{4ac-b^2}{4a}$.

这个结论和配方得到的结论是一致的
$y=a\left(x-\frac{b}{2a}\right)^2+\frac{4ac-b^2}{4a}$.
\end{newquote}

对于更复杂的函数, 也可以用同样的思路来求极值

\begin{itemize}

\item
  先对函数求导;
\item
  令导数为 $0$, 求得导数为 $0$ 处的自变量取值;
\item
  将求得的自变量取值代入原函数, 得到函数极值.
\end{itemize}

\end{tcolorbox}

\begin{tcolorbox}[size=fbox, breakable, enhanced jigsaw, title={二次导和凹凸性}]

我们可以对导数继续求导, 二次导可以这么记: $f''(x)\equiv \frac{\mathrm{d}^2f}{\mathrm{d}x^2}\equiv \frac{\mathrm{d}}{\mathrm{d}x}\frac{\mathrm{d}f}{\mathrm{d}x}$. 考虑一个有二次导的的函数\footnote{有的学科里会称为 $C^2$ 连续.}:

\begin{itemize}

\item
  若 $f''(x)>0$, 则函数是上凹的 (concave up);
\item
  若 $f''(x)<0$, 则函数是下凹的 (concave down);
\end{itemize}

\begin{tcolorbox}[size=fbox, breakable, enhanced jigsaw, sidebyside]
\includegraphics[width=0.9\textwidth]{img/image-20230614143547029.png}
\tcblower
\kaishu{\small }
\end{tcolorbox}

对于二次函数而言, 上凹下凹便对应着开口向上和开口向下.

函数的凹凸性发生变化的点我们称之为拐点 (point of inflection),
有时也称为驻点. 在拐点处, $f''(x)=0$ 或不存在 (例如趋向于无穷).

\end{tcolorbox}

\subsection{洛必达法则 (L'Hôpital's
rule)}

在【\ref{013}\nameref{013}】中, 在求三角函数的导数时, 我们遇到了极限是不定式/未定型的情况,
先前我们利用几何法绕开了这种情况, 现在我们来看怎么和这种情况刚正面.

\begin{tcolorbox}[size=fbox, breakable, enhanced jigsaw, title={罗尔中值定理 (Rolle's theorem)}]



\begin{tcolorbox}[size=fbox, breakable, enhanced jigsaw, sidebyside]
\includegraphics[width=0.9\textwidth]{img/image-20230615082313476.png}
\tcblower
\kaishu{\small 若 $f(x)$ 在区间 $[a,b]$ 连续, 且在区间 $(a,b)$ 可微或存在导数,
如果 $f(a)=f(b)$, 则至少存在一个点 $x=c$ 使得 $f'(c)=0$.\\很直观, 一定存在一个斜率正负值变化的地方, 处非是一个 trivial 的情况,
即函数为一个常数, 那么斜率处处为 $0$.}
\end{tcolorbox}

\begin{newquote}
证明思路:

极值定理 (参见【\ref{011}\nameref{011}】) 告诉我们区间 $[a,b]$ 存在极值, 前面我们还发现,
极值处的导数为 $0$.
\end{newquote}

\end{tcolorbox}

\begin{tcolorbox}[size=fbox, breakable, enhanced jigsaw, title={拉格朗日中值定理 (Lagrange mean value theorem)}]



\begin{tcolorbox}[size=fbox, breakable, enhanced jigsaw, sidebyside]
\includegraphics[width=0.9\textwidth]{img/image-20230615082602908.png}
\tcblower
\kaishu{\small 若 $f(x)$ 在区间 $[a,b]$ 连续, 且在区间 $(a,b)$ 可微或存在导数,
则至少存在一个点 $x=c$ 使得 $f'(c)=\frac{f(b)-f(a)}{b-a}$.}
\end{tcolorbox}

\begin{newquote}
证明思路:

这是罗尔中值定理''加强版'', 可以从罗尔中值定理出发去证明.
最后的结论可以理解为,
这个区间存在一个点使得【函数的斜率】和【函数在这个区间两个端点连线的斜率】一致.
因为斜率和连线的斜率一致, 也就是说, 【原函数】减去【两端点连线的函数】,
所得到的新的函数在这个点斜率是 $0$.

\begin{itemize}

\item
  综上, 构造这样一个函数 (最难的一步, 关于这一点的吐槽参见【\ref{011}\nameref{011}】,
  因为最后的), 令 $g(x):=f(x)-f(a)-\frac{f(b)-f(a)}{b-a}(x-a)$.
\item
  易见这样一来 $g(a)=g(b)=0$, 利用罗尔中值定理, 可得存在一点 $x=c$
  使得 $g'(c)=0$.
\item
  $g'(x)=f'(x)-\frac{f(b)-f(a)}{b-a}$, 这里要注意 $f(a)$ 和 $f(b)$
  已经将 $x=a$ 和 $x=b$ 分别代入了原函数, 是一个具体数值了;
\item
  于是 $g'(c)=f'(c)-\frac{f(b)-f(a)}{b-a}=0$, 证毕.
\end{itemize}
\end{newquote}

\end{tcolorbox}

\begin{tcolorbox}[size=fbox, breakable, enhanced jigsaw, title={柯西中值定理 (Cauchy's mean value theorem)}]



\begin{tcolorbox}[size=fbox, breakable, enhanced jigsaw, sidebyside]
\includegraphics[width=0.9\textwidth]{img/image-20230615082658188.png}
\tcblower
\kaishu{\small 若 $f(x)$ 和 $g(x)$ 在区间 $[a,b]$ 连续, 且在区间 $(a,b)$
可微或存在导数, 且在区间 $(a,b)$ 内 $g(x)\neq0$, 则至少存在一个点
$x=c$ 使得 $\frac{f'(c)}{g'(c)}=\frac{f(b)-g(a)}{g(b)-g(a)}$.}
\end{tcolorbox}

\begin{newquote}
证明思路:

构造一个函数 $h(x):=f(x)-f(a)-\frac{f(b)-f(a)}{g(b)-g(a)}(g(x)-g(a))$
再利用拉格朗日中值定理易得上述结论.
\end{newquote}

为了理解这一条, 可以引入参数化曲线 (parametric curve) 这个概念.
一条曲线, 除了用类似 $y=f(x)$ 这样的形式来表示,
若是这条曲线是一个物体的运动曲线, 还可以引入时间 $t$, 这个参数
(parameter), 然后用时间来表示 $\{x,y\}$ 的变化, 即 \[
\begin{cases}
x=x(t),\\
y=y(t);
\end{cases}
\] 于是柯西中值定理可以视作, $f(x)$ 和 $g(x)$ 分别代表两个坐标值,
$x$ 是一个参数, 这样一来 $f'(x)/g'(x)$
就可以理解为这条参数化曲线的切线的''斜率''了.

\end{tcolorbox}

回到我们的主题, 当出现类似 $f(a)=0$, $g(a)=0$, 且希望求得极限
$\lim_{x\rightarrow a}\frac{f(x)}{g(x)}$ 时, 我们可以从 $a$
的某一侧逼近 $a$, 利用柯西中值定理可以知道在此区间一定存在一点 $x=c$
使得 \[
\frac{f'(c)}{g'(c)}=\frac{f(x)-f(a)}{g(x)-g(a)},
\] 已知 $f(a)=0$, $g(a)=0$, 于是 \[
\frac{f'(c)}{g'(c)}=\frac{f(x)}{g(x)},
\] 再令 $x\rightarrow a$, 因为 $c$ 一定在 $x$ 与 $a$ 之间,
于是类似三明治定理的情况, 在这个极限下, $c$ 也会趋向于 $a$;
于是我们便得到了我们希望得到的结果: \[
\boxed{\lim_{x\rightarrow a}\frac{f(x)}{g(x)}=\lim_{x\rightarrow a}\frac{f'(x)}{g'(x)}}.
\] 也就是说, 求极限时, 出现 $\frac{0}{0}$ 的不定式,
我们可以非常简单粗暴地来看它导数的极限.

\begin{newquote}
例: $\lim_{x\rightarrow 0}\frac{\sin(x)}{x}$ \[
\lim_{x\rightarrow 0}\frac{\sin(x)}{x}\underset{\text{L.H.}}{=}\lim_{x\rightarrow 0}\frac{\cos(x)}{1}=1.
\]
\end{newquote}