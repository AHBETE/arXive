\section{泰勒级数}\label{016}

\begin{tcolorbox}[size=fbox, breakable, enhanced jigsaw, title={求和 (summation) - 复习}]

求和的标记 $\sum$ 在\ref{007}\nameref{007}其实已经介绍过了, 这里再复习一遍.
求和符号右边是代求和的每一项的表达式,
求和符号下面标注了右边表达式的第一项需要代入的值,
求和符号上面则标注了最后一项需要代入的值.

\begin{newquote}
举一个例子, 小学的高斯求和法,
\[
1+2+...+99+100=\sum_{n=1}^{100}n=5050.    
\]
高斯的思路无非是, 将这个求和拆为 $\{1,100\}$, $\{2,99\}$, \ldots,
$\{49,52\}$$, \{50,51\}$ 的组合, 每一组的和都是 $101$, 一共有
$50$ 个这样的组合, 于是得到``一加到一百''为 $5050$ \footnote{小学高我一年级的一学长刚学完这一课向我耍宝的时候,
  其实我也想到了类似的方法, 把求和凑成 $\{1,99\}$, $\{2,98\}$,
  \ldots, $\{48,52\}$$, \{49,51\}$ 的一对对, 最后还剩下 $50$ 和
  $100$, 也能得到答案, 不过这个方法并不普适.}.
推广一下便可得到对一组公差为 $1$ 的等差数列求和,
和为【首项】加【末项】乘【项数】除以二. 于是有 \[
\sum_{n=1}^Nn=\frac{N(N+1)}{2}.
\] 另有两个可能会有一些用的结论: \[
\begin{aligned}
  \sum_{n=1}^Nn^2&=\frac{N(N+1)(2N+1)}{6},\\
  \sum_{n=1}^Nn^3&=\left(\frac{N(N+1)}{2}\right)^2.
\end{aligned}
\] 证明留作练习, 提示是可以利用数学归纳法 (参见\ref{007}\nameref{007}).
\end{newquote}

因为前面提到过, 这并不是一本严谨的数学书, 数列和级数我们就跳过了;
但是接下来要讲泰勒级数, 还是来一点开胃菜 (appetizer): \textbf{几何级数}
(geometric series).

几何级数又叫等比级数, 即它的每一项和之前一项的倍数是恒定的. 于是第一项为
$a$, 相邻两项倍数为 $r$ 的几何级数可以记作 \[
\sum_nar^{n-1}.
\] 不难证明前 $n$ 项之和应为 \[
S_n=a\frac{r^n-1}{r-1}.
\]

\begin{newquote}
上结论证明如下:

前 $n$ 项之和展开写是 $S_n=a+ar+ar^2+...+ar^{n-2}+ar^{n-1}$,
两边同乘 $r$ 得到 $rS_n=ar+ar^2+ar^3+...+ar^{n-1}+ar^n$. 将 $rS_n$
与 $S_n$ 相减, 消去相同项便有 $(r-1)S_n=a(r^n-1)$,
整理便可得到结论.\footnote{在这个推导中, 可以看到我们将代求的 $S_n$
  带着计算, 要习惯这种和逆向思维相对的``正向思维'',
  \ref{014}\nameref{014}逆函数的导数的推导也有这么一丝味道,
  【剧透警告】之后积分中非常重要的分步积分法也会有类似的思路.}
\end{newquote}

这样一来, 当这个级数无限长, 即 $n$ 趋向于无穷, 且公倍数 $r$
的绝对值小于一, 这个级数 (求和) 收敛到 \[
\sum_nar^{n-1}=\frac{a}{1-r}.
\]

\end{tcolorbox}

\begin{tcolorbox}[size=fbox, breakable, enhanced jigsaw, title={泰勒级数 (Taylor Series)}]

有的时候, 我们研究的函数 $f(x)$ 可能并不是一个非常 nice 的函数,
它的很多特性并不那么``好''; 更直观一些,
很多时候我们研究的函数可能是\textbf{超越函数} (Transcendental
Functions), 即变量之间的关系不能用有限次加, 减, 乘, 除,
次方运算表示的函数, 最简单的例子, 比如三角函数, 不利用计算器等工具,
我们很难去直接计算. 正好, 很多现实场景下, 我们也并不需要任意精确的结果,
我们可能只需要几位有效数字, 这个时候,
我们便可以利用那些``好''的函数去\textbf{拟合} (fit)
这些不那么``好''的函数.

最常见的比较``好''的函数是什么? 多项式 (Polynomial)! 它取值相对简单,
对它求导更是非常轻松. 所以很多时候, 我们会用多项式去拟合.

现在假设我们手头上有一个函数 $f(x)$, 我们希望能用多项式去拟合它,
那么它大约可以被写成以下的形式:

\[
f(x)=\sum_{n=0}^\infty a_n(x-x_0)^n,
\]

这里 $a_n$ 是每一项的系数, $x_0$ 是某个固定的值, 可以看到 $x$
的最高系数逐项增加. 对上式不断求导可以得到

\begin{gather*}
f'(x)=a_1+2!\cdot a_2(x-x_0)+\mathcal{O}(x-x_0)^2\\
f''(x)=2!\cdot a_2+3!\cdot a_3(x-x_0)+\mathcal{O}(x-x_0)^2\\
...
\end{gather*}

仅在这里, 我们规定 $\mathcal{O}$ 的含义是, 例如: $\mathcal{O}(x)$
表示包含 $x$ 以及 $x$ 更高次数的项 (即, $ax, bx^2, cx^3, ...$ ).
通常而言, $n$ 次导可以得到:

\[
f^{(n)}(x)=n!a_n+\mathcal{O}(x-x_0)
\]

令 $x:=x_0$ 便可得到每一项系数应该是

\[
a_n=\frac{f^{(n)}(x_0)}{n!},
\]

于是用来拟合一个函数的多项式便可以是

\[
\boxed{f(x)=\sum_{n=0}^\infty\frac{f^{(n)}(x_0)}{n!}(x-x_0)^n.}
\]

将一个函数用上式展开, 便得到了这个函数的\textbf{泰勒级数}\footnote{剧透:
  把这个结论推广到复变函数, 泰勒级数就变成了\textbf{洛朗级数 (Laurent
  series)}.}. \ref{007}\nameref{007}中提到的二项式展开,
其实可以看做是泰勒级数的一个特例.

\begin{tcolorbox}[size=fbox, breakable, enhanced jigsaw, title={收敛性}]

前面我们似乎是假定随着项数增加泰勒级数可以趋向原函数了, 但是一定如此吗?
这边有一个定理.

\begin{tcolorbox}[size=fbox, breakable, enhanced jigsaw, title={泰勒中值定理 (Taylor's Theorem)}]

有一个 $n$ 次可导的函数 $f(x)$, 将其的导数依次记作
$f'(x),f''(x),...,f^{(n)}(x)$, 考虑一个闭区间 $[a,b]$,
那么在这个区间内一定存在一个 $c$ 使得 \[
f(b)=f(a)+f'(a)(b-a)+\frac{f''(a)}{2!}(b-a)^2+...+\frac{f^{(n)}(c)}{n!}(b-a)^n.
\] 注意等式右边前面的函数和导数都是在 $x=a$ 处取值, 而最后一项是在
$x=c$ 处. 不难看出这其实是中值定理的推广.

如此一来, 换言之, $f(x)$ 与
$f(x_0)+f'(x_0)(x-x_0)+...+\frac{f^{(n-1)}(x_0)}{(n-1)!}(x-x_0)^{n-1}$
的偏差等于 $\frac{f^{(n)}(\xi)}{n!}(x-x_0)^n$, 对于某个在 $x$ 和
$x_0$ 之间的 $\xi$.

\end{tcolorbox}

实际使用的时候, 我们常常这么说: 【在 (某个特定的) $x_0$ 附近展开
$f(x)$ \ldots】 这样一来, 就要注意, 展开后,
取\textbf{有限项}级数来\textbf{近似}原函数时, 只有取 $x$
\textbf{足够}接近 $x_0$ 时才是有效的. 例如: 利用在 $0$ 附近展开的
$\sin(x)$ 函数的泰勒级数的前若干项, 来近似当 $x$ 很大时 (例如大于
$2\pi$) 时, 结果会是非常荒谬的\footnote{一些题外话: 在计算科学,
  数据分析等专业, 当我们有许多数据点,
  经常会用多项式去试图拟合这一些数据点, 更高次的多项式,
  通常意味着对已有数据更好的拟合 (比如 $n$ 次的多项式可以完美拟合
  $(n+1)$ 个数据点), 但是这往往并不是我们想要的,
  一方面在所有数据点两端, 高次多项式的行为会非常不合理, 另一方面,
  事实上一个模型也不应该具有如此高的自由度 (多项式每高一次,
  对应着模型多一个自由度). 这就引出了两个概念: 插值 (interpolation)
  vs.~拟合 (fitting), 插值要求函数通过每一个给定的数据点,
  而拟合是在现有模型的基础上调整参数, 保证函数和数据点是最小二乘的
  (least squared).}; 这里可以利用 $\sin(x)$ 函数的周期性, 将 $x$
减去若干个 $2\pi$ 再计算.

\end{tcolorbox}

\end{tcolorbox}

\begin{tcolorbox}[size=fbox, breakable, enhanced jigsaw, title={练习}]

求以下函数的泰勒级数: (a) $(1\pm x)^{-1}$, (b) $\mathrm{e}^x$, (c)
$\sin(x)$, (d) $\cos(x)$, (e) $\ln(1+x)$, (f) $\arctan(x)$.

展开后不难发现, 对于 $x\approx 0$ (或者说 $x\ll1$) 有
$\sin(x)\approx x$, $\cos(x)\approx 1$ (当 $x$ 足够小, 含 $x$
及其高次的项便足够小, 回收了\ref{009}\nameref{009}中提过的小角度近似).
另外还有一个常用的近似是 $\ln(x+1)\approx x$ 对于 $x\ll 1$. 在还有,
展开的形式可以进一步地佐证欧拉公式 (回顾\ref{009}\nameref{009}).

\end{tcolorbox}