\section{为什么写这个系列? Why?}
{\small
鄙人说起来本科也有一个数学专业, 但是水分很多, 最多只能说是应用数学, 比起纯数的大佬们低到不知道哪里去了, 但是比起大多数非理工科人还是稍微多知道一些, 因此很多时候很想分享一些``有趣"的点, 但又惧于遭来更专业的批评.

因此, 在这里, 我可能要用非专业语言, 呈现一些数学的图景, 对正统的学习没太大帮助的那种.

这个系列呈现的内容必然会有很多数学不严谨的地方, 甚至可能会犯一些错误, 届时欢迎大家指正和提出修改意见; 但是这个系列的主旨是呈现一些, 工科或自然科学甚至社科里实用的数学, 并在此基础上稍作衍生, 若是形式上有一些无伤大雅但不影响内容传达的问题, 往各位看官海涵.

既然提到了内容会不严谨且有出错的可能, 这个系列非常不推荐作为严谨的学习参考, 推荐仅当作趣味性的阅读, 类似工科人, 自然科学人, 社科人的一点数学营养补充剂 (本品为保健品, 不可替代药物和治疗.png).}

\section{如何阅读数学书? How to read math?}
{\small
其实不局限于数学教科书, 但是对于数学教科书更适用. 有的知识点, 初读觉得很抽象, 难以接受的话呢, 就先``take it'': 不求甚解地拿着它, 未必要彻头彻尾地理解, 先读下去; 经过几个例子, 或许理解便到位了; 再或者, 了解了一个更后面的知识点, 回头再来看, 便会发出大彻大悟的``哦-''.

就像『星际穿越』- \textit{Interstellar} 中所说的:
\begin{newquote}
    ``Don't try to understand it. Feel it.''
\end{newquote}
有时要用心去感受那种感觉, 而不要很刻意地去尝试理解. 去感受那种趋势, 感受那种自然的流动, 那种规则, 顺势而为.}