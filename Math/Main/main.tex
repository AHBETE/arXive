\documentclass[openany]{book}
\usepackage[UTF8]{ctex}
\title{从零开始的数学}
\newcommand{\booksubtitle}{从零开始系列}
\newcommand{\booklicense}{Creative Commons}

\author{Zibo (AHBETE) Wang}
% Author subtitle could be a university or a geographical location, for example
\newcommand{\authorsubtitle}{}

% Create convenient commands \booktitle and \bookauthor
\makeatletter
\newcommand{\booktitle}{\@title}
\newcommand{\bookauthor}{\@author}
\makeatother

% This utf8 declaration is not needed for versions of latex > 2018 but may
% be helpful for older software. Eventually it may not be worth keeping.
\usepackage[utf8]{inputenc}  
\usepackage{fix-cm}  % this package allows large \fontsize
\usepackage{tikz}    % this is for graphics. e.g. rectangle on title page
\usepackage{amsmath,amssymb} % Used by equations
\usepackage{bookmark,tcolorbox}%physics
\usepackage{hyperref}
\tcbuselibrary{breakable}
\tcbuselibrary{skins}

\newtcolorbox[blend into=figures]{myfigure}[2][]{float=htb,capture=hbox,
title={#2},every float=\centering,#1}

\newenvironment{newquote}{\begin{quote}\kaishu\small}{\end{quote}}
\newcommand{\envkaibf}[1]{{\small \bfseries \Kaishu #1} }

\setlength\parindent{0pt}

% The following dimensions specify 4.75" X 7.5" content on 6 3/8" by 9 1/4"
% paper. The paper width and height can be tweaked as required and the content
% should size to fit within the margins accordingly.
%
% The (inside) bindingoffset should be larger for books with more pages. Some
% standard recommended sizes are .375in minimum up to 1in for 600+ page books.
% Sizes .75in and .875in are also recommended roughly at 150 and 400 pages.
\usepackage[bindingoffset=0.625in,
            left=.5in, right=.5in,
            top=.8125in, bottom=.9375in,
            paperwidth=6.375in, paperheight=9.25in]{geometry}
% Here is an alternative geometry for reading on letter size paper:
% \usepackage[margin=.75in, paperwidth=8.5in, paperheight=11in]{geometry}

\renewcommand{\contentsname}{Table of Contents} % default is {Contents}
\usepackage{makeidx}
\makeindex % Initialize an index so we can add entries with \index

% Content Starts Here
\begin{document}
\frontmatter

% ---- Title Page ----
% current geometry will be restored after title page
\newgeometry{top=1.75in,bottom=.5in}
\begin{titlepage}
    \begin{flushleft}

        % Title
        \begin{center}
            \textbf{{\fontsize{42}{42}\selectfont  \}ESSENTIAL\{}\\{\LARGE\selectfont  从零开始的数学}}
        \end{center}%\fontfamily{qcs}

        % Draw a line 4pt high
        \par\noindent\rule{\textwidth}{4pt}\\

        % Shaded box from left to right with Subtitle
        % The text node is midway (centered).
        \begin{tikzpicture}
            \shade[bottom color=lightgray,top color=white]
            (0,0) rectangle (\textwidth, 1.5)
            node[midway] {\textbf{\large {\booksubtitle}}};
        \end{tikzpicture}

        % Edition Number
        \begin{flushright}
            \large Zeroth Edition
        \end{flushright}

        \vspace{\fill}

        % Author and Location
        \textbf{\large Author: }\textbf{\large \bookauthor}\\[3.5pt]
        \textbf{\large \textit{\authorsubtitle}}\\
        % \textbf{\large Editor: Jiyuan (Maki) Lu}\\
        % \textbf{\large }

        \vspace{\fill}

    \end{flushleft}
\end{titlepage}
\restoregeometry
% ---- End of Title Page ----

% Do not show page numbers on colophon page
\thispagestyle{empty}

\begin{flushleft}
    \vspace*{\fill}
    % AHBETE's physics notes series\\
    % A member of Maki's Lab (https://www.maki-math.com)\\
    % Contact info: wangz57@rpi.edu\\
    \vspace{\fill}
    Copyright \textcopyright{} \the\year{}  \bookauthor\\
    License: \booklicense
\end{flushleft}

% A title page resets the page # to 1, but the second title page
% was actually page 3. So add two to page counter.
\addtocounter{page}{2}

% Three-level Table of Contents
\setcounter{tocdepth}{1}
\tableofcontents

% The asterisk excludes chapter from the table of contents.
\chapter*{Preface}
\section{为什么写这个系列? Why?}
{\small
鄙人说起来本科也有一个数学专业, 但是水分很多, 最多只能说是应用数学, 比起纯数的大佬们低到不知道哪里去了, 但是比起大多数非理工科人还是稍微多知道一些, 因此很多时候很想分享一些``有趣"的点, 但又惧于遭来更专业的批评.

因此, 在这里, 我可能要用非专业语言, 呈现一些数学的图景, 对正统的学习没太大帮助的那种.

这个系列呈现的内容必然会有很多数学不严谨的地方, 甚至可能会犯一些错误, 届时欢迎大家指正和提出修改意见; 但是这个系列的主旨是呈现一些, 工科或自然科学甚至社科里实用的数学, 并在此基础上稍作衍生, 若是形式上有一些无伤大雅但不影响内容传达的问题, 往各位看官海涵.

既然提到了内容会不严谨且有出错的可能, 这个系列非常不推荐作为严谨的学习参考, 推荐仅当作趣味性的阅读, 类似工科人, 自然科学人, 社科人的一点数学营养补充剂 (本品为保健品, 不可替代药物和治疗.png).}

\section{如何阅读数学书? How to read math?}
{\small
其实不局限于数学教科书, 但是对于数学教科书更适用. 有的知识点, 初读觉得很抽象, 难以接受的话呢, 就先``take it'': 不求甚解地拿着它, 未必要彻头彻尾地理解, 先读下去; 经过几个例子, 或许理解便到位了; 再或者, 了解了一个更后面的知识点, 回头再来看, 便会发出大彻大悟的``哦-''.

就像『星际穿越』- \textit{Interstellar} 中所说的:
\begin{newquote}
    ``Don't try to understand it. Feel it.''
\end{newquote}
有时要用心去感受那种感觉, 而不要很刻意地去尝试理解. 去感受那种趋势, 感受那种自然的流动, 那种规则, 顺势而为.}

% % Three-level Table of Contents
% \setcounter{tocdepth}{1}
% \tableofcontents

\mainmatter

\part{}

\chapter{预备微积分 (precalculus)}
\hypertarget{ux52a0ux6cd5-addition}{%
\subsubsection{加法 (addition)}\label{ux52a0ux6cd5-addition}}

\(1+1=2\) , 好了, 这一节结束 (玩笑). 加法的运算, 九九加法表,
这里就不赘述.

我们首先考虑\textbf{整数} (integer), 记作 \(\mathbb{Z}\),
当我们考虑一类数字或者一类符合某种性质的对象时,
我们称这一类对象组成的东西叫做\textbf{集合} (set),
集合中的东西便叫做\textbf{元素} (element).

\begin{itemize}
\tightlist
\item
  不难发现, 从整数里取出两个元素, 例如 \(1\) 和 \(2\) , 将他们相加,
  得到的结果 \(1+2=3\) 依旧是整数.
  这样的性质叫做\textbf{封闭性}或者\textbf{闭包性} (closed, closure).
\item
  再考虑三个元素在加法下的运算. 三个元素的运算可以看作,
  两个元素的运算结果与第三个元素再次运算, 还是不难发现,
  任意从整数取出三个元素, 例如 \(1\) , \(2\) 和 \(3\) , 有
  \((1+2)+3=6=1+(2+3)\) ,
  即前两个元素先进行运算和后两个元素先进行运算的结论时一致的.
  这样的性质叫做\textbf{结合律} (assosiative).
\item
  整数里存在一个特殊的元素,
  使得加法这个运算不对其他任何元素''产生效果'', 这个特殊的元素是 \(0\) ,
  \(0+n=n+0=n\) , 这里 \(n\) 是任意整数, 我们可以这么标记:
  \(n\in\mathbb{Z}\) , 中间这个符号表示属于 (belong to).
  这个特殊的元素叫做\textbf{单位元}或\textbf{幺元} (identity
  element)\footnote{单位和幺都有一的含义, 因为在乘法中单位元是1,
    这可能是名字来源.}.
\item
  整数的加法中, 对于任何一个元素, 都能找到另一个元素,
  使得它们运算结果为单位元- \(0\) , 比如 \(1+(−1)=(−1)+1=0\) , 我们便叫
  \((−1)\) 是 \(1\) 的\textbf{逆元} (inverse element).
\end{itemize}

一个集合, 再附加一个二元运算(像加法这样输入两个元素输出一个元素的运算),
并且拥有上述性质和元素的, 我们便把它叫做\textbf{群} (group), 整数和加法,
便是这样构成了一个群 \((\mathbb{Z},+)\) .

好的,我们在学习加法的过程中顺便体验了以下群论. 要注意的是,
上文并没有强调\textbf{交换性} (commutative), 因为往后看我们会发现,
很多运算其实并不满足交换律,
满足交换律的群我们可以称它为\textbf{交换群}或\textbf{阿贝尔群} (Abelian
group).

\begin{quote}
有人问一个小朋友, ``3+4 等于几啊?'' 小朋友说: ``不知道, 但我知道 3+4
等于 4+3.'' 那人接着问: ``为什么呀?''
答曰:``因为整数与整数加法构成了阿贝尔群.''
这个笑话讽刺了某次法国一场幼儿园从抽象数学教起的实验,
不过最后实验的结果是以失败告终.
\end{quote}

\hypertarget{ux51cfux6cd5-subtraction}{%
\subsubsection{减法 (subtraction)}\label{ux51cfux6cd5-subtraction}}

思路要打开, 减法可以看作是加法的逆运算; 又或者, 减去一个数,
可看作加上这个数字的逆元.

减法的一些性质:

\begin{itemize}
\tightlist
\item
  \textbf{反交换律} (anti-commutativity), 例如 \(4−3=−(3−4)\) ,
  交换两个元素的顺序会导致结果变为之前结果的逆.
\item
  \textbf{非结合律} (non-associativity), 例如 \((6−3)−2\neq 6−(3−2)\) .
\end{itemize}

因为整数减法不满足结合律, 所以整数和减法不构成群, 只构成\textbf{拟群}
(quasi-group), 字面上可以理解成, 像群, 但是不是群, 这里不做展开.

\hypertarget{ux4e58ux6cd5-multiplication}{%
\subsubsection{乘法
(multiplication)}\label{ux4e58ux6cd5-multiplication}}

还是九九乘法表, 结束了 (玩笑). 乘法可以视作是多个同样加法的标记, 例如:
\(2×3=2+2+2=3+3=3×2\) .

来看看乘法的一些性质:

\begin{itemize}
\tightlist
\item
  易见\textbf{封闭性}或者\textbf{闭包性}是满足的.
\item
  也不难看出乘法具有\textbf{结合律}.
\item
  乘法的\textbf{幺元}是 1 .
\item
  再\textbf{逆元}上似乎出了问题, 到目前为止, 我们讨论的都还是整数
  \(\mathbb{Z}\) , 这个范围内, 似乎找不到逆元, 但是没有关系,
  我们把范围扩大到非零\textbf{有理数} (rational number)\footnote{有理数其实是谬译,
    rational在这里其实意为可约的, 而不是有理的.}, 记作
  \(\{\mathbb{Q}/\{0\}\}\) , 这样每一个元素 \(n\in\{\mathbb{Q}/\{0\}\}\)
  都有逆元 \(\frac{1}{n}\) . 将 \(0\) 剔除是因为它没有逆元, 我们应该知道
  0 不能作为分母.
\end{itemize}

以上性质已经决定了非零有理数和乘法构成群, \((\mathbb{Q}/\{0\}\,×)\) .
乘法另外还有特性:

\begin{itemize}
\tightlist
\item
  乘法与加法的混合运算, 会有\textbf{分配律} distributive property, 例如
  \(2×(3+4)=2×3+2×4\) .
\item
  任何数乘上 \(0\) 得到 \(0\) , \(0\) 可以称作乘法的\textbf{零元} (zero
  element), 零元没有逆.
\end{itemize}

\hypertarget{ux9664ux6cd5-division}{%
\subsubsection{除法 (division)}\label{ux9664ux6cd5-division}}

乘法和除法的关系类似加法和减法的关系. 除以零在大多数场景下是不被定义的.

\begin{quote}
轻清者上浮而为天 重浊者下凝而为地
\end{quote}

\hypertarget{ux5e42ux8fd0ux7b97-exponentiation}{%
\subsubsection{幂运算
(exponentiation)}\label{ux5e42ux8fd0ux7b97-exponentiation}}

幂运算可以视作重复的乘法, 即
\(a^n=\underbrace{a\times ...\times a}_{n}\), 这里 \(a\) 称为底数 (base)
, \(n\) 称为指数 (exponent)\footnote{从这一篇开始,
  文章的叙述讲逐渐从''具体→抽象''过渡到''抽象→具体'',
  即由之前先给一个具体数字运算的例子推广到用字母表示的通常情况,
  变为反过来的顺序; 阅读过程中如果觉得不适应, 抽象的点读不懂时,
  可以先接着往下看, 若之后有一个具体的例子, 可能对理解会有帮助.},
\(a^n\) 读作 \(a\) 的 \(n\) 次幂,或 \(a\) 的 \(n\) 次方.

先考虑正整数次幂, 一些运算规律:

\begin{itemize}
\tightlist
\item
  \(\begin{aligned}a^m\times a^n=\underbrace{a\times ...\times a}_{m}\times\underbrace{a\times ...\times a}_{n}&&\\  =\underbrace{a\times ...\times a}_{n+m}&&=a^{n+m}\end{aligned}\)
\item
  \(\begin{aligned}a^m\div a^n=\underbrace{a\times ...\times a}_{m}\div\underbrace{(a\times ...\times a)}_{n}&&\\  =\underbrace{a\times ...\times a}_{n-m}&&=a^{n-m}\end{aligned}\)
\end{itemize}

再来考虑 \(0\) 次幂, 因为上述运算规律 \(a^n\times a^0=a^{n+0}=a^n\),
因此应该有 \(a^0=1\) ; 要注意, 当底数为 \(0\) 时, \(0^0\)
是不被定义的\footnote{一说理由和 \(0\) 不能作为除数类似;
  另一说要从函数的角度出发, 这里稍稍剧透, 即,
  构建不同的函数极限试图求这个''值''会有不同的结果,
  所以这个''值''没有一个很好的公认的定义.}.

\begin{itemize}
\tightlist
\item
  \(a^0=1\)对于非零的 \(a\).
\end{itemize}

现在来看负整数为指数的幂, 参考第二条规律, 不难看出 \(a^{-n}\)
可以理解为除掉了 \(n\) 个 \(a\), 因此有

\begin{itemize}
\tightlist
\item
  \(a^{-n}=\frac{1}{a^n}\).
\end{itemize}

因为在前面一节我们已经把我们研究的范围扩充到了所有有理数,
所以不妨来看看分数作为指数的情况. 考虑 \(a^{\frac{1}{2}}\), 这里 \(a\)
是有理数, 令 \(b:=a^{\frac{1}{2}}\), 平方可得
\(b^2=a^{\frac{1}{2}}\times a^{\frac{1}{2}}=a^{\frac{1}{2}+\frac{1}{2}}=a\);
事实上我们知道, 要求 \(b\) 的话, 只需进行''开方''这个操作, 记作
\(b=\sqrt{a}\), 因此有 \(a^{\frac{1}{2}}=b=\sqrt{a}\); 然而,
这个操作其实是有一点''小问题''的.

\begin{quote}
这个''小问题''便是, 目前为止, 我们讨论的范围还限于有理数,
然而上述操作得到的 \(\sqrt{a}\) 并不一定是有理数;
这个问题在历史上也困扰了人们很久.

起初人们认为数轴上所有的数都应该可以用整数之比 (也就是有理数) 来表示,
但有人发现, 例如边长为1的正方形, 其对角线的平方利用\textbf{勾股定律}
(Pythagorean theorem - 毕达哥拉斯定律) 应该是 \(2\),
找不出一个有理数使得其平方正好为 \(2\).

然后为了解决问题, 提出问题的人就被解决掉了, 悲伤的故事.

现在, 平方正好为 \(2\) 的数字被记作了 \(\sqrt{2}\),
它不是有理数的证明可以留作证明, 一点提示就是可以利用反证法,
首先假设它是一个有理数, 并可以表示为例如 \(\frac{p}{q}\), 且 \(p\) 和
\(q\) 都是正整数, 然后证明这样的 \(\frac{p}{q}\) 不可能存在.
\end{quote}

解决这个''小问题''的方法, 是要再次扩展我们研究的范围; 这次我们将有理数,
即整数和分数, 以及数轴上''剩余''的那些不能表示成分数形式的无理数,
统称为\textbf{实数} (real number), 记作 \(\mathbb{R}\).
这样一来我们便不必担忧开方的结果''掉到''范围外了, 上面的结论也不难推广为

\begin{itemize}
\tightlist
\item
  \(a^{\frac{n}{m}}=\sqrt[m]{a^n}=(\sqrt[m]{b})^n\).
\end{itemize}

\hypertarget{ux5bf9ux6570-logarithm}{%
\subsubsection{对数 (logarithm)}\label{ux5bf9ux6570-logarithm}}

对数是幂运算的逆运算. 若 \(y=a^x\), 定义对数运算为 \(x=\log_a(y)\),
\(a\) 叫做底数 (base) , \(y\) 叫做真数.

对数有以下运算规律:

\begin{itemize}
\tightlist
\item
  \$ \log\_a(XY)= \log\_a(X)+ \log\_a(Y)\$. 证明如下: 令
  \(x=\log_a(X)\), \(y=\log_a(Y)\), 根据对数定义则有 \(a^x=X\),
  \(a^y=Y\);
  \(\log_a(XY)=\log_a(a^x\times a^y)=\log_a(a^{x+y})=x+y=\log_a(X)+ \log_a(Y)\).
\item
  \$ \log\_a\left(\frac{X}{Y}\right)= \log\_a(X)- \log\_a(Y)\$.
  证明和上一条类似.
\item
  \(\log_a(x^n)=n\log_a(x)\). 由第一条规律可得
  \(\log_a(x^n)=\underbrace{\log_a(x)\times...\times\log_a(x)}_{n}=n\log_a(x)\).
\item
  \(\log_a(x)=\frac{\log_b(x)}{\log_b(a)}\). 令 \(\log_a(x)=t\), 则有
  \(x=a^t\), 对两边同时取以 \(b\) 为底数的对数,
  \(\log_b(x)=\log_b(a^t)=t\log_b(a)=\log_a(x)\log_b(a)\),
  整理便可得上述规律.
\end{itemize}

以上四条为最基本最常用得运算规律,
还有一些运算规律可以从上面几条推到而来, 证明留作练习.

\begin{itemize}
\tightlist
\item
  \(\log_{a^n}x=\frac{1}{n}\log_{a}x\).
\item
  \(a^{\log_a(x)}=\log_a(a^x)=x\).
\item
  \(x^{\log_a(y)}=y^{\log_a(x)}\).
\item
  \(\log_a(x)=\frac{1}{\log_x(a)}\).
\item
  \(\log_a(b)\log_b(x)=\log_a(x)\).
\end{itemize}

\input{ch/003}
\input{ch/004}
\hypertarget{ux4e09ux89d2ux51fdux6570-trigonometry}{%
\subsubsection{三角函数
(Trigonometry)}\label{ux4e09ux89d2ux51fdux6570-trigonometry}}

三角函数最基本的使用应该是表示直角三角形的变长比. 如下图所示, 三角形
\[ABC\] 为直角三角形, 将 \[\angle BAC\] 记作 \[\theta\], 对于两条直角边
\[AB\] 和 \[BC\], 边 \[AB\] 在 \[\theta\] 边上, 称它为\textbf{邻边}
(adjacent), 边 \[BC\] 在 \[\theta\] 对面, 称它为\textbf{对边}
(opposite), 剩余的边 \[AC\] 被称为\textbf{斜边} (hypotenuse)。

易见, 各变长比仅和 \[\theta\] 相关\footnote{当然也可以说和除了直角外的另一个角
  \[(90^\circ-\theta)\] 相关; 边长比可以通过一个除直角外的角确定是因为,
  除直角外另一角相等的直角三角形都相似, 它们的边长比是一致的。},
三角函数便是用来表示各个比例的, 常用的三角函数有

\[\begin{align*}\cos\theta&=\frac{邻边}{斜边}=\frac{AB}{AC},\\
\sin\theta&=\frac{对边}{斜边}=\frac{BC}{AC},\\
\tan\theta&=\frac{对边}{邻边}=\frac{BC}{AB}.\end{align*}\]

不难看出\[\tan\theta=\frac{\sin\theta}{\cos\theta}\].

另外还有

\[\begin{align*}\sec\theta&\equiv\frac{1}{\sin\theta},\\
\csc\theta&\equiv\frac{1}{\cos\theta},\\
\cot\theta&\equiv\frac{1}{\tan\theta}.\end{align*}\]

\[\csc\] 很多时候也记作 \[\text{cosec}\].

一个非常实用的关系, 直角三角形中有\textbf{勾股定理} (Pythagorean
theorem): 斜边边长平方等于两直角边边长的平方之和, 即 \[AC^2=AB^2+BC^2\];
两边同时除以 \[AC^2\] 便有

\begin{itemize}
\tightlist
\item
  \[\boxed{1=\cos^2\theta+\sin^2\theta}\].\footnote{三角函数的平方:
    cos(x)\textsuperscript{2} 通常理解为 cos((x)\textsuperscript{2});
    cos\textsuperscript{2}x 约定俗成表示 (cos(x))\textsuperscript{2}.}
\end{itemize}

\textbf{正弦定律 Law of sine}

将三角形三个角分别记作 \[\alpha\], \[\beta\], 和 \[\gamma\],
将它们的对边分别记作 \[A\], \[B\], 和 \[C\]. 先是结论:

\begin{itemize}
\tightlist
\item
  \[\boxed{\frac{A}{\sin\alpha}=\frac{B}{\sin\beta}=\frac{C}{\sin\gamma}}\].
\end{itemize}

推导如下:

如上图所示, 以 \[C\] 为底做高, 将原本的三角形分为左右两个直角三角形,
这条高利用左边的直角三角形可以表示为 \[A\sin\beta\],
利用右边的直角三角形则是 \[B\sin\alpha\], 于是有
\[A\sin\beta=B\sin\alpha\], 整理可得
\[\frac{A}{\sin\alpha}=\frac{B}{\sin\beta}\];
再做另一条高重复前面的操作, 便可得到完整的结论.

\textbf{余弦定律 Law of cosine}

还是先上结论:

\begin{itemize}
\tightlist
\item
  \[\boxed{B^2=A^2+C^2-2AC\cos\beta}\],
\end{itemize}

即, 【一条边的边长平方】等于【另两条边的边长平方之】和加上【两倍的
(另两条边边长的乘积) 乘以 (另两条边的夹角的余弦)】.

推导如下:

如下图所示, 依旧利用底边 \[C\] 上的高将其分为左右两个直角三角形;
左边的直角三角形, 利用斜边 \[A\] 和角 \[\beta\], 两直角边分别可以表示为
\[A\cos\beta\] 和 \[A\sin\beta\], 于是右边的直角三角形边长便可表述为
\[A\sin\beta\] 和 \[(C-A\cos\beta)\]; 对右边的直角三角形使用勾股定理

\[\begin{align*}B^2&=A^2\sin^2\beta+(C-A\cos\beta)^2\\
&=A^2\sin^2\beta+C^2+A^2\cos^2\beta-2AC\cos\beta\\
&=A^2+C^2-2AC\cos\beta.\end{align*}\]

其中等式的后两行用到了之前得出的 \[1=\cos^2\theta+\sin^2\theta\].

\hypertarget{ux4efbux610fux89d2ux5ea6ux7684ux4e09ux89d2ux51fdux6570}{%
\subsubsection{任意角度的三角函数}\label{ux4efbux610fux89d2ux5ea6ux7684ux4e09ux89d2ux51fdux6570}}

不难发现, 前面讨论的情况似乎都是锐角的情况 (主要是因为插图\ldots),
钝角的三角函数似乎没那么直观了, 因为做不成一个含有钝角的直角三角形,
没法简单地用边长比来表示 \[\sin\] 和 \[\cos\] 等. 于是,
我们需要想办法将前面的情形推广.

如下左图所示, 建立直角坐标系, 做一圆心位于原点的单位圆, 即半径为 \[1\]
的圆, 考虑在第一象限的圆上的一点, 将其与原点做连线, 将从
\[x\]-轴正方向与这条连线\textbf{顺时针}方向形成的夹角记作 \[\theta\],
不难看出这个点的坐标 \[(x,y)\] 满足

\[\begin{cases}x=\cos\theta,\\y=\sin\theta.\end{cases}\]

于是不妨将其他象限的情况也按此定义,
于是如上右图所示的钝角甚至更大角度的三角函数便可以被定义了.

\hypertarget{ux5f27ux5ea6ux5236-radian}{%
\subsubsection{弧度制 (Radian)}\label{ux5f27ux5ea6ux5236-radian}}

为什么一个周角是 \[360^\circ\] 呢, 听说过一个不可考的说法: \[360\]
是一个有很多因数的数字 (1, 2, 3, 4, 5, 6, 8, 9, 10, 12\ldots),
等分起来的时候数字会比较友好, 所以 \[360^\circ\] 其实是非常随意地规定的.
那么有没有更好的用来描述角度方法呢? 答案是弧度.

一个半径为 \[r\] 的圆的周长是 \[2\pi r\], 一个圆心角为 \[n^\circ\]
的扇形的弧长是 \[2\pi r\frac{n}{360}\]. 可见圆心角越大弧越长,
且圆心角和弧长成正比. 既然如此,
不如重新将角度定义为圆心角与弧长的比值以方便计算, 于是便有了,
在新的这套单位系统中, 若圆心角大小为 \[\theta\], 其对应弧长应为
\[r\theta\]; 当圆心角是一个周角时, 对应弧长便成了圆的周长 \[r(2\pi)\].
所以角度和这个新的单位的换算有 \[360^\circ\equiv 2\pi\ \text{rad}\],
因为这个单位把圆心角和对应的弧长联系起来了, 因此称之为\textbf{弧度}
(radian).

扇形面积在这套单位制, 即弧度制下, 便也成了 \[\frac{1}{2}r^2\theta\].

\hypertarget{ux4e09ux89d2ux51fdux6570ux7684ux56feux50cf}{%
\subsubsection{三角函数的图像}\label{ux4e09ux89d2ux51fdux6570ux7684ux56feux50cf}}

现在这个时代, 大家都或多或少能接触到科学计算器,
再不济在bing.com上搜索''solver''用微软的 Microsoft Solver
也可以计算某个特定角度的三角函数值, 自然也可以绘制函数图像.
下图分别展示了 \[\sin(x)\] 和 \[\cos(x)\] 的图像,

一些值得关注的点是它们都是\textbf{周期函数} (periodic function),
随着自变量-角度的变化, 因变量-函数值的变化是周期性的, 它们的周期都是
\[2\pi\], 这一点从上文的单位圆里便可看出些许原因,
当角度变化超过一个周角时, 和角度刚从 \[0\] 开始的情况是一样的.

一个个人很喜欢的可视化如下:

右下显示的是角度 \[\theta\] 不断增加, 左下的图可以看作右下的点的 \[y\]
坐标也就是 \[\sin\theta\] 的值的变化, 左上则是 \[x\] 坐标也就是
\[\cos\theta\] 的值的变化.

\input{ch/006}
\input{ch/007}
\input{ch/008}
\begin{quote}
\textbf{虚}中有\textbf{实}者, 或山穷水尽处, 一折而豁然开朗. - 沈复
『浮生六记』 Lesson 5: 最短的捷径就是绕远路,
绕远路才是我的最短捷径.\footnote{\emph{ジョジョの奇妙な冒険 Part7
  スティール・ボール・ラン}.}
\end{quote}

\hypertarget{ux6781ux5750ux6807-polar-coordinate}{%
\subsubsection{极坐标 (polar
coordinate)}\label{ux6781ux5750ux6807-polar-coordinate}}

其实在【005】中已经借用了一点极坐标的概念, 这里再正式地介绍一遍.
在平面直角坐标系里, 一个点所在的位置具有两个自由度 (degree of freedom),
因此一般不多不少需要两个独立变量来锚定这个点, 通常我们会有一个点的
\(x\)-轴坐标和 \(y\)-轴坐标来表示这个点的位置 \((x,y)\). 当然,
我们也可以在建立坐标系后, 利用和原点的距离 \(r\), 和一个方向,
例如【\(x\)-轴正方向】至【这个点和原点的连线】顺时针方向形成的夹角
\(\theta\), 来表示一个点的位置 \((r,\theta)\).

不难看出, 存在以下的转换

\(\begin{cases}x=r\cos\theta\\y=r\sin\theta\end{cases};\ \begin{cases}r=\sqrt{x^2+y^2}\\\theta=\arctan (y/x)\end{cases}.\)

上式中 \(\arctan\) 是 \(\tan\) 的逆运算, 即
\(\tan\theta=y/x\Rightarrow \theta=\arctan (y/x)\), 有时 \(\arctan\)
也记作 \(\tan^{-1}\).

\hypertarget{ux590dux5e73ux9762-complex-plane}{%
\subsubsection{复平面 (complex
plane)}\label{ux590dux5e73ux9762-complex-plane}}

考虑实数的时候, 有时我们会想象有一条数轴, 在【001】和【002】中,
我们将这条数轴添上了最初离散分布的整数,
然后又补上了似乎没有空隙的有理数, 最后才用全体实数彻底''填满''了.
在【006】中, 我们研究的对象再次扩展到了复数,
于是一条实轴似乎''放不下''这些复数的存在了, 我们便添加一条额外的,
与之前实数轴垂直的虚数轴, 如此一来便构成了一个复平面.

那么考虑一个复数 \((a+bi)\), 它的实部大小便是 \(a\), 虚部大小便是 \(b\),
仿照着实二维平面直角坐标系, 我们便可以在复平面上标出 \((a+bi)\)
对应的一个点. 既然如此, 我们可以继续仿照着极坐标, 表示出至原点的距离,
以及 \(x\)-轴正方向到这个点和原点的连线顺时针方向形成的夹角,
这两个量在复平面中分别被称作\textbf{模} (modulus, 合理怀疑有音译成分)
和\textbf{辐角} (argument), 记作

\$ \textbar a+bi\textbar=\sqrt{a^2+b^2},~\arg(a+bi)=\arctan (b/a).\$

\begin{quote}
像, 太像了.\footnote{\emph{让子弹飞}.}
\end{quote}

若是把模和辐角分别记作 \(r\) 和 \(\theta\), 则这个复数 \((a+bi)\)
便可以表示为

\((a+bi)=r\cos\theta+ir\sin\theta.\)

\hypertarget{ux590dux6570ux7684ux6307ux6570ux5f62ux5f0f}{%
\subsubsection{复数的指数形式}\label{ux590dux6570ux7684ux6307ux6570ux5f62ux5f0f}}

有那么一点跳脱, 怎么突然扯到指数了呢? 在【002】中,
其实我们只讨论过指数为有理数的情况, 当然指数是任意实数的情况, 例如
\(\sqrt{2}\) , 我们也可以利用 \(\sqrt{2}=1.414213...\) 去估算,
或者丢给一个科学计算器; 但是当指数是虚数或者复数时,
似乎情况就不大一样了\ldots 考虑自然常数为底数, 我们从自然常数的定义出发,
【008】中我们有

\(\lim_{n\rightarrow\infty}\left(1+\frac{1}{n}\right)^n\equiv\mathrm{e}.\)

不难推出, 对于任意实数 \(x\), 有

\(\lim_{n\rightarrow\infty}\left(1+\frac{x}{n}\right)^n=\mathrm{e}^x.\)

把上面结论推广到虚数,

\(\lim_{n\rightarrow\infty}\left(1+\frac{i}{n}\right)^n=\mathrm{e}^i.\)

如果上式让您感到不适 (uncomfortable), 您大可用二项式展开 (参见【007】)
来确认一下上述结果的正当性 (validity). 怎么理解这个 \(\mathrm{e}^i\) 呢,
不妨来看看它的模和辐角.

在此之前, 我们还需要一些小\textbf{引理} (lemma \footnote{\textbf{公理/假定}
  (axiom/postulate): 默认为真无需证明的陈述. \textbf{定义} (definition):
  准确无歧义的对一个术语的描述. \textbf{定理} (theorem):
  证明为真的大结论. \textbf{引理} (lemma): 为了证明定理的小结论.
  \textbf{推论} (Corollary): 借助定理可简短地证明的结论.
  一本严格的数学书经常会出现前面这些令人畏惧的词,
  类似的还有\textbf{命题} (proposition), \textbf{推测/猜想}
  (conjecture), \textbf{断言} (claim)等等.}, 区别于 llama - 大羊驼,
好冷), 这里没有严谨证明, 仅提供一个思路:

\textbf{引理1}: \(\boxed{|(a+bi)^n|=|a+bi|^n}\).

\begin{quote}
当 \(n=2\) 时, \((a+bi)^2=a^2-b^2+2abi\). 于是

\(\begin{aligned}|(a+bi)^2|=&\sqrt{(a^2-b^2)^2+(2ab)^2}\\=&\sqrt{(a^2+b^2)^2}\\=&\sqrt{(a^2+b^2)}^2\\=&|a+bi|^2.\end{aligned}\)

用数学归纳法 (关于数学归纳法, 可以参见【007】中的一个实例)
便不难得出结论.
\end{quote}

\textbf{引理2}: \(\boxed{\arg((a+bi)^n)=n\cdot\arg(a+bi)}\).

\begin{quote}
这里需要引入一个新的视角, 我们把复数看作一个''作用'',
将一个复数作用在某一个数 \(z\) 上, 在复平面上事实上是将 \(z\)
对应的点关于原点旋转了. 考虑 \(i\times1\), 即将 \(i\) 作用到 \(1\) 上,
得到了 \(i\), 相当于逆时针旋转了 \(90^\circ\); 继续作用 \(i\), 得到了
\(-1\), 相当于又逆时针旋转了 \(90^\circ\)\ldots{} 于是用 \(i\) 作用
\(n\) 次, 即作用了 \(i^n\), 相当于将其作用对象, 关于原点, 旋转了 \(i\)
的辐角 \(90^\circ\) \(n\) 次, 便是旋转了 \(n90^\circ\). ( \(90^\circ\)
用 \(\pi/2\) 弧度表述其实会更好, 关于弧度可以参考【005】,
下文若无额外说明, 角度皆用弧度制).

这个结论可以推广到其他任意复数, 便有了上述结论.
\end{quote}

事实上, 利用辐角和模来表述一个复数, 上述两则引理可以总结成一条:

\textbf{引理1+2}:
\(\boxed{\left[r(\cos\theta+i\cos\theta)\right]^n=r^n(\cos n\theta+i\sin n\theta)}\).

那么来看 \(\mathrm{e}^i\) , 它的模

\(\begin{aligned} |\mathrm{e}^i|=&\left|\lim_{n\rightarrow\infty}\left(1+\frac{i}{n}\right)^n\right|\\ =&\lim_{n\rightarrow\infty}\left|\left(1+\frac{i}{n}\right)^n\right|\\ =&\lim_{n\rightarrow\infty}\sqrt{1+\frac{1}{n^2}}^n\\ =&\lim_{n\rightarrow\infty}\left(1+\frac{1}{n^2}\right)^{n/2}\\ =&\lim_{n\rightarrow\infty}\left[\left(1+\frac{1}{n^2}\right)^{n^2}\right]^{1/2n}=\lim_{n\rightarrow\infty}\mathrm{e}^{1/2n}=1. \end{aligned}\)

上式第二行到第三行利用了引理1, 最后一行先是利用了自然常数的定义, 然后当
\(n\rightarrow\infty\) 便有 \(1/2n\rightarrow0\), 于是
\(\mathrm{e}^0=1\). 再来看辐角

\(\begin{aligned} \arg(\mathrm{e}^i)=&\lim_{n\rightarrow\infty}n\arg\left(1+\frac{i}{n}\right)\\ =&\lim_{n\rightarrow\infty}n\arctan\frac{1}{n}=\lim_{n\rightarrow\infty}n\frac{1}{n}=1.\\ \end{aligned}\)

上式先是利用了引理2, 然后当 \(n\rightarrow\infty\) 时
\(1/n\rightarrow0\), 然后在这个极限下 (啊, 还是,
极限我们晚点再稍严格地讨论) 有
\(\arctan\frac{1}{n}\rightarrow\frac{1}{n}\).

\begin{quote}
当然也可以用\textbf{小角度近似} (small angle approximation) 来理解
(注意, 是理解不是证明) 这个过程. 小角度近似即, 使用弧度制时, 当
\(\theta\ll1\), 有 \(\theta\approx\sin\theta\approx\tan\theta\).
\end{quote}

这样利用模和辐角, 我们有

\(\mathrm{e}^i=\cos1+i\sin1\).

所以 \(\mathrm{e}^i\) 在复平面上对应一个距离原点 \(1\), 与原点连线和
\(x\)-轴夹角是 \(1\) 的点, 或者说 \(\mathrm{e}^i\) 是一个实部是
\(\cos1\), 虚部是 \(\sin1\) 的复数. 同理可得 \(|\mathrm{e}^{ib}|=1\),
\(\arg(\mathrm{e}^{ib})=b\), 进而

\(\mathrm{e}^{ib}=\cos b+i\sin b\).

早一些地结论
\(\lim_{n\rightarrow\infty}\left(1+\frac{i}{n}\right)^n=\mathrm{e}^i\)
其实可以推广到任意复数 \((a+ib)\), 即
\(\lim_{n\rightarrow\infty}\left(1+\frac{a+ib}{n}\right)^n=\mathrm{e}^{a+ib}=\mathrm{e}^{a}\mathrm{e}^{ib}\),
那么

\(\boxed{\mathrm{e}^{a+ib}=\mathrm{e}^a(\cos b+i\sin b)}\)\footnote{a+ib
  取 iπ 便可以得到著名的欧拉公式, 确实非常美丽, 虚实正负在此交汇,
  圆周率和自然常数也藏于其中.}.

\section{三角函数的指数形式}\label{010}

\begin{flushright}{\kaishu (有名、无名) 此两者同出而异名, 同谓之玄, 玄之又玄, 众妙之门.}\end{flushright}

前面介绍三角函数 (\ref{005}\nameref{005}) 的时候, 跳过了一些比较重要的\textbf{恒等式}
(identities), 比如: $\cos (a\pm b)=?$, $\sin (a\pm  b)=?$

事实上这些恒等式可以从纯几何出发去推导,
或者也可以用所谓诱导公式\footnote{有一说``诱导''是谬译, induction
  在数学中更常译作``归纳''.} (induction formula, 即三角函数的周期性),
再或者, 在\ref{009}\nameref{009}中我们发现三角函数和指数函数有着联系,
我们也可以利用这层关系.

\begin{tcolorbox}[size=fbox, breakable, enhanced jigsaw, title={三角函数的指数形式}]

先前我们有,

$\mathrm{e}^{i\theta}=\cos \theta+i\sin \theta$,

代入 $\theta:=-\theta\  $\footnote{这里仿照了编程里常用的记法,
  在很多语言中程序猿可能会写 a = a + 1, 这行指令并不表示 a 等于 a + 1,
  而表示将 a 这个变量赋值它原先的值加一, 例如若起先有 a = 1,
  那么在执行了 a = a + 1 后, a 变为了 a = 1 + 1 = 2.}, 便有
$\mathrm{e}^{-i\theta}=\cos (-\theta)+i\sin (-\theta)$,
利用三角函数的周期性可得,

$\mathrm{e}^{-i\theta}=\cos \theta-i\sin \theta$,

将前两式相加便可消去 $\sin\theta$ 项, 相减便可消去 $\cos\theta$ 项,
化简便可得

$\boxed{\cos\theta=\frac{\mathrm{e}^{i\theta}+\mathrm{e}^{-i\theta}}{2},\ \sin\theta=\frac{\mathrm{e}^{i\theta}-\mathrm{e}^{-i\theta}}{2i}}$.

\end{tcolorbox}

\begin{tcolorbox}[size=fbox, breakable, enhanced jigsaw, title={利用三角函数的指数形式推导三角函数恒等式}]

以 $\cos (a+b)$ 为例:

$\begin{aligned}&\cos(a+b)\\
=&\frac{\mathrm{e}^{i(a+b)}+\mathrm{e}^{-i(a+b)}}{2}\\
=&\frac{2\mathrm{e}^{i(a+b)}+2\mathrm{e}^{-i(a+b)}}{4}\\
=&\frac{\mathrm{e}^{i(a+b)}{+\mathrm{e}^{i(a-b)}+\mathrm{e}^{i(-a+b)}}+\mathrm{e}^{-i(a-b)}}{4}\\
&+\frac{\mathrm{e}^{i(a+b)}{-\mathrm{e}^{i(a-b)}-\mathrm{e}^{i(-a+b)}}+\mathrm{e}^{-i(a-b)}}{4}\\
=&\frac{\mathrm{e}^{ia}+\mathrm{e}^{-ia}}{2}\frac{\mathrm{e}^{ib}+\mathrm{e}^{-ib}}{2}-\frac{\mathrm{e}^{ia}-\mathrm{e}^{-ia}}{2i}\frac{\mathrm{e}^{ib}-\mathrm{e}^{-ib}}{2i}\\
=&\cos a\cos b-\sin a\sin b.\end{aligned}$

类似的, 可以得到

$\boxed{\begin{aligned}
&\cos(a\pm b)=\cos a\cos b\mp \sin a \sin b,\\
&\sin(a\pm b)=\sin a\cos b\pm \cos a\sin b.    
\end{aligned}}$

另外还有常用的二倍角公式, 可令上式中的 $a=b$ 得到,

$\boxed{\begin{aligned}
&\sin2a=2\sin a\cos a,\\
&\cos2a=\cos^2a-\sin^2a=2\cos^2a-1=1-2\sin^2a,    
\end{aligned}}$

第二个等式利用了 $\sin^2\theta+\cos^2\theta=1$. 再还有有时会有半角,
可以用 $\cos$ 的二倍角公式推导, 例如, 求 $\cos$ 的半角公式可以令
$\cos$ 的二倍角公式中的 $a:=a/2$,

$\cos a=2\cos^2(a/2)-1$,

整理可得

$\boxed{\cos\frac{a}{2}=\pm\sqrt{\frac{1+\cos a}{2}}}.$

类似的

$\boxed{\sin\frac{a}{2}=\pm\sqrt{\frac{1-\cos a}{2}}}.$

$\tan$ 相关的公式则可以利用 $\tan=\sin/\cos$ 求得.

\end{tcolorbox}
\chapter{微积分 (calculus)}
\begin{flushright}{\kaishu 天\ 工\ 开\ 物\ !\\「Made in Heaven!」要开始加速了\footnote{ジョジョの奇妙な冒险 Part 6
  スト-ンオ-シャン}}\end{flushright}

\section{极限和连续性}\label{011}

跳过一些内容: 集合 (set), 点集拓扑 (point set topology), 数列与级数
(sequence and series); 如果您希望习得更加严谨的数学语言, 那么可以移步
Walter Rudin - \emph{Principles of Mathematical Analysis} (俗称 baby
Rudin\footnote{Baby = 数学分析原理 \emph{Principles of Mathematical
  Analysis}; Papa/Big = 实分析与复分析 \emph{Real and Complex Analysis};
  Grandpa = 泛函分析 \emph{Functional Analysis}; 好好地学严格的数学,
  逃不掉这三本分析, 这个系列也就看个乐子.}).

\subsection{极限 (limit)}

\begin{tcolorbox}[size=fbox, breakable, enhanced jigsaw, title={定义}]
当 $x$ 趋向于 $p$ 时, $x\rightarrow p$ , $f(x)$
趋向于 $q$, $f(x)\rightarrow q$ , 记作
$\lim_{x\rightarrow p}f(x)=q.$

用 $\epsilon - \delta$ 语言 (出现了!) 来说,
$\lim_{x\rightarrow p}f(x)=q$ 便是: 对于任意的 $\epsilon>0$ , 存在
$\delta>0$ 使得【若 $0<|x-p|<\delta$, 便有 $|f(x)-q|<\epsilon$
】\footnote{其实这里的定义还是很不严谨, 比如没有说明定义域和值域.
  为了省笔墨下文都默认取值范围在合适的区间内.}.

\end{tcolorbox}

为什么要用 $\epsilon - \delta$ 语言? 原本的``趋向于''其实很不严格,
什么叫趋向于呢, 于是 $\epsilon - \delta$ 语言如是说道:

\begin{itemize}

\item
  我先任意选定一个 $\epsilon$,
\item
  然后我要试着找到一个 $\delta$,
\item
  使得 $x$ 与 $p$ 足够接近时 - 有多接近呢? 它们差的绝对值
  (或者说``距离'', 不过这边还没定义距离, 233) 小于 $\delta$ - 便有
  $f(x)$ 与 $q$ 足够接近 - 多近呢? 他们差的绝对值小于 $\epsilon$.
\item
  若对于任意小的 $\epsilon$, 总能找到一个这样的 $\delta$,
  那么便可以放心地说, 确有$\lim_{x\rightarrow p}f(x)=q$.
\end{itemize}

\begin{tcolorbox}[size=fbox, breakable, enhanced jigsaw]
  \includegraphics[width=0.45\textwidth]{img/image-20230503152041842.png}
\end{tcolorbox}

\begin{newquote}
\textbf{例子}: 一个平凡的情况 (a trivial case), $f(x)=ax$, 证明
$\lim_{x\rightarrow 1}f(x)=a$.

\textbf{思路} (草稿纸上或者脑子里的部分): 对于任意 $\epsilon>0$
我们需要找到 $\delta>0$, 满足当 $0<|x-1|<\delta$ 时,
$|f(x)-a|<\epsilon$ 成立.

\begin{itemize}
\item
  $|f(x)-a|=|ax-a|=a|x-1|$
\item
  令上式小于 $\epsilon$, 发现有 $|x-1|<\epsilon/a$
\item
  令 $\delta=\epsilon/a$ , 即可出锅食用 (bushi).
\end{itemize}

\textbf{证明} (写下来的正式的书面的部分): 对任意 $\epsilon>0$ , 令
$\delta=\epsilon/a$, 则当 $0<|x-1|<\delta$ 时, 有

\begin{itemize}

\item
  $|f(x)-a|=|ax-a|=a|x-1|<a\delta<\epsilon$.
\item
  于是根据极限的定义, $\lim_{x\rightarrow 1}f(x)=a$
\end{itemize}

Q.E.D\footnote{Quod erat demonstrandum - 这被证明了.}

\textbf{吐槽}: 鄙人学分析的时候学得就很不到位,
写证明的时候常常向同学``借鉴'', 时常觉得, 证明本身并不难写,
难得是想到并``构造''出一些证明需要的东西, 就如在上面的例子中构造一个
$\delta=\epsilon/a$;
殊不知``借鉴''的那些作业其实只有上面例子中【证明】的部分,
而【思路】部分被写在草稿纸上丢掉了.
这种狡猾如雪地上的狐狸一般用尾巴扫去自己的踪迹的行为\ldots{}
于是有这样的说法:

一位菲尔兹得主告诉我, 顶级的数学家们会秘密地像物理学家一样思考,
等他们得到证明的一个大框架之后, 他们再用 epsilon 和 delta
的语言把证明过程包装起来.
\end{newquote}

\begin{itemize}

\item
  若 $f(x)$ 在 $x\rightarrow p$ 处存在极限,
  这个极限是\textbf{唯一}的 (unique).
\end{itemize}

考虑 $\lim_{x\rightarrow p}f(x)=a$, $\lim_{x\rightarrow p}g(x)=b$,
极限还存在以下规律 (啊, 美好的线性) :

\begin{itemize}

\item
  $\lim_{x\rightarrow p}(f\pm g)(x)=a\pm b$;
\item
  $\lim_{x\rightarrow p}(fg)(x)=ab$;
\item
  $\lim_{x\rightarrow p}\frac{f}{g}(x)=\frac{a}{b}$, 若 $b\neq 0$.
\end{itemize}

\begin{newquote}
回收一个坑. 【\ref{002}\nameref{002}】中提到过 $\mathrm{0}^0$ 是未被定义的问题,
一种解释便是, 若希望用极限来定义它的取值, 那么应该从
$\lim_{x\rightarrow0}x^0$ 出发, 得到 $1$, 还是应该从
$\lim_{x\rightarrow0}0^x$ 出发, 得到 $0$?
\end{newquote}


\subsection{连续性 (continuity)}

\textbf{定义}: 对于一函数 $f(x)$, 对于任意的 $\epsilon>0$ , 存在
$\delta>0$ 使得【对于某个特定的 $x_0$, 若有 $x$ 满足
$|x-x_0|<\delta$, 便有 $|f(x)-f(x_0)|<\epsilon$】, 那么我们便可以说,
$f(x)$ 在 $x_0$ 处连续.

\begin{itemize}

\item
  若 $f(x)$ 和 $g(x)$ 连续, 那么 $f(g(x))$ 也连续.
\item
  若 $f(x)$ 和 $g(x)$ 连续, 那么 $(f\pm g)(x)$, $fg(x)$,
  $\frac{f}{g}(x)$ 都连续, 最后一条要求 $g(x)$ 对于任意 $x$ 不为
  $0$.
\end{itemize}

从连续性出发, 可以得到以下几个定理, 图像上非常直观, 这边暂时忽略严格证明.

\begin{tcolorbox}[size=fbox, breakable, enhanced jigsaw, title={极值定理 (extreme value theorem)}]

\begin{tcolorbox}[size=fbox, breakable, enhanced jigsaw, sidebyside]
\includegraphics[width=0.9\textwidth]{img/image-20230503152124313.png}
\tcblower
\kaishu{\small 若函数 $f(x)$ 在区间 $[a,b]$ 连续, 则 $f(x)$ 必然在区间 $[a,b]$
存在最大值和最小值.}
\end{tcolorbox}

\end{tcolorbox}

\begin{tcolorbox}[size=fbox, breakable, enhanced jigsaw, title={介值定理 (intermediate value theorem)}]

\begin{tcolorbox}[size=fbox, breakable, enhanced jigsaw, sidebyside]
\includegraphics[width=0.9\textwidth]{img/image-20230503152147199.png}
\tcblower
\kaishu{\small 若函数 $f(x)$ 在区间 $[a,b]$ 连续, 且有 $f(a)<C<f(b)$ 或
$f(a)>C>f(b)$, 那么总是存在 $c\in(a,b)$ 或者说 $a\le c\le b$ 使得
$f(c)=C$.}
\end{tcolorbox}

\end{tcolorbox}

\begin{tcolorbox}[size=fbox, breakable, enhanced jigsaw, title={零点定理 (zero theorem)}]

\begin{tcolorbox}[size=fbox, breakable, enhanced jigsaw, sidebyside]
\includegraphics[width=0.9\textwidth]{img/image-20230503152707849.png}
\tcblower
\kaishu{\small 若函数 $f(x)$ 在区间 $[a,b]$ 连续, 且$f(a)f(b)<0$, 则存在
$c\in(a,b)$ 使得 $f(c)=0$.}
\end{tcolorbox}

\end{tcolorbox}

\section{导数}\label{012}

初中阶段常有求二次函数切线的题目,
利用初中知识求切线的表达式往往过程冗长, 事实上, \textbf{微积分}
(calculus) 会大大简化这个过程. 微积分, 如其名所示, 分为\textbf{微分}
(differentiation) 和\textbf{积分 }(integration).

在这里我们姑且先用经典的角度来了解微积分. \textbf{微分} (differential)
可以不严谨的理解为\textbf{无穷小量} (infinitesimal),
这是一些物理教材中更常见的表述, 例如某函数 $y=f(x)$,
它的一个有限的变化通常记作 $\Delta y$, 这个小三角念作``delta'',
在很多场景下表示变化, 而一个无穷小量的变化便记作 $\mathrm{d}y$.
任何实数都比无穷小量大, 就像无穷大 $\infty$ 大于任何实数\footnote{【\ref{007}\nameref{007}】中有提到过``浮点数'', 例如一个有理数 a,
它不会被以分数的形式记录, 而是记录其小数形式并精确到某一位,
那么第一个非零位数低于这一位的数字 b, 它对于 a 来说在计算机数值上看来,
便等效于无穷小量. 在 MATLAB 中, 给定一个实数 a, 利用函数 eps(), eps(a)
便会输出对于 a 来说的``无穷小量''.}.

另有一个非常类似的概念叫做\textbf{导数}, 导数可以理解为函数的变化率.
考虑某函数 $y=f(x)$ 在一个很小但是有限的一个定义域区间 $[a,b]$,
有自变量的变化为 $\Delta x=b-a$, 函数值的变化为
$\Delta y= f(b)-f(a)$, 于是在此区间内``平均''变化率是

$\frac{\Delta y}{\Delta x}=\frac{f(b)-f(a)}{b-a}.$

若是希望得到 $x=a$ 处的瞬时变化率, 我们令 $h:=b-a$, 直觉上应该是
$\lim_{h\rightarrow0}[f(a+h)-f(a)]/h$, 这便是函数 $f(x)$ 在 $x=a$
处的变化率, 或 $f(x)$ 在 $x=a$ 处的导数. 推广到任意位置 $x$, 遂有:

\begin{tcolorbox}[size=fbox, breakable, enhanced jigsaw, title={定义}]
函数 $f(x)$ 的导(函)数\footnote{取某个特定的点 $x=x\_0$,
  得到的导数是一个具体数值, 便叫它导数; 若不将 $x$ 锚定到一个特定的值,
  导数便依旧是一个含 $x$ 的函数, 便叫它导函数. 下文不再区分. 类似的,
  后文再出现的微分, 导数和无穷小量也不再严格区分.}为

$\lim_{h\rightarrow0}\frac{f(x+h)-f(x)}{h}.$
\end{tcolorbox}

很多数学教材会这么标记导数

$\boxed{f'(x)=\lim_{h\rightarrow0}\frac{f(x+h)-f(x)}{h}}.$

\begin{tcolorbox}[size=fbox, breakable, enhanced jigsaw, sidebyside]
\includegraphics[width=0.9\textwidth]{img/image-20230614124234964.png}
\tcblower
\kaishu{\small 图像上, 不难看出, 当 $h$ 越来越小时, 两点的连线越来越趋向于切线, 当取到极限 $h\rightarrow0$ 时, 取到的这个值, 也就是导数, 便是切线的斜率.}
\end{tcolorbox}

物理教材更偏好

$\frac{\mathrm{d}y}{\mathrm{d}x}=\lim_{h\rightarrow0}\frac{f(x+h)-f(x)}{h}.$

\begin{newquote}
举一个例子, 若一个物体非匀速运动, 将速度 $v$ 表示为一个关于时间 $t$
的函数 $v(t)$, 那么便有加速度 (acceleration) 即速度的变化率:
$a(t)=v'(t)=\frac{\mathrm{d}v}{\mathrm{d}t}$.

另: 很多时候, 物理中, 关于时间的求导还有一个标记,
$\dot{f}\equiv \frac{\mathrm{d}f}{\mathrm{d}t}$\footnote{很多人经常吐槽物理中标记的不统一.
  个人当然也觉得, 如果有一套统一的标记, 信息的沟通自然会便利不少;
  但是学习的过程中, 既然标记不统一已经客观存在, 与其花时间吐槽,
  不如去抓住本质, 不要拘泥于标记, 而去理解标记背后的含义.}.
\end{newquote}

类似 $\frac{\mathrm{d}y}{\mathrm{d}x}$ 这样记法的好处是,
``变化率''这个概念被表现得很直观, 若有
$\frac{\mathrm{d}y}{\mathrm{d}x}=y'(x)$, 我们可以将它改写为
$\mathrm{d}y=y'(x)\mathrm{d}x$, 虽然``两边同乘 $\mathrm{d}x$''
这个说法非常不正确的, 但是从``变化率''的角度出发的确可以这么理解.

\begin{tcolorbox}[size=fbox, breakable, enhanced jigsaw, title={定理}]
若 $f(x)$ 在 $x=c$ 存在导数 (我们也可以说, 它在
$x=c$ 处可以被求导), 那么它在 $x=c$ 处连续.
\end{tcolorbox}

(不严格的) {\kaishu 证明}: 不使用 $\epsilon - \delta$ 语言的话,
只需证明 $\lim_{x\rightarrow c}f(x)=f(c)$ 即可. 于是, 对于有限的
$h$, $f(c+h)=f(c)+f(c+h)-f(c)=f(c)+\frac{f(c+h)-f(c)}{h}\cdot h$,
取极限 $h\rightarrow0$ 有
$\lim_{h\rightarrow0}f(c+h)=\lim_{h\rightarrow0}f(c)+\lim_{h\rightarrow0}\frac{f(c+h)-f(c)}{h}\cdot h=f(c)+f'(c)\cdot 0$,
第一个等号利用了极限的线性, 第二个等号则需要导数存在,
于是便有存在导数隐含 (imply) 连续.

若有函数 $f(x)$ 和 $g(x)$, 导数的一些性质:

\begin{itemize}

\item
  线性: $(af+bg)'(x)=af'(x)+bg'(x)$, 这里 $a$ 和 $b$ 是常数;
\item
  乘法法则 (product rule) : $(fg)'(x)=f'(x)g(x)+f(x)g'(x)$;
\item
  除法法则 (quotient rule) : $\left(\frac{f}{g}\right)'(x)=\frac{f'(x)g(x)-f(x)g'(x)}{g^2(x)}$,
  若 $g(x)\neq 0$.
\end{itemize}

第一条性质可以由极限的线性而来; 第二条和第三条可以分别令
$h(x)=f(x)g(x)$ 和 $h(x)=\frac{f(x)}{g(x)}$, 然后将 $h(x)$
代入导数的定义.

下面是一个非常 trivial 的例子,

\begin{newquote}
\textbf{例子}: $y(x)=x^2$, 求 $y'(x)$. 利用定义:

$\begin{aligned}y'(x)=&\lim_{h\rightarrow0}\frac{(x+h)^2-x^2}{h}\\=&\lim_{h\rightarrow0}\frac{x^2+h^2+2xh-x^2}{h}\\=&\lim_{h\rightarrow0}(2x+h^2)\\=&2x.\end{aligned}$
\end{newquote}

不难将结果推广为

$\boxed{\frac{\mathrm{d}}{\mathrm{d}x}x^n=nx^{n-1}}.$

\begin{newquote}
这样一来, 初中阶段得求二次函数在某点的切线就很 trivial 了. 例如: 求 $y (x)=ax^2+bx+c$, 在 $x=d$ 处的切线的表达式.

先对函数进行求导, $y'(x)=2ax+b$, 代入 $x=d$, 得到在 $x=d$ 处函数图像切线的斜率, $y'(c)=2ad+b$.

再将 $x=c$ 代入原函数, 得到, $y(c)=ad^2+bd+c$, 即切线的还经过 $(d, ad^2+bd+c)$ 这个点.

已知切线经过点 $(d, ad^2+bd+c)$, 且斜率为 $(2ad+b)$, 易得该切线的表达式.
\end{newquote}
\hypertarget{ux5e38ux89c1ux51fdux6570ux7684ux5bfcux6570}{%
\subsubsection{常见函数的导数}\label{ux5e38ux89c1ux51fdux6570ux7684ux5bfcux6570}}

\textbf{指数函数}

从指数函数开始, 首先考虑自然常数作为底数的情况, 代入导数的定义: \$\$

\begin{aligned}
\frac{\mathrm{d}}{\mathrm{d}x}(\mathrm{e}^x)=&\lim_{h\rightarrow0}\frac{\mathrm{e}^{x+h}-\mathrm{e}^x}{h}\\

=&\lim_{h\rightarrow0}\mathrm{e}^x\left(\frac{\mathrm{e}^h-1}{h}\right)\\
\\
&\text{令 }n:=1/h\\
\\
=&\lim_{n\rightarrow\infty}\mathrm{e}^x\left(\frac{\mathrm{e}^{1/n}-1}{1/n}\right)\\
\\
&\text{根据定义 }\mathrm{e}\equiv\lim_{n\rightarrow\infty}\left(1+\frac{1}{n}\right)^n\\
\\
=&\lim_{n\rightarrow\infty}\mathrm{e}^x\left(\frac{{\left(\left(1+1/n\right)^n\right)}^{1/n}-1}{1/n}\right)\\

=&\lim_{n\rightarrow\infty}\mathrm{e}^x\left(\frac{1+1/n-1}{1/n}\right)\\

=&\mathrm{e}^x.
\end{aligned}

\$\$

于是 \[
\boxed{\frac{\mathrm{d}}{\mathrm{d}x}\mathrm{e}^x=\mathrm{e}^x}.
\]

对于其他的底数, 例如 \(a^x\), 我们需要一些额外的知识了:

\textbf{链式法则 chain rule}

结论上有: \[
\boxed{\frac{\mathrm{d}y}{\mathrm{d}x}=\frac{\mathrm{d}y}{\mathrm{d}u}\frac{\mathrm{d}u}{\mathrm{d}x}}.
\]

\begin{quote}
即原先有一个函数 \(y=h(x)\), 将它改写为一个符合函数 \(y=f(g(x))\), 并令
\(u:=g(x)\).

不严格的直觉上的证明:

\(\Delta u=g(x+\Delta x)-g(x)\), \(\Delta y=f(u+\Delta u)-f(u)\). 于是
\(\frac{\Delta y}{\Delta x}=\frac{\Delta y}{\Delta u}\frac{\Delta u}{\Delta x}\).
再取 \(\Delta x\rightarrow0\) 的极限, 若 \(g(x)\) 是连续的, 在
\(\Delta x\rightarrow0\) 时便有 \(\Delta u\rightarrow 0\),
于是便得到了上述结论.
\end{quote}

现在我们再来看 \(a^x\), 我们可以先利用【002】的知识换个底数,
\(a^x=\mathrm{e}^{\ln(a)x}\), 再利用链式法则, 令 \(y:=\mathrm{e}^u\)
\(u:=\ln{(a)}x\), 于是 \$\$

\begin{aligned}
\frac{\mathrm{d}}{\mathrm{d}x}(a^x)=&\frac{\mathrm{d}y}{\mathrm{d}u}\frac{\mathrm{d}u}{\mathrm{d}x}\\

=&(\mathrm{e}^u)(\ln{a})\\

=&(\ln{a})\mathrm{e}^{\ln{a}}=(\ln{a})a^x.
\end{aligned}

\[
更加通常 (generalize) 一点, 我们可以有
\]
\boxed{\frac{\mathrm{d}}{\mathrm{d}x}a^{u(x)}=(\ln{a})a^u\frac{\mathrm{d}u}{\mathrm{d}x}}.
\$\$ \textbf{三角函数}

先考虑 \(\sin\), \$\$

\begin{aligned}
\frac{\mathrm{d}}{\mathrm{d}x}(\sin(x))=&\lim_{h\rightarrow0}\frac{\sin(x+h)-\sin(x)}{h}\\
\\
& \text{ 参见【010】, }\sin(a\pm b)=\sin a\cos b\pm \cos a\sin b\\
\\
=&\lim_{h\rightarrow0}\frac{\sin(x)\cos(h)+\cos(x)\sin(h)-\sin(x)}{h}\\

=&\lim_{h\rightarrow0}\left(\frac{\sin(x)(\cos(h)-1)}{h}+\frac{\cos(x)\sin(h)}{h}\right),
\end{aligned}

\$\$

到这一步似乎就不很直观了, 若是直接取 \(h=0\), 则有 \(\cos(h)-1=0\) 和
\(\sin(h)=0\), 两项的分子分母都同时为零了, 类似这种出现了
\(\frac{0}{0}\) 或者 \(\frac{\infty}{\infty}\) 的情况称为不定式/未定型
(indeterminate forms), 后面我们会看到将有一种更便 (简单) 捷 (粗暴)
的方式来解决这样的问题 (剧透: 洛必达法则 - L'Hôpital's rule),
目前我们先老老实实地来解决. 对于上面的形式, 我们可以用几何方法入手:

如上图所示, 考虑一个单位圆, 三角形 \(OAB\), 扇形 \(OAD\), 和三角形
\(OCD\) 的面积关系是 \(\sin(\theta)/2<\theta/2<\tan(\theta)/2\);
将每个式子都除以 \(\sin(\theta)/2\), 有
\(1<\theta/\sin(\theta)<1/\cos(\theta)\); 再取倒数, 得到
\(1>\sin(\theta)/\theta>\cos\theta\).
这个结论至少在第一象限应该是成立的, 于是当我们取 \(\theta=0\) 时,
我们发现不等式的最右边也是 \(1\) 了, 直觉上, 中间一项既要比 \(1\) 小,
又要比 \(1\) 大, 那么它只能是等于 \(1\) 了. 这个朴素的想法被叫做
``三明治定理'' (Sandwich Theorem, 也叫夹逼定理\ldots).

这样一来, 便有当 \(h\rightarrow0\), \(\sin(h)/h\rightarrow 1\),
另一项的话, 也不难利用三角函数的恒等式变形: \[
\begin{aligned}
&\frac{\cos(h)-1}{h}\\
\\
&\text{参见【010】}\cos2a=1-2\sin^2a\\
\\
=&-\frac{2\sin^2(h/2)}{h}\\
\\
&\text{令 }\theta:=h/2
\\
=&-\frac{\sin(\theta)}{\theta}\cdot\sin(\theta),
\end{aligned}
\]

于是当 \(h\rightarrow0\), \(\theta\rightarrow0\), \[
\begin{aligned}
&\frac{\cos(h)-1}{h}\\
=&-\frac{\sin(\theta)}{\theta}\cdot\sin(\theta)\\
=&-(1)\cdot(0)\\=&0.
\end{aligned}
\] 综上, \[
\boxed{\frac{\mathrm{d}}{\mathrm{d}x}(\sin(x))=\cos(x)}.
\] 类似的, 不难推出 \[
\boxed{\frac{\mathrm{d}}{\mathrm{d}x}(\cos(x))=-\sin(x)}.
\]

\input{ch/014}
\section{导数的应用}\label{015}

先前我们有利用导数求函数的切线过, 这里我们继续来看导数还有那些应用.

\begin{tcolorbox}[size=fbox, breakable, enhanced jigsaw, title={一次导和极值}]

导数可以视作切线的斜率, 所以可以利用导数的正负判断函数的增减性,
当导数大于$0$ 时, 函数是递增的, 当导数小于$0$ 时, 函数是递减的.

若函数 $f(x)$ 有局域的最大或最小值 (统称极值) 在 $x=c$ 处, 且
$f'(x)$ 在 $x=c$ 处是被定义的, 那么 $f'(c)=0$.

\begin{newquote}
不太严谨的证明:

\begin{itemize}

\item
  考虑 $x=c$ 附近足够小的邻域 (neighborhood) , 使得在此领域内
  $f'(x)$ 是连续的;
\item
  因为 $f(c)$ 是极值: 则在 $x=c$ 的一侧, 有 $f'(x)>0$
  (切线斜率为正, 函数递增); 另一侧, 有 $f'(x)<0$ (切线斜率为负,
  函数递减);
\item
  利用零点定理 (参见【\ref{011}\nameref{011}】) , 这个领域中必然有一点使得 $f'(x)=0$,
  利用极限可得这个点恰好在 $x=c$ 上.
\end{itemize}
\end{newquote}

\begin{newquote}
一个 trivial 的例子:

利用上述性质, 我们来找一下二次函数的顶点.

若有函数 $f(x)=y=ax^2+bx+c$, 于是 $f'(x)=2ax+b$, 当 $f'(x)=0$,
$2ax+b=0$, 于是 $x=\frac{b}{2a}$. 将 $x=\frac{b}{2a}$ 代入原函数,
得到 $y=f\left(\frac{b}{2a}\right)=\frac{4ac-b^2}{4a}$.

这个结论和配方得到的结论是一致的
$y=a\left(x-\frac{b}{2a}\right)^2+\frac{4ac-b^2}{4a}$.
\end{newquote}

对于更复杂的函数, 也可以用同样的思路来求极值

\begin{itemize}

\item
  先对函数求导;
\item
  令导数为 $0$, 求得导数为 $0$ 处的自变量取值;
\item
  将求得的自变量取值代入原函数, 得到函数极值.
\end{itemize}

\end{tcolorbox}

\begin{tcolorbox}[size=fbox, breakable, enhanced jigsaw, title={二次导和凹凸性}]

我们可以对导数继续求导, 二次导可以这么记: $f''(x)\equiv \frac{\mathrm{d}^2f}{\mathrm{d}x^2}\equiv \frac{\mathrm{d}}{\mathrm{d}x}\frac{\mathrm{d}f}{\mathrm{d}x}$. 考虑一个有二次导的的函数\footnote{有的学科里会称为 $C^2$ 连续.}:

\begin{itemize}

\item
  若 $f''(x)>0$, 则函数是上凹的 (concave up);
\item
  若 $f''(x)<0$, 则函数是下凹的 (concave down);
\end{itemize}

\begin{tcolorbox}[size=fbox, breakable, enhanced jigsaw, sidebyside]
\includegraphics[width=0.9\textwidth]{img/image-20230614143547029.png}
\tcblower
\kaishu{\small }
\end{tcolorbox}

对于二次函数而言, 上凹下凹便对应着开口向上和开口向下.

函数的凹凸性发生变化的点我们称之为拐点 (point of inflection),
有时也称为驻点. 在拐点处, $f''(x)=0$ 或不存在 (例如趋向于无穷).

\end{tcolorbox}

\subsection{洛必达法则 (L'Hôpital's
rule)}

在【\ref{013}\nameref{013}】中, 在求三角函数的导数时, 我们遇到了极限是不定式/未定型的情况,
先前我们利用几何法绕开了这种情况, 现在我们来看怎么和这种情况刚正面.

\begin{tcolorbox}[size=fbox, breakable, enhanced jigsaw, title={罗尔中值定理 (Rolle's theorem)}]



\begin{tcolorbox}[size=fbox, breakable, enhanced jigsaw, sidebyside]
\includegraphics[width=0.9\textwidth]{img/image-20230615082313476.png}
\tcblower
\kaishu{\small 若 $f(x)$ 在区间 $[a,b]$ 连续, 且在区间 $(a,b)$ 可微或存在导数,
如果 $f(a)=f(b)$, 则至少存在一个点 $x=c$ 使得 $f'(c)=0$.\\很直观, 一定存在一个斜率正负值变化的地方, 处非是一个 trivial 的情况,
即函数为一个常数, 那么斜率处处为 $0$.}
\end{tcolorbox}

\begin{newquote}
证明思路:

极值定理 (参见【\ref{011}\nameref{011}】) 告诉我们区间 $[a,b]$ 存在极值, 前面我们还发现,
极值处的导数为 $0$.
\end{newquote}

\end{tcolorbox}

\begin{tcolorbox}[size=fbox, breakable, enhanced jigsaw, title={拉格朗日中值定理 (Lagrange mean value theorem)}]



\begin{tcolorbox}[size=fbox, breakable, enhanced jigsaw, sidebyside]
\includegraphics[width=0.9\textwidth]{img/image-20230615082602908.png}
\tcblower
\kaishu{\small 若 $f(x)$ 在区间 $[a,b]$ 连续, 且在区间 $(a,b)$ 可微或存在导数,
则至少存在一个点 $x=c$ 使得 $f'(c)=\frac{f(b)-f(a)}{b-a}$.}
\end{tcolorbox}

\begin{newquote}
证明思路:

这是罗尔中值定理''加强版'', 可以从罗尔中值定理出发去证明.
最后的结论可以理解为,
这个区间存在一个点使得【函数的斜率】和【函数在这个区间两个端点连线的斜率】一致.
因为斜率和连线的斜率一致, 也就是说, 【原函数】减去【两端点连线的函数】,
所得到的新的函数在这个点斜率是 $0$.

\begin{itemize}

\item
  综上, 构造这样一个函数 (最难的一步, 关于这一点的吐槽参见【\ref{011}\nameref{011}】,
  因为最后的), 令 $g(x):=f(x)-f(a)-\frac{f(b)-f(a)}{b-a}(x-a)$.
\item
  易见这样一来 $g(a)=g(b)=0$, 利用罗尔中值定理, 可得存在一点 $x=c$
  使得 $g'(c)=0$.
\item
  $g'(x)=f'(x)-\frac{f(b)-f(a)}{b-a}$, 这里要注意 $f(a)$ 和 $f(b)$
  已经将 $x=a$ 和 $x=b$ 分别代入了原函数, 是一个具体数值了;
\item
  于是 $g'(c)=f'(c)-\frac{f(b)-f(a)}{b-a}=0$, 证毕.
\end{itemize}
\end{newquote}

\end{tcolorbox}

\begin{tcolorbox}[size=fbox, breakable, enhanced jigsaw, title={柯西中值定理 (Cauchy's mean value theorem)}]



\begin{tcolorbox}[size=fbox, breakable, enhanced jigsaw, sidebyside]
\includegraphics[width=0.9\textwidth]{img/image-20230615082658188.png}
\tcblower
\kaishu{\small 若 $f(x)$ 和 $g(x)$ 在区间 $[a,b]$ 连续, 且在区间 $(a,b)$
可微或存在导数, 且在区间 $(a,b)$ 内 $g(x)\neq0$, 则至少存在一个点
$x=c$ 使得 $\frac{f'(c)}{g'(c)}=\frac{f(b)-g(a)}{g(b)-g(a)}$.}
\end{tcolorbox}

\begin{newquote}
证明思路:

构造一个函数 $h(x):=f(x)-f(a)-\frac{f(b)-f(a)}{g(b)-g(a)}(g(x)-g(a))$
再利用拉格朗日中值定理易得上述结论.
\end{newquote}

为了理解这一条, 可以引入参数化曲线 (parametric curve) 这个概念.
一条曲线, 除了用类似 $y=f(x)$ 这样的形式来表示,
若是这条曲线是一个物体的运动曲线, 还可以引入时间 $t$, 这个参数
(parameter), 然后用时间来表示 $\{x,y\}$ 的变化, 即 \[
\begin{cases}
x=x(t),\\
y=y(t);
\end{cases}
\] 于是柯西中值定理可以视作, $f(x)$ 和 $g(x)$ 分别代表两个坐标值,
$x$ 是一个参数, 这样一来 $f'(x)/g'(x)$
就可以理解为这条参数化曲线的切线的''斜率''了.

\end{tcolorbox}

回到我们的主题, 当出现类似 $f(a)=0$, $g(a)=0$, 且希望求得极限
$\lim_{x\rightarrow a}\frac{f(x)}{g(x)}$ 时, 我们可以从 $a$
的某一侧逼近 $a$, 利用柯西中值定理可以知道在此区间一定存在一点 $x=c$
使得 \[
\frac{f'(c)}{g'(c)}=\frac{f(x)-f(a)}{g(x)-g(a)},
\] 已知 $f(a)=0$, $g(a)=0$, 于是 \[
\frac{f'(c)}{g'(c)}=\frac{f(x)}{g(x)},
\] 再令 $x\rightarrow a$, 因为 $c$ 一定在 $x$ 与 $a$ 之间,
于是类似三明治定理的情况, 在这个极限下, $c$ 也会趋向于 $a$;
于是我们便得到了我们希望得到的结果: \[
\boxed{\lim_{x\rightarrow a}\frac{f(x)}{g(x)}=\lim_{x\rightarrow a}\frac{f'(x)}{g'(x)}}.
\] 也就是说, 求极限时, 出现 $\frac{0}{0}$ 的不定式,
我们可以非常简单粗暴地来看它导数的极限.

\begin{newquote}
例: $\lim_{x\rightarrow 0}\frac{\sin(x)}{x}$ \[
\lim_{x\rightarrow 0}\frac{\sin(x)}{x}\underset{\text{L.H.}}{=}\lim_{x\rightarrow 0}\frac{\cos(x)}{1}=1.
\]
\end{newquote}
\section{泰勒级数}\label{016}

\begin{tcolorbox}[size=fbox, breakable, enhanced jigsaw, title={求和 (summation) - 复习}]

求和的标记 $\sum$ 在\ref{007}\nameref{007}其实已经介绍过了, 这里再复习一遍.
求和符号右边是代求和的每一项的表达式,
求和符号下面标注了右边表达式的第一项需要代入的值,
求和符号上面则标注了最后一项需要代入的值.

\begin{newquote}
举一个例子, 小学的高斯求和法,
\[
1+2+...+99+100=\sum_{n=1}^{100}n=5050.    
\]
高斯的思路无非是, 将这个求和拆为 $\{1,100\}$, $\{2,99\}$, \ldots,
$\{49,52\}$$, \{50,51\}$ 的组合, 每一组的和都是 $101$, 一共有
$50$ 个这样的组合, 于是得到``一加到一百''为 $5050$ \footnote{小学高我一年级的一学长刚学完这一课向我耍宝的时候,
  其实我也想到了类似的方法, 把求和凑成 $\{1,99\}$, $\{2,98\}$,
  \ldots, $\{48,52\}$$, \{49,51\}$ 的一对对, 最后还剩下 $50$ 和
  $100$, 也能得到答案, 不过这个方法并不普适.}.
推广一下便可得到对一组公差为 $1$ 的等差数列求和,
和为【首项】加【末项】乘【项数】除以二. 于是有 \[
\sum_{n=1}^Nn=\frac{N(N+1)}{2}.
\] 另有两个可能会有一些用的结论: \[
\begin{aligned}
  \sum_{n=1}^Nn^2&=\frac{N(N+1)(2N+1)}{6},\\
  \sum_{n=1}^Nn^3&=\left(\frac{N(N+1)}{2}\right)^2.
\end{aligned}
\] 证明留作练习, 提示是可以利用数学归纳法 (参见\ref{007}\nameref{007}).
\end{newquote}

因为前面提到过, 这并不是一本严谨的数学书, 数列和级数我们就跳过了;
但是接下来要讲泰勒级数, 还是来一点开胃菜 (appetizer): \textbf{几何级数}
(geometric series).

几何级数又叫等比级数, 即它的每一项和之前一项的倍数是恒定的. 于是第一项为
$a$, 相邻两项倍数为 $r$ 的几何级数可以记作 \[
\sum_nar^{n-1}.
\] 不难证明前 $n$ 项之和应为 \[
S_n=a\frac{r^n-1}{r-1}.
\]

\begin{newquote}
上结论证明如下:

前 $n$ 项之和展开写是 $S_n=a+ar+ar^2+...+ar^{n-2}+ar^{n-1}$,
两边同乘 $r$ 得到 $rS_n=ar+ar^2+ar^3+...+ar^{n-1}+ar^n$. 将 $rS_n$
与 $S_n$ 相减, 消去相同项便有 $(r-1)S_n=a(r^n-1)$,
整理便可得到结论.\footnote{在这个推导中, 可以看到我们将代求的 $S_n$
  带着计算, 要习惯这种和逆向思维相对的``正向思维'',
  \ref{014}\nameref{014}逆函数的导数的推导也有这么一丝味道,
  【剧透警告】之后积分中非常重要的分步积分法也会有类似的思路.}
\end{newquote}

这样一来, 当这个级数无限长, 即 $n$ 趋向于无穷, 且公倍数 $r$
的绝对值小于一, 这个级数 (求和) 收敛到 \[
\sum_nar^{n-1}=\frac{a}{1-r}.
\]

\end{tcolorbox}

\begin{tcolorbox}[size=fbox, breakable, enhanced jigsaw, title={泰勒级数 (Taylor Series)}]

有的时候, 我们研究的函数 $f(x)$ 可能并不是一个非常 nice 的函数,
它的很多特性并不那么``好''; 更直观一些,
很多时候我们研究的函数可能是\textbf{超越函数} (Transcendental
Functions), 即变量之间的关系不能用有限次加, 减, 乘, 除,
次方运算表示的函数, 最简单的例子, 比如三角函数, 不利用计算器等工具,
我们很难去直接计算. 正好, 很多现实场景下, 我们也并不需要任意精确的结果,
我们可能只需要几位有效数字, 这个时候,
我们便可以利用那些``好''的函数去\textbf{拟合} (fit)
这些不那么``好''的函数.

最常见的比较``好''的函数是什么? 多项式 (Polynomial)! 它取值相对简单,
对它求导更是非常轻松. 所以很多时候, 我们会用多项式去拟合.

现在假设我们手头上有一个函数 $f(x)$, 我们希望能用多项式去拟合它,
那么它大约可以被写成以下的形式:

\[
f(x)=\sum_{n=0}^\infty a_n(x-x_0)^n,
\]

这里 $a_n$ 是每一项的系数, $x_0$ 是某个固定的值, 可以看到 $x$
的最高系数逐项增加. 对上式不断求导可以得到

\begin{gather*}
f'(x)=a_1+2!\cdot a_2(x-x_0)+\mathcal{O}(x-x_0)^2\\
f''(x)=2!\cdot a_2+3!\cdot a_3(x-x_0)+\mathcal{O}(x-x_0)^2\\
...
\end{gather*}

仅在这里, 我们规定 $\mathcal{O}$ 的含义是, 例如: $\mathcal{O}(x)$
表示包含 $x$ 以及 $x$ 更高次数的项 (即, $ax, bx^2, cx^3, ...$ ).
通常而言, $n$ 次导可以得到:

\[
f^{(n)}(x)=n!a_n+\mathcal{O}(x-x_0)
\]

令 $x:=x_0$ 便可得到每一项系数应该是

\[
a_n=\frac{f^{(n)}(x_0)}{n!},
\]

于是用来拟合一个函数的多项式便可以是

\[
\boxed{f(x)=\sum_{n=0}^\infty\frac{f^{(n)}(x_0)}{n!}(x-x_0)^n.}
\]

将一个函数用上式展开, 便得到了这个函数的\textbf{泰勒级数}\footnote{剧透:
  把这个结论推广到复变函数, 泰勒级数就变成了\textbf{洛朗级数 (Laurent
  series)}.}. \ref{007}\nameref{007}中提到的二项式展开,
其实可以看做是泰勒级数的一个特例.

\begin{tcolorbox}[size=fbox, breakable, enhanced jigsaw, title={收敛性}]

前面我们似乎是假定随着项数增加泰勒级数可以趋向原函数了, 但是一定如此吗?
这边有一个定理.

\begin{tcolorbox}[size=fbox, breakable, enhanced jigsaw, title={泰勒中值定理 (Taylor's Theorem)}]

有一个 $n$ 次可导的函数 $f(x)$, 将其的导数依次记作
$f'(x),f''(x),...,f^{(n)}(x)$, 考虑一个闭区间 $[a,b]$,
那么在这个区间内一定存在一个 $c$ 使得 \[
f(b)=f(a)+f'(a)(b-a)+\frac{f''(a)}{2!}(b-a)^2+...+\frac{f^{(n)}(c)}{n!}(b-a)^n.
\] 注意等式右边前面的函数和导数都是在 $x=a$ 处取值, 而最后一项是在
$x=c$ 处. 不难看出这其实是中值定理的推广.

如此一来, 换言之, $f(x)$ 与
$f(x_0)+f'(x_0)(x-x_0)+...+\frac{f^{(n-1)}(x_0)}{(n-1)!}(x-x_0)^{n-1}$
的偏差等于 $\frac{f^{(n)}(\xi)}{n!}(x-x_0)^n$, 对于某个在 $x$ 和
$x_0$ 之间的 $\xi$.

\end{tcolorbox}

实际使用的时候, 我们常常这么说: 【在 (某个特定的) $x_0$ 附近展开
$f(x)$ \ldots】 这样一来, 就要注意, 展开后,
取\textbf{有限项}级数来\textbf{近似}原函数时, 只有取 $x$
\textbf{足够}接近 $x_0$ 时才是有效的. 例如: 利用在 $0$ 附近展开的
$\sin(x)$ 函数的泰勒级数的前若干项, 来近似当 $x$ 很大时 (例如大于
$2\pi$) 时, 结果会是非常荒谬的\footnote{一些题外话: 在计算科学,
  数据分析等专业, 当我们有许多数据点,
  经常会用多项式去试图拟合这一些数据点, 更高次的多项式,
  通常意味着对已有数据更好的拟合 (比如 $n$ 次的多项式可以完美拟合
  $(n+1)$ 个数据点), 但是这往往并不是我们想要的,
  一方面在所有数据点两端, 高次多项式的行为会非常不合理, 另一方面,
  事实上一个模型也不应该具有如此高的自由度 (多项式每高一次,
  对应着模型多一个自由度). 这就引出了两个概念: 插值 (interpolation)
  vs.~拟合 (fitting), 插值要求函数通过每一个给定的数据点,
  而拟合是在现有模型的基础上调整参数, 保证函数和数据点是最小二乘的
  (least squared).}; 这里可以利用 $\sin(x)$ 函数的周期性, 将 $x$
减去若干个 $2\pi$ 再计算.

\end{tcolorbox}

\end{tcolorbox}

\begin{tcolorbox}[size=fbox, breakable, enhanced jigsaw, title={练习}]

求以下函数的泰勒级数: (a) $(1\pm x)^{-1}$, (b) $\mathrm{e}^x$, (c)
$\sin(x)$, (d) $\cos(x)$, (e) $\ln(1+x)$, (f) $\arctan(x)$.

展开后不难发现, 对于 $x\approx 0$ (或者说 $x\ll1$) 有
$\sin(x)\approx x$, $\cos(x)\approx 1$ (当 $x$ 足够小, 含 $x$
及其高次的项便足够小, 回收了\ref{009}\nameref{009}中提过的小角度近似).
另外还有一个常用的近似是 $\ln(x+1)\approx x$ 对于 $x\ll 1$. 在还有,
展开的形式可以进一步地佐证欧拉公式 (回顾\ref{009}\nameref{009}).

\end{tcolorbox}
\hypertarget{ux79efux5206-integration}{%
\subsubsection{积分 (integration)}\label{ux79efux5206-integration}}

来到微积分的积分. 首先是动机 (motivation), 为什么需要积分?
我们需要导数或者说微分是通常是因为我们需要知道一些物理量的\textbf{变化率}
(rate of change), 当然之前提到的求切线也可以作为一个非常 trivial
的一个情况. 积分呢, 图像上来说, 积分往往反映的是函数图像与
\(x\)-轴围起来的面积; 这里要注意, 不要被这种可视化个梏桎住了思想,
就像切线只是导数的一种可视化一样,
函数图像下方的面积也仅仅只是积分众多的可视化的一种,
不能局限于这一层理解.

若有一个函数 \(f(x)\), 我们希望求得它在 \([a,b]\) 这个区间内, 函数图像与
\(x\)-轴 (左右再加两条竖线) 围成的面积; 如下图所示,

\begin{itemize}
\tightlist
\item
  首先我们可以尝试用一个个矩形去近似代求的面积, 在 \([a,b]\)
  之间选取一系列的点, 使得 \(a=x_0<x_1<x_2<...<x_{n-1}<x_{n}=b\),
  于是便有了一系列形如 \([x_i,x_{i+1}]\) 的子区间 (subinterval);
  这称之为一个分割 (partition);
\item
  直觉上, 如果将子区间进一步地分割, 或者专业一点的说: 如果有更精细
  (fine) 的分割, 矩形会更加贴合函数曲线, 即近似地误差会变得更小,
  矩形的面积之和会更接近函数围成的实际面积;
\item
  当每一个子区间的''宽度''趋向于 \(0\) 时,
  直觉上矩形的面积之和便会趋向于函数围成的实际面积.
\end{itemize}

\begin{figure}
\centering
\includegraphics{image-20230906163643414.png}
\caption{image-20230906163643414}
\end{figure}

正式一点的, 我们把这些矩形面积和称作黎曼和 (Riemann sum): \[
\sum_{i=0}^{i=n-1}f(\xi_i)(x_{i-1}-x_i),
\] 其中 \(x_{i-1}\le\xi_i\le x_i\). 当分割不断变得更加精细,
黎曼和最终趋向的值, 我们称其为函数 \(f(x)\) 在区间 \([a,b]\)
的\textbf{定积分} (definite integral), 暂时将这个值记作 \(I\),
对于任意的 \(\epsilon>0\) 都有 \(\delta>0\) 使得: 对于任意的分割,
子区间的数量 \(n<\delta\), 以及任意选择的每一个 \(\xi_i\) , 我们有: \[
\left|\sum_{i=0}^{i=n-1}f(\xi_i)(x_{i-1}-x_i)-I\right|<\epsilon.
\] 我们可以将定积分定义为, \[
\boxed{I=\lim_{n\rightarrow\infty}\sum_{i=0}^nf(\xi_i)(x_{i+1}-x_i).}
\] 令 \(\Delta x:=x_{i+1}-x_i\), 这样一来,
上面取子区间的数量趋向无穷的极限一定程度上等价于每个子区间的''宽度''趋向于无穷小,
于是在这个极限下 \(\Delta x\) 变成了一个无穷小量, 即
\(\Delta x\rightarrow \mathrm{d}x\)\footnote{这个操作在物理中经常出现,
  要很熟悉这种找到有限的变化之间的关系, 然后令它变为无穷小量这样的操作.}.
对此, 莱布尼兹 (Leibniz) 发明了一个定积分的标记,
他将一个''S''拉长表示这种特殊的''求和'', 于是上面的定积分我们便通常写作
\[
\boxed{\int_a^bf(x)\mathrm{d}x.}
\] 有一点要注意的是, 这个积分之和函数本身 \(f\), 积分的区间 \([a,b]\)
相关, 和积分对象 \(x\) 无关, 所以将定积分的积分对象由 \(x\)
换为其他任意字母, 得到的结果是不变的. 于是在这种情况下, 像 \(x\)
这样出现在公式中, 但又不实际太多地参与到运算中的变量,
我们称之为虚拟变量/哑变量 (dummy variable,
个人认为傀儡变量更信雅达一些).

\begin{quote}
很多初学者最早在接触积分的时候, 会忘记最后的 \(\mathrm{d}x\),
本人最初的理解是, 这个 \(\mathrm{d}x\) 是用来表示积分对象是 \(x\);
但从函数图像面积这样的几何角度出发, 这个 \(\mathrm{d}x\)
事实上是每一个子区间面积的''底'', 因此这个 \(\mathrm{d}x\)
是不能被省略的.
\end{quote}

\textbf{更加严格的版本 - 选读}

对于某一个特定的分割, 在子区间 \([x_{i-1},x_i]\) 中, 令 \[
\begin{gather*}
M_i:=\sup \left(f(x)\right),\\
m_i:=\inf \left(f(x)\right).
\end{gather*}
\]

\begin{quote}
这里 \(\sup\) 和 \(\inf\) 全写是 supremum 和 infimum,
中文分别是上确界和下确界, 和最大值 \(\max\) 和最小值 \(\min\) 很像,
区别在于, 例如在某个区间内, 若 \(f(x)\le \eta\), 我们可以说
\(\eta=\max \left(f(x)\right)=\sup \left(f(x)\right)\), 但若
\(f(x)<\eta\), 我们只能说 \(\eta=\sup \left(f(x)\right)\) 因为
\(\max \left(f(x)\right)\) 取不到 \(\eta\).
\end{quote}

再令 \[
\begin{gather*}
U:=\sum M_i\Delta_i,\\
L:=\sum m_i\Delta_i,
\end{gather*}
\] 相当于某一个特定的分割下, 黎曼和的上限和下限, 最后令 \[
\overline{\int_a^b}f(x)\mathrm{d}x:=\inf U,\\
\underline{\int_a^b}f(x)\mathrm{d}x:=\sup L,
\] 即所有分割下, \(U\) 的下限和 \(L\) 的上限 (上限的下限和下限的上限,
lol), 分别称他们为黎曼上积分和黎曼下积分.
若一个函数在一个区间内黎曼上积分和下积分相等,
我们就说这个函数在这个区间内是黎曼可积的 (Riemann integrable),
积分的值记作 \(\int_a^bf(x)\mathrm{d}x\).

\section{微积分基本定理-上}\label{018}

\begin{tcolorbox}[size=fbox, breakable, enhanced jigsaw, title={微积分基本定理 (fundamental theorem of calculus)}]

前面只讲了积分的动机和定义, 但是并没有涉及具体的运算,
那么这里先摆出结论: 积分和微分可以视作互逆的运算.
这一点在很多高中的教材似乎是默认的, 但是事实上真的如此吗?
为了证明这一点, 首先, 令 \[
F(x):=\int_a^xf(\xi)\mathrm{d}\xi,
\]

\begin{newquote}
怎么突然出现了 \(\xi\) ? 其实原则上写 \(f(x)\mathrm{d}x\)
也没有太大问题, 但是积分的上限也是 \(x\), 为了加以区分,
所以将积分对象的变量改为其他字母, 其含义是没有发生任何变化的
(参见\ref{017}\nameref{017}对于 dummy variable 的讨论). 只是积分对象和积分上限都用
\(x\) 有滥用标记 (abuse of notation) 之嫌, 不是非常专业.
\end{newquote}

考虑 \(F(x)\) 的斜率, 首先是近似的形式 \[
\begin{aligned}
&\frac{F(x+h)-F(x)}{h}\\
=&\frac{1}{h}\left(\int_a^{x+h}f(\xi)\mathrm{d}\xi-\int_a^xf(\xi)\mathrm{d}\xi\right)\\
=&\frac{1}{h}\int_x^{x+h}f(\xi)\mathrm{d}\xi.
\end{aligned}
\] 上式最后两行可以如下图所示地用函数下的面积来理解,
不难看出第一个积分得到的是深色面积,
第二个积分得到的是深色和浅色的面积之和, 于是他们的差便是浅色的面积,
浅色面积对应的区间是 \([x,x+h]\), 于是相当于是上下限分别是 \(x\) 和
\((x+h)\) 的积分.

\begin{tcolorbox}[size=fbox, breakable, enhanced jigsaw]
\includegraphics[width=0.3\textwidth]{img/image-20230912145204089.png}

\end{tcolorbox}

\begin{tcolorbox}[size=fbox, breakable, enhanced jigsaw, title={插曲: 定积分的中值定理}]\footnote{其实应该现有了微积分基本定理再推积分的中值定理会更方便,
  中值定理有 \(f'(c)=(f(b)-f(a))/(b-a)\) 对于某个区间 \([a,b]\) 内的
  \(c\). 令 \(F(x):=\int_a^xf(\xi)\mathrm{d}\xi\), 将 \(F(x)\)
  直接套入中值定理, 再利用积分和求导互为逆运算, 便有
  \(f(c)=F'(c)=(F(b)-F(a))/(b-a)\).}

若 \(f(x)\) 在区间 \([a,b]\) 连续, 则存在某点 \(c\) 使得, \[
\boxed{f(c)=\frac{1}{b-a}\int^b_af(x)\mathrm{d}x.}
\] 像 \(\frac{1}{b-a}\int_a^bf(x)\mathrm{d}x\) 这样的形式,
事实上是取了函数 \(f(x)\) 在区间 \([a,b]\) 的``平均值'', 即:
【把这个形式得到的值想象成高, 把 \([a,b]\) 想象成底,
相乘得到的矩形面积】和【积分对应的函数图像下的面积】应该是一致的;
那么自然, 这个``平均值''是小于这个区间内 \(f(x)\) 的最大值,
而大于这个区间内 \(f(x)\) 的最小值的; 同时, 因为 \(f(x)\) 是连续的,
那么必然在这个区间内存在至少一点 \(c\) 使得 \(f(c)\) 等于这个``平均值''.

\end{tcolorbox}

有了定积分的中值定理, 再对比一下前面 \(F(x)\) 的斜率的近似形式, 不难发现
\([x,x+h]\) 这个区间内, 也应该存在某点 \(c\), 使得
\(f(c)=\frac{1}{h}\int_x^{x+h}f(\xi)\mathrm{d}\xi\). 这里要注意, 这个
\(c\) 并不是固定的, 随着 \(h\) 的变化, \(c\) 应该也是随之变化的.

接下来的操作很不严谨的 (hand-wavy, 这边有一个很难直译的词,
当说一些很模棱两可又暧昧的说法时, 我们可能会习惯性地波动我们的手),
取极限 \(h\rightarrow0\) 是, 于是 \([x,x+h]\) 这个区间也越来越窄,
到最后迫使 \(f(c)\rightarrow f(x)\). 于是便有 \[
F'(x)=\lim_{h\rightarrow 0}\frac{1}{h}\int_x^{x+h}f(\xi)\mathrm{d}\xi=f(x).
\] 这便是\textbf{微积分基本定理} (fundamental theorem of calculus),
重新表述如下: \[
\boxed{\frac{\mathrm{d}}{\mathrm{d}x}\int_a^xf(\xi)\mathrm{\xi}=f(x).}
\] 可以看到, 一个函数经过积分又求导后变回了原函数,
可见微分和求导一定程度上确实应该是互为逆运算的. 这样,
当我们实际计算积分时可以利用\textbf{反导数 }(antiderivative) 来完成计算.

\begin{tcolorbox}[size=fbox, breakable, enhanced jigsaw, title={反导数}]

在某个区间内, 若对于任意 \(x\) 都有
\(F'(x)=\frac{\mathrm{d}}{\mathrm{d}x}F(x)=f(x)\), 则在此区间内函数
\(F(x)\) 是 \(f(x)\) 的反导数.

于是, 对于一个在区间 \([a,b]\) 内连续的函数 \(f(x)\), 它的定积分,
用微积分基本定理可以表述为: \[
\boxed{\int_a^bf(x)\mathrm{d}x=F(b)-F(a),}
\] 这里 \(F(x)\) 是 \(f(x)\) 在区间 \([a,b]\) 的反导数.

\end{tcolorbox}

\begin{newquote}
例: \(\int_a^b x\mathrm{d}x\)

我们知道, 求导时, 多项式的会先乘上次数, 然后次数会减一,
那么逆导数便应该是次数加一, 然后除以新的次数 (顺过来的运算后执行的,
逆运算先执行),

\begin{tcolorbox}[size=fbox, breakable, enhanced jigsaw]
\includegraphics[width=0.6\textwidth]{img/image-20230912161734076.png}

\end{tcolorbox}

类似的, 三角函数, 指数函数, 对数函数, 符合函数 (复习链式法则,
参见\ref{013}\nameref{013}) 等等的逆导数也可以用这样的思路去求.

如此这般, 首先, 若 \(f(x)=x\), 应该对应逆导数 \(F(x)=\frac{1}{2}x^2\).
然后 \[
\begin{aligned}
  \int_a^bf(x)\mathrm{d}x&=F(b)-F(a)\\
  &=\frac{1}{2}b^2-\frac{1}{2}a^2.
\end{aligned}
\] 习惯上我们更通常这么写 \[
\begin{aligned}
\int_a^bx\mathrm{d}x=&\left.\frac{1}{2}x^2\right|_{x=a}^b\\
=&\frac{1}{2}b^2-\frac{1}{2}a^2.
\end{aligned}
\] 右边一竖的意思是在 \(x=a\) 和 \(x=b\) 处分别求值 (evaluate)
再求差的意思.
\end{newquote}

\begin{tcolorbox}[size=fbox, breakable, enhanced jigsaw, title={不定积分}]

前面积分时, 都强调了积分的上下限,
但是有的时候我们可能需要一个通常的表达式, 而不需要求值,
这种积分我们称作不定积分. 一个函数的不定积分等于它的逆求导加上一个常数:
\[
\int f(x)\mathrm{d}x=F(x)+\text{const.}
\] 等价的说法是, 一个函数其实并不只有一个逆导数,
但是这些逆导数只差了一个常数 (和积分变量无关). 这是因为求导时,
常数项直接``消失''了. 这就带来了规范自由 (gauge freedom)\footnote{超纲警告!
  规范自由本质上来说就是一个标量场加上一个常数并不会影响它的梯度.}.

\end{tcolorbox}
\end{tcolorbox}
\section{微积分基本定理-下}\label{019}

\begin{tcolorbox}[size=fbox, breakable, enhanced jigsaw, title={更加严格的版本 - 选读}]

考虑一个在区间 \([a,b]\) 可积的 \(f(x)\), 和先前一样的, 令 \[
F(x):=\int_a^xf(\xi)\mathrm{d}\xi,
\] 这里 \(a\le x \le b\). 考虑 \(f(x)\) 在此区间的上下界, 若有
\(M\ge |f(x)|\) 在此区间内, 那么对于 \(a\le p \le q \le b\) 显然有 \[
|F(q)-F(p)|=\left|\int_p^qf(\xi)\mathrm{d}\xi\right|\le M\cdot(q-p).
\] (如果觉得不那么显然的话, 可以参考【\ref{018}\nameref{018}】``平均值''这一套论述).
于是对于 \(\epsilon>0\), 只要\(|q-p|<\epsilon/M\), 便有
\(|F(q)-F(p)|<\epsilon\), 这样首先证明了 \(F(x)\) 是连续的.

继而, 选定某个 \(x\), 且在此处 \(f(x)\) 是连续的. 给定 \(\epsilon>0\),
选择一个 \(\delta>0\), 使得若有 \(|\xi-x|<\delta\) 且 \(a\le\xi\le b\),
便有 \(|f(\xi)-f(x)|<\epsilon\).

如此一来, 如果 \(x-\delta<p\le x\le q<x+\delta\), 且 \(a\le p<q\le b\),
便有 \[
\begin{aligned}
\left|\frac{F(q)-F(p)}{q-p}-f(x)\right|&=\left|\frac{1}{q-p}\left(\int_a^qf(\xi)\mathrm{d}\xi-\int_a^pf(\xi)\mathrm{d}\xi\right)-f(x)\right|\\
&=\left|\frac{1}{q-p}\int_p^qf(\xi)\mathrm{d}\xi-f(x)\right|\\
&=\left|\frac{1}{q-p}\int_p^q[f(\xi)-f(x)]\mathrm{d}\xi\right|\\
&<\left|\frac{1}{q-p}\int_p^q\epsilon\mathrm{d}\xi\right|=\epsilon.
\end{aligned}
\] 也就是说, 在区间 \([a,b]\) 有 \(F'(x)=f(x)\). 以上我们用
\(\epsilon-\delta\) 语言更严格地证明了某种意义上, 积分和微分互为逆运算.

类似的, 尝试证明对于任意 \(\epsilon>0\) 都有
\(\left|F(b)-F(a)-\int_a^bf(x)\mathrm{d}x\right|<\epsilon\),
也可以严格证明定积分的微积分基本定理, 这一部分留作练习.

\end{tcolorbox}

\begin{tcolorbox}[size=fbox, breakable, enhanced jigsaw, title={应用? - 和物理的联系}]

经典的速度的定义是位移的导数,
\(v\equiv \frac{\mathrm{d}}{\mathrm{d}t}s(t)\).

如下图所示: 匀速的情况, 位移很好理解, 无非是速度乘上时间.
速度是阶梯函数的话, 位移依旧很好求, 分段计算即可.
若速度是连续变化的函数, 那么在我们有积分这个工具之前, 就非常困难了.

\begin{tcolorbox}[size=fbox, breakable, enhanced jigsaw]
\includegraphics[width=0.9\textwidth]{img/image-20230912164801681.png}
\end{tcolorbox}

我们可以尝试将整个过程分成很多段, 每一段估计一个平均速度
(这一段时间内最大速度和最小速度之间的值), 然后把这一段近似成匀速计算,
或者将这一小段近似成匀变速的情况, 即 \(s=ut+\frac{1}{2}at^2\)
(如果没见过这个公式可以忽略它\ldots{} 不影响后面的理解). 不管怎么样,
对于匀速和速度是阶梯函数, 我们其实还是计算了\(v-t\)函数图像下的面积,
对于速度是连续函数的情况, 我们也还是在估算函数下方的面积, 于是很自然的,
若我们有积分这个工具, 速度又可以表示成一个关于时间的函数, 应该有 \[
s=\int v\mathrm{d}t.
\] 当然, 不定积分需要加上积分常数 (参见【\ref{018}\nameref{018}】文末),
这通常可以靠初始条件 (比如 \(t=0\) 时的位移) 来确定;
定积分反映的则是两个时间点之间发生的位移. 类似的,
既然加速度的定义是速度的变化率 (导数),
\(a\equiv \frac{\mathrm{d}}{\mathrm{d}t}v(t)\), 那么也应该有 \[
v=\int a\mathrm{d}t.
\] 当然, 也要注意不定积分需要用初始条件确定积分常数,
而不定积分反映的是两个时间点之间速度的变化量.

再还有很多情况:

\begin{itemize}
\item
  比如力时关于位置的一个函数, 那么做功就不再是简单的 \(W=Fx\), 而是
  \(W=\int F(x)\mathrm{d}x\) 了;
\item
  比如密度是关于位置的函数, 那么质量就不再是简单的 \(m=\rho V\), 而是
  \(m=\int \rho(\vec{r})\mathrm{d}V\), 这个体积微元 (volume
  differential/infinitesimal) 实际运算的时候需要计算三重积分, 即
  \(\int\cdot\mathrm{d}V=\iiint\cdot\mathrm{d}x\mathrm{d}y\mathrm{d}z=\iiint\cdot r^2\sin\theta\mathrm{d}r\mathrm{d}\theta\mathrm{d}\phi\),
  这里给的是三维直角坐标系和球状坐标系的例子,
  视情况和方便程度而定使用哪种坐标, 这里暂时不具体展开.
\end{itemize}
\end{tcolorbox}
\section{积分技巧}\label{020}

\subsection{换元 (substitution)}

事实上积分时的换元就是求导时的链式规则 (参见\ref{013}\nameref{013}) 反过来的过程.
具体操作是这样的, 例如若代求 \(\int f(x)\mathrm{d}x\), 有时为了计算方便,
我们 (i) 先定义另一个函数 \(u(x)\), 并利用 \(f(x)\) 和 \(u(x)\) 的关系,
将代积分的部分的 \(x\) 消去, 将其变为一个关于 \(u\) 的函数; (ii)
因为现在积分的对象不显含 \(x\) 了, 因此我们应该将最后的 \(\mathrm{d}x\)
也尝试变为 \(\mathrm{d}u\) (乘上一些项); 实际操作中 (i) 和 (ii)
步一般是同时进行的, 保证最后的形式是不显含 \(x\) 而只剩 \(u\); (iii)
最后, 若积分是一个定积分, 还要利用 \(x\) 和 \(u\) 的关系,
将积分上下限替换.

\begin{newquote}
例: 求 \(\int\sin^2(x)\cos(x)\mathrm{d}x\)

令: \(u=\sin(x)\)

则
\(\frac{\mathrm{d}u}{\mathrm{d}x}=\cos(x)\Rightarrow\mathrm{d}u=\cos(x)\mathrm{d}x\).

于是积分转化为 \(\int u^2\mathrm{d}u\), 后面的步骤便非常容易了.
\end{newquote}

\subsection{分部积分法 (integration by parts)}

分部积分法是由求导的乘法法则 (参见\ref{012}\nameref{012}) 而来, 求导的乘法法则有 \[
(fg)'(x)=f'(x)g(x)+f(x)g'(x),
\] 对其积分可以得到 \[
\int(fg)'(x)\mathrm{d}x=\int f'(x)g(x)\mathrm{d}x+\int f(x)g'(x)\mathrm{d}x,
\] 等式左边也可以写成
\(\int\frac{\mathrm{d}}{\mathrm{d}x}f(x)g(x)\mathrm{d}x\),
根据微积分基本定理, 积分和求导可以视作互为逆运算, 于是等式左边事实上便是
\(f(x)g(x)\), 挪项可得 \[
\int f(x)g'(x)\mathrm{d}x=f(x)g(x)-\int f'(x)g(x)\mathrm{d}x.
\] 更通常的, 很多教材会使用 \(u(x)\) 和 \(v(x)\), 省略 \((x)\),
分部积分法可以记作 \[
\boxed{\int u\ \mathrm{d}v=uv-\int \mathrm{d}u\ v}.
\]

\begin{newquote}
注意: 上式中的 \(\mathrm{d}\cdot\) 并不表示积分对象是 \(v\) 和 \(u\),
例如 \(\int u\mathrm{d}v\) 要表示的事实上还是
\(\int u(x)v'(x)\mathrm{d}x\); 另外 \(\int \mathrm{d}u\ v\) 中的 \(v\)
也是需要被积分的, 这么记事实上非常不规范.

物理和工程上经常很类似的, 会有 \(\int\mathrm{d}x f(x)\) 这样的写法,
没有把代积分的部分夹在积分符号和 \(\mathrm{d}x\) 之间, 而把
\(\mathrm{d}x\) 前置, 某种程度上是先强调了一下积分的对象是哪一个变量.
\end{newquote}

实际计算中,
分部积分法经常出现在对于【一个函数和三角函数或是指数函数的乘积】的积分,
下面是一个例子.

\begin{newquote}
例: \(\int x\cos x\ \mathrm{d}x\)

这个积分是用先前的知识是无法直接进行的, 遂用分部积分法; 首先要规定 \(u\)
和 \(\mathrm{d}v\), 然后求 \(\mathrm{d}u\) 和 \(v\);
用部分积分法的时候要尽可能的让 \(\int \mathrm{d}u\ v\) 这一项方便计算,
因此这一题中, 我们选取 \(u=x\), 这样一来
\(\mathrm{d}u=1\cdot\mathrm{d}x\) 就很方便接下来的计算;
所以便有\footnote{下面这个盒子是我高中的数学老师 Paul Elkin 传授的,
  他把它称作 ``working block'' (工作区),
  当然习惯了之后可以完全省略这个盒子里的内容, 但是初上手的时候,
  这样一个盒子可以很好地把关键步骤和''计算草稿''分割开来,
  使得书写面板很整洁, 也方便检查.}: \[
\boxed{\begin{aligned}
&u=x,&&\mathrm{d}v=\cos x\ \mathrm{d}x;\\
&\mathrm{d}u=\mathrm{d}x,&&v=\sin x.
\end{aligned}}
\] 于是 \[
\int x\cos x\ \mathrm{d}x=x\sin x-\int\sin x\ \mathrm{d}x.
\] 后续的计算因为不包含两个函数乘积的积分就很容易了.

类似的, 一个多项式乘以三角函数或是指数函数都可以用上述方法;
有时分部积分后得到新的积分项还是无法直接进行,
这时便需要继续对这一项使用分部积分, 使得多项式不断降次.

还有一些情况也可以使用分部积分, 例如 \(\int\ln x\ \mathrm{d}x\),
乍一看似乎看不出这个积分的结果, 一个提示是可以把 \(\ln x\) 视作
\(1\cdot\ln x\), 后续的计算留作练习.
\end{newquote}

一些超纲: 事实上, 部分积分法的思想不止局限于对函数的积分, 在变分法
(calculus of variation) 中, 在推导欧拉-拉格朗日方程 (Euler-Lagrange
equation) 时, 对泛函 (functional)\footnote{函数从映射的角度,
  是将数映射到数, 这里的''数''可能是实数, 也可能是虚数, 还可以是向量
  (什么是向量? 后面再细讲); 而泛函, 可以理解为函数的函数,
  泛函可以将函数映射到数, 它的定义域是函数构成的向量空间
  (什么是向量空间? 有机会再细说), 值域一般是实数.} 也可以有类似的操作
(翻出了我本科的毕业论文中的一段) :

\begin{figure}[!h]
\centering
\includegraphics[width=0.9\textwidth]{img/image-20231101180718605.png}
%\caption{image-20231101180718605}
\end{figure}

除此之外, 很多三角函数的积分可以利用三角函数的恒等式来化简;
两个多项式相除的积分可以利用部分分式 (partial fraction) 化简,
即将其变为若干个分式之和的形式; 更复杂的一些积分可以参考积分表,
作为一个现代人更可以合理使用一些计算软件辅助,
没有必要过分地去追求精通很多积分; 而且实际上,
很多积分并不能得到很好的解析式, 一般会用数值法来估算结果.

\section{参数化曲线}\label{021}

\ref{004}\nameref{004}中提到过用隐函数来表示一些图像, 事实上,
我们还可以利用\textbf{参数化曲线} (parametric curve)
来比较方便表示一些较为复杂的图像. 还是以圆形位于原点的圆作为一个 trivial
例子, \ref{005}\nameref{005}中我们简单地了解过极坐标, 不难发现, 圆周上的任意一点
\((x,y)\) 都可以写作 \((\cos\theta, \sin\theta)\); 于是, 我们可以将
\(\theta\) 视作圆的一个参数, 在 \(\theta\) 从 \(0\) 变为 \(2\pi\)
的过程中, 下面这一组函数便可绘出单位圆: \[
\begin{cases}
x(\theta)=\cos\theta\\
y(\theta)=\sin\theta
\end{cases}.
\]

如果希望求得这个函数某处的切线, 我们需要知道切线的斜率,
切线的斜率通常是通过 \(\frac{\mathrm{d}y}{\mathrm{d}x}\)求得的,
那么现在应该怎么办呢? 利用链式法则 (参见\ref{013}\nameref{013}) 可得 \[
\frac{\mathrm{d}y}{\mathrm{d}\theta}=\frac{\mathrm{d}y}{\mathrm{d}x}\frac{\mathrm{d}x}{\mathrm{d}\theta},
\] 所以我们可以先对前面的参数方程关于 \(\theta\) 求导,
然后利用下式得到其在某处的斜率 \[
\frac{\mathrm{d}y}{\mathrm{d}x}=\frac{\frac{\mathrm{d}y}{\mathrm{d}\theta}}{\frac{\mathrm{d}x}{\mathrm{d}\theta}}.
\]

\begin{newquote}
数学人震怒-1: 虽然不能这么理解, 但是为了记忆方便, 可以想象两个
\(\mathrm{d}\theta\) ``约掉了''.
\end{newquote}

\subsection{弧长}

弧长 (arc length) 和圆的弧长其实关系并不大,
这个词在这里被用来描述一个函数图像的一段曲线长度.
初一看可能函数图像的一段曲线长度很难被描述, 但是要注意,
我们现在已经有了微积分这一非常有力的工具, 我们可以从''极限'',
``微元''这些角度去思考. 这种思路在物理和工程上尤为重要,
当我们试图找到变量之间的微分关系时.

回到弧长, 当我们试图找到一段函数曲线的长度时, 我们可以先 zoom in (放大),
然后 focus on (观查) 其中非常小的一段, 当只关注其中非常小的一段时,
这一小段应该表现得足够线性 (如果不够线性, 那就是放大得还不够,
关注的区域不够小), 那么这一段弧长 \(\Delta l\)
应该可以利用勾股定理近似为 \(\sqrt{\Delta x^2+\Delta y^2}\),
然后取极限便有 \[
\boxed{\mathrm{d}l=\sqrt{\mathrm{d}x^2+\mathrm{d}y^2}}.
\] 当我们已知函数 \(y(x)\) 的情况下, 可以将上式改写为 \[
\mathrm{d}l=\sqrt{1+\left(\frac{\mathrm{d}y}{\mathrm{d}x}\right)^2}\mathrm{d}x.
\]

\begin{newquote}
数学人震怒-2: \textbf{相当于}把 \(\mathrm{d}x\) 提取出来了,
当然这么理解数学上是非常不严谨的, 但是物理和工程上 it just works.
\end{newquote}

于是, 在实际计算中, 若给定了函数 \(y(x)\),
其弧长便可以利用如下所示的积分计算 \[
l=\int\mathrm{d}l=\int\sqrt{1+\left(\frac{\mathrm{d}y}{\mathrm{d}x}\right)^2}\mathrm{d}x.
\]

\begin{newquote}
类似上式的思路在物理中很常见:

\begin{itemize}
\item
  比如对于质量分布不均匀的物体, 给定密度关于位置的函数 \(\rho(x,y,z)\),
  求质量, 利用 \(m=\rho V\) 便有:
  \(m=\int\mathrm{d}m=\int\rho(x,y,z)\mathrm{d}V=\iiint\rho(x,y,z)\mathrm{d}x\mathrm{d}y\mathrm{d}z\),
  这个积分的边界是这个物体的表面,
  所以最后一步的三重积分的上下限需要被非常小心地确定;
\item
  比如给定了电荷的分布, 求电场,
  利用\(E=\frac{Q}{4\pi\epsilon_0 r^2}\)有:
  \(\mathrm{d}E=\frac{1}{4\pi\epsilon_0}\frac{\mathrm{d}Q}{r^2}=\frac{1}{4\pi\epsilon_0}\frac{\rho}{r^2}\mathrm{d}V=\cdots\).
  当然, 严格来说这会是一个向量的积分, 暂时就不具体讲了.
\item
  \ldots{}
\end{itemize}

总而言之, 先从比较通常的情况 (质量均匀分布/点电荷/\ldots)
的变量关系出发, 然后一步步抽丝剥茧地将微元变为可以积分的变量为止
(\(\mathrm{d}m\Rightarrow \mathrm{d}V\Rightarrow\mathrm{d}x\mathrm{d}y\mathrm{d}z\))
.
\end{newquote}

那么如果给定的形式是一个参数化曲线呢? 例如, 已知 \(\{x(t),y(t)\}\),
那么可以先对 \(x(t)\) 和 \(y(t)\) 先分辨关于 \(t\) 求导, 得到
\(\mathrm{d}x\) 和 \(\mathrm{d}y\) 与 \(\mathrm{d}t\) 之间的微分关系,
然后弧长的微元便可以写作 \[
\mathrm{d}l=\sqrt{\left(\frac{\mathrm{d}x}{\mathrm{d}t}\right)^2+\left(\frac{\mathrm{d}y}{\mathrm{d}t}\right)^2}\mathrm{d}t.
\] 对其积分便可得到弧长.


\backmatter
\addcontentsline{toc}{chapter}{Index}
\printindex
\end{document}

\documentclass{article}
\usepackage[utf8]{inputenc}
\usepackage{CJKutf8, indentfirst}
\usepackage{graphicx} % Required for inserting images

\title{0}
\author{Zibo Wang}
\date{April 2023}

\begin{document}
\begin{CJK*}{UTF8}{gbsn}
    \maketitle

    \section{Introduction}
    \hypertarget{ux52a0ux6cd5-addition}{%
\subsubsection{加法 (addition)}\label{ux52a0ux6cd5-addition}}

\(1+1=2\) , 好了, 这一节结束 (玩笑). 加法的运算, 九九加法表,
这里就不赘述.

我们首先考虑\textbf{整数} (integer), 记作 \(\mathbb{Z}\),
当我们考虑一类数字或者一类符合某种性质的对象时,
我们称这一类对象组成的东西叫做\textbf{集合} (set),
集合中的东西便叫做\textbf{元素} (element).

\begin{itemize}
\tightlist
\item
  不难发现, 从整数里取出两个元素, 例如 \(1\) 和 \(2\) , 将他们相加,
  得到的结果 \(1+2=3\) 依旧是整数.
  这样的性质叫做\textbf{封闭性}或者\textbf{闭包性} (closed, closure).
\item
  再考虑三个元素在加法下的运算. 三个元素的运算可以看作,
  两个元素的运算结果与第三个元素再次运算, 还是不难发现,
  任意从整数取出三个元素, 例如 \(1\) , \(2\) 和 \(3\) , 有
  \((1+2)+3=6=1+(2+3)\) ,
  即前两个元素先进行运算和后两个元素先进行运算的结论时一致的.
  这样的性质叫做\textbf{结合律} (assosiative).
\item
  整数里存在一个特殊的元素,
  使得加法这个运算不对其他任何元素''产生效果'', 这个特殊的元素是 \(0\) ,
  \(0+n=n+0=n\) , 这里 \(n\) 是任意整数, 我们可以这么标记:
  \(n\in\mathbb{Z}\) , 中间这个符号表示属于 (belong to).
  这个特殊的元素叫做\textbf{单位元}或\textbf{幺元} (identity
  element)\footnote{单位和幺都有一的含义, 因为在乘法中单位元是1,
    这可能是名字来源.}.
\item
  整数的加法中, 对于任何一个元素, 都能找到另一个元素,
  使得它们运算结果为单位元- \(0\) , 比如 \(1+(−1)=(−1)+1=0\) , 我们便叫
  \((−1)\) 是 \(1\) 的\textbf{逆元} (inverse element).
\end{itemize}

一个集合, 再附加一个二元运算(像加法这样输入两个元素输出一个元素的运算),
并且拥有上述性质和元素的, 我们便把它叫做\textbf{群} (group), 整数和加法,
便是这样构成了一个群 \((\mathbb{Z},+)\) .

好的,我们在学习加法的过程中顺便体验了以下群论. 要注意的是,
上文并没有强调\textbf{交换性} (commutative), 因为往后看我们会发现,
很多运算其实并不满足交换律,
满足交换律的群我们可以称它为\textbf{交换群}或\textbf{阿贝尔群} (Abelian
group).

\begin{quote}
有人问一个小朋友, ``3+4 等于几啊?'' 小朋友说: ``不知道, 但我知道 3+4
等于 4+3.'' 那人接着问: ``为什么呀?''
答曰:``因为整数与整数加法构成了阿贝尔群.''
这个笑话讽刺了某次法国一场幼儿园从抽象数学教起的实验,
不过最后实验的结果是以失败告终.
\end{quote}

\hypertarget{ux51cfux6cd5-subtraction}{%
\subsubsection{减法 (subtraction)}\label{ux51cfux6cd5-subtraction}}

思路要打开, 减法可以看作是加法的逆运算; 又或者, 减去一个数,
可看作加上这个数字的逆元.

减法的一些性质:

\begin{itemize}
\tightlist
\item
  \textbf{反交换律} (anti-commutativity), 例如 \(4−3=−(3−4)\) ,
  交换两个元素的顺序会导致结果变为之前结果的逆.
\item
  \textbf{非结合律} (non-associativity), 例如 \((6−3)−2\neq 6−(3−2)\) .
\end{itemize}

因为整数减法不满足结合律, 所以整数和减法不构成群, 只构成\textbf{拟群}
(quasi-group), 字面上可以理解成, 像群, 但是不是群, 这里不做展开.

\hypertarget{ux4e58ux6cd5-multiplication}{%
\subsubsection{乘法
(multiplication)}\label{ux4e58ux6cd5-multiplication}}

还是九九乘法表, 结束了 (玩笑). 乘法可以视作是多个同样加法的标记, 例如:
\(2×3=2+2+2=3+3=3×2\) .

来看看乘法的一些性质:

\begin{itemize}
\tightlist
\item
  易见\textbf{封闭性}或者\textbf{闭包性}是满足的.
\item
  也不难看出乘法具有\textbf{结合律}.
\item
  乘法的\textbf{幺元}是 1 .
\item
  再\textbf{逆元}上似乎出了问题, 到目前为止, 我们讨论的都还是整数
  \(\mathbb{Z}\) , 这个范围内, 似乎找不到逆元, 但是没有关系,
  我们把范围扩大到非零\textbf{有理数} (rational number)\footnote{有理数其实是谬译,
    rational在这里其实意为可约的, 而不是有理的.}, 记作
  \(\{\mathbb{Q}/\{0\}\}\) , 这样每一个元素 \(n\in\{\mathbb{Q}/\{0\}\}\)
  都有逆元 \(\frac{1}{n}\) . 将 \(0\) 剔除是因为它没有逆元, 我们应该知道
  0 不能作为分母.
\end{itemize}

以上性质已经决定了非零有理数和乘法构成群, \((\mathbb{Q}/\{0\}\,×)\) .
乘法另外还有特性:

\begin{itemize}
\tightlist
\item
  乘法与加法的混合运算, 会有\textbf{分配律} distributive property, 例如
  \(2×(3+4)=2×3+2×4\) .
\item
  任何数乘上 \(0\) 得到 \(0\) , \(0\) 可以称作乘法的\textbf{零元} (zero
  element), 零元没有逆.
\end{itemize}

\hypertarget{ux9664ux6cd5-division}{%
\subsubsection{除法 (division)}\label{ux9664ux6cd5-division}}

乘法和除法的关系类似加法和减法的关系. 除以零在大多数场景下是不被定义的.

    \begin{quote}
轻清者上浮而为天 重浊者下凝而为地
\end{quote}

\hypertarget{ux5e42ux8fd0ux7b97-exponentiation}{%
\subsubsection{幂运算
(exponentiation)}\label{ux5e42ux8fd0ux7b97-exponentiation}}

幂运算可以视作重复的乘法, 即
\(a^n=\underbrace{a\times ...\times a}_{n}\), 这里 \(a\) 称为底数 (base)
, \(n\) 称为指数 (exponent)\footnote{从这一篇开始,
  文章的叙述讲逐渐从''具体→抽象''过渡到''抽象→具体'',
  即由之前先给一个具体数字运算的例子推广到用字母表示的通常情况,
  变为反过来的顺序; 阅读过程中如果觉得不适应, 抽象的点读不懂时,
  可以先接着往下看, 若之后有一个具体的例子, 可能对理解会有帮助.},
\(a^n\) 读作 \(a\) 的 \(n\) 次幂,或 \(a\) 的 \(n\) 次方.

先考虑正整数次幂, 一些运算规律:

\begin{itemize}
\tightlist
\item
  \(\begin{aligned}a^m\times a^n=\underbrace{a\times ...\times a}_{m}\times\underbrace{a\times ...\times a}_{n}&&\\  =\underbrace{a\times ...\times a}_{n+m}&&=a^{n+m}\end{aligned}\)
\item
  \(\begin{aligned}a^m\div a^n=\underbrace{a\times ...\times a}_{m}\div\underbrace{(a\times ...\times a)}_{n}&&\\  =\underbrace{a\times ...\times a}_{n-m}&&=a^{n-m}\end{aligned}\)
\end{itemize}

再来考虑 \(0\) 次幂, 因为上述运算规律 \(a^n\times a^0=a^{n+0}=a^n\),
因此应该有 \(a^0=1\) ; 要注意, 当底数为 \(0\) 时, \(0^0\)
是不被定义的\footnote{一说理由和 \(0\) 不能作为除数类似;
  另一说要从函数的角度出发, 这里稍稍剧透, 即,
  构建不同的函数极限试图求这个''值''会有不同的结果,
  所以这个''值''没有一个很好的公认的定义.}.

\begin{itemize}
\tightlist
\item
  \(a^0=1\)对于非零的 \(a\).
\end{itemize}

现在来看负整数为指数的幂, 参考第二条规律, 不难看出 \(a^{-n}\)
可以理解为除掉了 \(n\) 个 \(a\), 因此有

\begin{itemize}
\tightlist
\item
  \(a^{-n}=\frac{1}{a^n}\).
\end{itemize}

因为在前面一节我们已经把我们研究的范围扩充到了所有有理数,
所以不妨来看看分数作为指数的情况. 考虑 \(a^{\frac{1}{2}}\), 这里 \(a\)
是有理数, 令 \(b:=a^{\frac{1}{2}}\), 平方可得
\(b^2=a^{\frac{1}{2}}\times a^{\frac{1}{2}}=a^{\frac{1}{2}+\frac{1}{2}}=a\);
事实上我们知道, 要求 \(b\) 的话, 只需进行''开方''这个操作, 记作
\(b=\sqrt{a}\), 因此有 \(a^{\frac{1}{2}}=b=\sqrt{a}\); 然而,
这个操作其实是有一点''小问题''的.

\begin{quote}
这个''小问题''便是, 目前为止, 我们讨论的范围还限于有理数,
然而上述操作得到的 \(\sqrt{a}\) 并不一定是有理数;
这个问题在历史上也困扰了人们很久.

起初人们认为数轴上所有的数都应该可以用整数之比 (也就是有理数) 来表示,
但有人发现, 例如边长为1的正方形, 其对角线的平方利用\textbf{勾股定律}
(Pythagorean theorem - 毕达哥拉斯定律) 应该是 \(2\),
找不出一个有理数使得其平方正好为 \(2\).

然后为了解决问题, 提出问题的人就被解决掉了, 悲伤的故事.

现在, 平方正好为 \(2\) 的数字被记作了 \(\sqrt{2}\),
它不是有理数的证明可以留作证明, 一点提示就是可以利用反证法,
首先假设它是一个有理数, 并可以表示为例如 \(\frac{p}{q}\), 且 \(p\) 和
\(q\) 都是正整数, 然后证明这样的 \(\frac{p}{q}\) 不可能存在.
\end{quote}

解决这个''小问题''的方法, 是要再次扩展我们研究的范围; 这次我们将有理数,
即整数和分数, 以及数轴上''剩余''的那些不能表示成分数形式的无理数,
统称为\textbf{实数} (real number), 记作 \(\mathbb{R}\).
这样一来我们便不必担忧开方的结果''掉到''范围外了, 上面的结论也不难推广为

\begin{itemize}
\tightlist
\item
  \(a^{\frac{n}{m}}=\sqrt[m]{a^n}=(\sqrt[m]{b})^n\).
\end{itemize}

\hypertarget{ux5bf9ux6570-logarithm}{%
\subsubsection{对数 (logarithm)}\label{ux5bf9ux6570-logarithm}}

对数是幂运算的逆运算. 若 \(y=a^x\), 定义对数运算为 \(x=\log_a(y)\),
\(a\) 叫做底数 (base) , \(y\) 叫做真数.

对数有以下运算规律:

\begin{itemize}
\tightlist
\item
  \$ \log\_a(XY)= \log\_a(X)+ \log\_a(Y)\$. 证明如下: 令
  \(x=\log_a(X)\), \(y=\log_a(Y)\), 根据对数定义则有 \(a^x=X\),
  \(a^y=Y\);
  \(\log_a(XY)=\log_a(a^x\times a^y)=\log_a(a^{x+y})=x+y=\log_a(X)+ \log_a(Y)\).
\item
  \$ \log\_a\left(\frac{X}{Y}\right)= \log\_a(X)- \log\_a(Y)\$.
  证明和上一条类似.
\item
  \(\log_a(x^n)=n\log_a(x)\). 由第一条规律可得
  \(\log_a(x^n)=\underbrace{\log_a(x)\times...\times\log_a(x)}_{n}=n\log_a(x)\).
\item
  \(\log_a(x)=\frac{\log_b(x)}{\log_b(a)}\). 令 \(\log_a(x)=t\), 则有
  \(x=a^t\), 对两边同时取以 \(b\) 为底数的对数,
  \(\log_b(x)=\log_b(a^t)=t\log_b(a)=\log_a(x)\log_b(a)\),
  整理便可得上述规律.
\end{itemize}

以上四条为最基本最常用得运算规律,
还有一些运算规律可以从上面几条推到而来, 证明留作练习.

\begin{itemize}
\tightlist
\item
  \(\log_{a^n}x=\frac{1}{n}\log_{a}x\).
\item
  \(a^{\log_a(x)}=\log_a(a^x)=x\).
\item
  \(x^{\log_a(y)}=y^{\log_a(x)}\).
\item
  \(\log_a(x)=\frac{1}{\log_x(a)}\).
\item
  \(\log_a(b)\log_b(x)=\log_a(x)\).
\end{itemize}

    \input{003}
    \input{004}
    \hypertarget{ux4e09ux89d2ux51fdux6570-trigonometry}{%
\subsubsection{三角函数
(Trigonometry)}\label{ux4e09ux89d2ux51fdux6570-trigonometry}}

三角函数最基本的使用应该是表示直角三角形的变长比. 如下图所示, 三角形
\[ABC\] 为直角三角形, 将 \[\angle BAC\] 记作 \[\theta\], 对于两条直角边
\[AB\] 和 \[BC\], 边 \[AB\] 在 \[\theta\] 边上, 称它为\textbf{邻边}
(adjacent), 边 \[BC\] 在 \[\theta\] 对面, 称它为\textbf{对边}
(opposite), 剩余的边 \[AC\] 被称为\textbf{斜边} (hypotenuse)。

易见, 各变长比仅和 \[\theta\] 相关\footnote{当然也可以说和除了直角外的另一个角
  \[(90^\circ-\theta)\] 相关; 边长比可以通过一个除直角外的角确定是因为,
  除直角外另一角相等的直角三角形都相似, 它们的边长比是一致的。},
三角函数便是用来表示各个比例的, 常用的三角函数有

\[\begin{align*}\cos\theta&=\frac{邻边}{斜边}=\frac{AB}{AC},\\
\sin\theta&=\frac{对边}{斜边}=\frac{BC}{AC},\\
\tan\theta&=\frac{对边}{邻边}=\frac{BC}{AB}.\end{align*}\]

不难看出\[\tan\theta=\frac{\sin\theta}{\cos\theta}\].

另外还有

\[\begin{align*}\sec\theta&\equiv\frac{1}{\sin\theta},\\
\csc\theta&\equiv\frac{1}{\cos\theta},\\
\cot\theta&\equiv\frac{1}{\tan\theta}.\end{align*}\]

\[\csc\] 很多时候也记作 \[\text{cosec}\].

一个非常实用的关系, 直角三角形中有\textbf{勾股定理} (Pythagorean
theorem): 斜边边长平方等于两直角边边长的平方之和, 即 \[AC^2=AB^2+BC^2\];
两边同时除以 \[AC^2\] 便有

\begin{itemize}
\tightlist
\item
  \[\boxed{1=\cos^2\theta+\sin^2\theta}\].\footnote{三角函数的平方:
    cos(x)\textsuperscript{2} 通常理解为 cos((x)\textsuperscript{2});
    cos\textsuperscript{2}x 约定俗成表示 (cos(x))\textsuperscript{2}.}
\end{itemize}

\textbf{正弦定律 Law of sine}

将三角形三个角分别记作 \[\alpha\], \[\beta\], 和 \[\gamma\],
将它们的对边分别记作 \[A\], \[B\], 和 \[C\]. 先是结论:

\begin{itemize}
\tightlist
\item
  \[\boxed{\frac{A}{\sin\alpha}=\frac{B}{\sin\beta}=\frac{C}{\sin\gamma}}\].
\end{itemize}

推导如下:

如上图所示, 以 \[C\] 为底做高, 将原本的三角形分为左右两个直角三角形,
这条高利用左边的直角三角形可以表示为 \[A\sin\beta\],
利用右边的直角三角形则是 \[B\sin\alpha\], 于是有
\[A\sin\beta=B\sin\alpha\], 整理可得
\[\frac{A}{\sin\alpha}=\frac{B}{\sin\beta}\];
再做另一条高重复前面的操作, 便可得到完整的结论.

\textbf{余弦定律 Law of cosine}

还是先上结论:

\begin{itemize}
\tightlist
\item
  \[\boxed{B^2=A^2+C^2-2AC\cos\beta}\],
\end{itemize}

即, 【一条边的边长平方】等于【另两条边的边长平方之】和加上【两倍的
(另两条边边长的乘积) 乘以 (另两条边的夹角的余弦)】.

推导如下:

如下图所示, 依旧利用底边 \[C\] 上的高将其分为左右两个直角三角形;
左边的直角三角形, 利用斜边 \[A\] 和角 \[\beta\], 两直角边分别可以表示为
\[A\cos\beta\] 和 \[A\sin\beta\], 于是右边的直角三角形边长便可表述为
\[A\sin\beta\] 和 \[(C-A\cos\beta)\]; 对右边的直角三角形使用勾股定理

\[\begin{align*}B^2&=A^2\sin^2\beta+(C-A\cos\beta)^2\\
&=A^2\sin^2\beta+C^2+A^2\cos^2\beta-2AC\cos\beta\\
&=A^2+C^2-2AC\cos\beta.\end{align*}\]

其中等式的后两行用到了之前得出的 \[1=\cos^2\theta+\sin^2\theta\].

\hypertarget{ux4efbux610fux89d2ux5ea6ux7684ux4e09ux89d2ux51fdux6570}{%
\subsubsection{任意角度的三角函数}\label{ux4efbux610fux89d2ux5ea6ux7684ux4e09ux89d2ux51fdux6570}}

不难发现, 前面讨论的情况似乎都是锐角的情况 (主要是因为插图\ldots),
钝角的三角函数似乎没那么直观了, 因为做不成一个含有钝角的直角三角形,
没法简单地用边长比来表示 \[\sin\] 和 \[\cos\] 等. 于是,
我们需要想办法将前面的情形推广.

如下左图所示, 建立直角坐标系, 做一圆心位于原点的单位圆, 即半径为 \[1\]
的圆, 考虑在第一象限的圆上的一点, 将其与原点做连线, 将从
\[x\]-轴正方向与这条连线\textbf{顺时针}方向形成的夹角记作 \[\theta\],
不难看出这个点的坐标 \[(x,y)\] 满足

\[\begin{cases}x=\cos\theta,\\y=\sin\theta.\end{cases}\]

于是不妨将其他象限的情况也按此定义,
于是如上右图所示的钝角甚至更大角度的三角函数便可以被定义了.

\hypertarget{ux5f27ux5ea6ux5236-radian}{%
\subsubsection{弧度制 (Radian)}\label{ux5f27ux5ea6ux5236-radian}}

为什么一个周角是 \[360^\circ\] 呢, 听说过一个不可考的说法: \[360\]
是一个有很多因数的数字 (1, 2, 3, 4, 5, 6, 8, 9, 10, 12\ldots),
等分起来的时候数字会比较友好, 所以 \[360^\circ\] 其实是非常随意地规定的.
那么有没有更好的用来描述角度方法呢? 答案是弧度.

一个半径为 \[r\] 的圆的周长是 \[2\pi r\], 一个圆心角为 \[n^\circ\]
的扇形的弧长是 \[2\pi r\frac{n}{360}\]. 可见圆心角越大弧越长,
且圆心角和弧长成正比. 既然如此,
不如重新将角度定义为圆心角与弧长的比值以方便计算, 于是便有了,
在新的这套单位系统中, 若圆心角大小为 \[\theta\], 其对应弧长应为
\[r\theta\]; 当圆心角是一个周角时, 对应弧长便成了圆的周长 \[r(2\pi)\].
所以角度和这个新的单位的换算有 \[360^\circ\equiv 2\pi\ \text{rad}\],
因为这个单位把圆心角和对应的弧长联系起来了, 因此称之为\textbf{弧度}
(radian).

扇形面积在这套单位制, 即弧度制下, 便也成了 \[\frac{1}{2}r^2\theta\].

\hypertarget{ux4e09ux89d2ux51fdux6570ux7684ux56feux50cf}{%
\subsubsection{三角函数的图像}\label{ux4e09ux89d2ux51fdux6570ux7684ux56feux50cf}}

现在这个时代, 大家都或多或少能接触到科学计算器,
再不济在bing.com上搜索''solver''用微软的 Microsoft Solver
也可以计算某个特定角度的三角函数值, 自然也可以绘制函数图像.
下图分别展示了 \[\sin(x)\] 和 \[\cos(x)\] 的图像,

一些值得关注的点是它们都是\textbf{周期函数} (periodic function),
随着自变量-角度的变化, 因变量-函数值的变化是周期性的, 它们的周期都是
\[2\pi\], 这一点从上文的单位圆里便可看出些许原因,
当角度变化超过一个周角时, 和角度刚从 \[0\] 开始的情况是一样的.

一个个人很喜欢的可视化如下:

右下显示的是角度 \[\theta\] 不断增加, 左下的图可以看作右下的点的 \[y\]
坐标也就是 \[\sin\theta\] 的值的变化, 左上则是 \[x\] 坐标也就是
\[\cos\theta\] 的值的变化.

    \input{006}
    \input{007}
    \input{008}

\end{CJK*}
\end{document}
