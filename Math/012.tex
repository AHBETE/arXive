\hypertarget{ux5bfcux6570-derivative}{%
\subsubsection{导数 (derivative)}\label{ux5bfcux6570-derivative}}

初中阶段常有求二次函数切线的题目,
利用初中知识求切线的表达式往往过程冗长, 事实上, \textbf{微积分}
(calculus) 会大大简化这个过程. 微积分, 如其名所示, 分为\textbf{微分}
(differentiation) 和\textbf{积分 }(integration).

在这里我们姑且先用经典的角度来了解微积分. \textbf{微分} (differential)
可以不严谨的理解为\textbf{无穷小量} (infinitesimal),
这是一些物理教材中更常见的表述, 例如某函数 \(y=f(x)\),
它的一个有限的变化通常记作 \(\Delta y\), 这个小三角念作''delta'',
在很多场景下表示变化, 而一个无穷小量的变化便记作 \(\mathrm{d}y\).
任何实数都比无穷小量大, 就像无穷大 \(\infty\) 大于任何实数{[}\^{}1{]}.

另有一个非常类似的概念叫做\textbf{导数}, 导数可以理解为函数的变化率.
考虑某函数 \(y=f(x)\) 在一个很小但是有限的一个定义域区间 \([a,b]\),
有自变量的变化为 \(\Delta x=b-a\), 函数值的变化为
\(\Delta y= f(b)-f(a)\), 于是在此区间内''平均''变化率是

\(\frac{\Delta y}{\Delta x}=\frac{f(b)-f(a)}{b-a}.\)

若是希望得到 \(x=a\) 处的变化率, 我们令 \(h:=b-a\), 直觉上应该是
\(\lim_{h\rightarrow0}[f(a+h)-f(a)]/h\), 这便是函数 \(f(x)\) 在 \(x=a\)
处的变化率, 或 \(f(x)\) 在 \(x=a\) 处的导数. 推广到任意位置 \(x\), 遂有:

\textbf{定义}: 函数 \(f(x)\) 的导(函)数\footnote{取某个特定的点 x=x\_0,
  得到的导数是一个具体数值, 便叫它导数; 若不将 x 锚定到一个特定的值,
  导数便依旧是一个含 x 的函数, 便叫它导函数. 下文不再区分. 类似的,
  后文再出现的微分, 导数和无穷小量也不再严格区分.}为

\(\lim_{h\rightarrow0}\frac{f(x+h)-f(x)}{h}.\)

很多数学教材会这么标记导数

\(\boxed{f'(x)=\lim_{h\rightarrow0}\frac{f(x+h)-f(x)}{h}}.\)

物理教材更偏好

\(\frac{\mathrm{d}y}{\mathrm{d}x}=\lim_{h\rightarrow0}\frac{f(x+h)-f(x)}{h}.\)

\begin{quote}
举一个例子, 若一个物体非匀速运动, 将速度 \(v\) 表示为一个关于时间 \(t\)
的函数 \(v(t)\), 那么便有加速度 (acceleration) 即速度的变化率:
\(a(t)=v'(t)=\frac{\mathrm{d}v}{\mathrm{d}t}\).

另: 很多时候, 物理中, 关于时间的求导还有一个标记,
\(\dot{f}\equiv \frac{\mathrm{d}f}{\mathrm{d}t}\)\footnote{很多人经常吐槽物理中标记的不统一.
  个人当然也觉得, 如果有一套统一的标记, 信息的沟通自然会便利不少;
  但是学习的过程中, 既然标记不统一已经客观存在, 与其花时间吐槽,
  不如去抓住本质, 不要拘泥于标记, 而去理解标记背后的含义.}.
\end{quote}

类似 \(\frac{\mathrm{d}y}{\mathrm{d}x}\) 这样记法的好处是,
``变化率''这个概念被表现得很直观, 若有
\(\frac{\mathrm{d}y}{\mathrm{d}x}=y'(x)\), 我们可以将它改写为
\(\mathrm{d}y=y'(x)\mathrm{d}x\), 虽然''两边同乘 \(\mathrm{d}x\)''
这个说法非常不正确的, 但是从''变化率''的角度出发的确可以这么理解.

\textbf{定理}: 若 \(f(x)\) 在 \(x=c\) 存在导数 (我们也可以说, 它在
\(x=c\) 处可以被求导), 那么它在 \(x=c\) 处连续.

(不严格的) \textbf{证明}: 不使用 \(\epsilon - \delta\) 语言的话,
只需证明 \(\lim_{x\rightarrow c}f(x)=f(c)\) 即可. 于是, 对于有限的
\(h\), \(f(c+h)=f(c)+f(c+h)-f(c)=f(c)+\frac{f(c+h)-f(c)}{h}\cdot h\),
取极限 \(h\rightarrow0\) 有
\(\lim_{h\rightarrow0}f(c+h)=\lim_{h\rightarrow0}f(c)+\lim_{h\rightarrow0}\frac{f(c+h)-f(c)}{h}\cdot h=f(c)+f'(c)\cdot 0\),
第一个等号利用了极限的线性, 第二个等号则需要导数存在,
于是便有存在导数隐含 (imply) 连续.

若有函数 \(f(x)\) 和 \(g(x)\), 导数的一些性质:

\begin{itemize}
\tightlist
\item
  \((af+bg)'(x)=af'(x)+bg'(x)\), 这里 \(a\) 和 \(b\) 是常数;
\item
  \((fg)'(x)=f'(x)g(x)+f(x)g'(x)\);
\item
  \(\left(\frac{f}{g}\right)'(x)=\frac{f'(x)g(x)-f(x)g'(x)}{g^2(x)}\),
  若 \(g(x)\neq 0\).
\end{itemize}

第一条性质可以由极限的线性而来; 第二条和第三条可以分别令
\(h(x)=f(x)g(x)\) 和 \(h(x)=\frac{f(x)}{g(x)}\), 然后将 \(h(x)\)
代入导数的定义.

下面是一个非常 trivial 的例子,

\begin{quote}
\textbf{例子}: \(y(x)=x^2\), 求 \(y'(x)\). 利用定义:
\(\begin{aligned}y'(x)=&\lim_{h\rightarrow0}\frac{(x+h)^2-x^2}{h}\\=&\lim_{h\rightarrow0}\frac{x^2+h^2+2xh-x^2}{h}\\=&\lim_{h\rightarrow0}(2x+h^2)\\=&2x.\end{aligned}\)
\end{quote}

不难将结果推广为

\(\boxed{\frac{\mathrm{d}}{\mathrm{d}x}x^n=nx^{n-1}}.\)

{[}\^{} 1{]}: 【007】中有提到过''浮点数'', 例如一个有理数 a,
它不会被以分数的形式记录, 而是记录其小数形式并精确到某一位,
那么第一个非零位数低于这一位的数字 b, 它对于 a 来说在计算机数值上看来,
便等效于无穷小量. 在 MATLAB 中, 给定一个实数 a, 利用函数 eps(), eps(a)
便会输出对于 a 来说的''无穷小量''.
