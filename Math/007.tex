\begin{quote}
绕行之绕行 (detour of the detour).
\end{quote}

\hypertarget{ux4e8cux9879ux5f0fux5c55ux5f00-binomial-expansion}{%
\subsubsection{二项式展开 (binomial
expansion)}\label{ux4e8cux9879ux5f0fux5c55ux5f00-binomial-expansion}}

\textbf{二项式展开}指的是将类似 \((x+y)^n\) 的表达式展开的过程. 结论上有

\(\boxed{(x+y)^n=\sum_{r=0}^n\binom{n}{r}x^{n-r}y^r}.\)

这里 \(\binom{n}{r}=\frac{n!}{r!(n-r)!}\) 也记作 \(_nC_r\),
这个''C''是\textbf{组合} (combination) 的意思, 其中又有
\(n!=n\times(n-1)\times...\times3\times2\times1\).

\begin{quote}
上面第一次出现了求和符号 \(\sum\), 在此用一些例子说明,

\(\sum_{i=1}^{10}i=1+2+3+...+9+10;\)

\(\sum_{x=1}^{10}x^2=\left.x^2\right|_{x=1}+\left.x^2\right|_{x=2}+...+\left.x^2\right|_{x=10}=1^2+2^2+...+10^2.\)

即, 求和符号后的表达式, 依次代入求和符号下方的值,
符号下方的值加一\ldots, 直至代入求和符号上方的值, 最后将这些项求和.
\end{quote}

二项式展开的证明思路如下:

\textbf{从组合的思路出发}

\begin{itemize}
\item
  将 \((x+y)^n=\underbrace{(x+y)(x+y)...(x+y)}_{n}\) 展开,
  并将同类项合并, 易见可能出现的项的形式仅为 \(x^n=x^ny^0\),
  \(x^{n-1}y=x^{n-1}y^1\), \(x^{n-2}y^2\), \ldots{} \(x^2y^{n-2}\),
  \(xy^{n-1}=x^1y^{n-1}\), \(y^n=x^0y^n\).
\item
  合并后这些项的系数是合并前它们分别出现的次数,

  \begin{itemize}
  \tightlist
  \item
    \(x^n\) 相当于展开时每一个 \((x+y)\) 都选取 \(x\) 的情况,
    只有一种这样的情况, 即合并前 \(x^n\) 只可能出现 \(1\) 次,
    那么合并后它的系数便是 \(1\).
  \item
    \((x^{n-1}y^1)\) 相当于展开时每一个 \((x+y)\) 选取了 \((n-1)\) 个
    \(x\) 和 \(1\) 个 \(y\), 利用组合学的知识有
    \(n=\frac{n!}{1!(n-1)!}\) 种这样的情况, 即合并前 \((x^{n-1}y^1)\)
    出现 \(n\) 次, 那么合并后它的系数便是 \(n\).
  \end{itemize}

  \begin{quote}
  这个 \(n\) 可以这么看待, 选取的这个 \(y\) 可以出自这 \(n\) 项
  \((x+y)\) 中的任意一个, 于是便有 \(n\) 种可能.
  \end{quote}

  \begin{itemize}
  \tightlist
  \item
    \((x^{n-2}y^2)\) 相当于展开时每一个 \((x+y)\) 选取了 \((n-2)\) 个
    \(x\) 和 \(2\) 个 \(y\), 利用组合学的知识有
    \(\frac{n(n-1)}{2!}=\frac{n!}{1!(n-1)!}\) 种这样的情况, 即合并前
    \((x^{n-1}y^1)\) 出现 \(n\) 次, 那么合并后它的系数便是 \(n\).
  \end{itemize}

  \begin{quote}
  这个 \(\frac{n(n-1)}{2!}\) 可以这么看待, 选取的这两个 \(y\) 可以出自这
  \(n\) 项 \((x+y)\) 中的任意两个, 第一个 \(y\) 有 \(n\) 种选法,
  第二个因为第一个''占用''了一个 \((x+y)\), 因此它只有 \((n-1)\) 种选法,
  综上便有了 \(n(n-1)\); 然后两个 \(y\) 的顺序是无所谓的, 两个 \(y\)
  本身先后的排序会额外引入一个倍数 \(2\), 于是除掉.
  \end{quote}

  \begin{itemize}
  \tightlist
  \item
    \ldots{}
  \item
    \((x^{n-r}y^r)\) 相当于展开时每一个 \((x+y)\) 选取了 \((n-r)\) 个
    \(x\) 和 \(r\) 个 \(y\), 利用组合学的知识有
    \(\frac{n(n-1)...(n-r)}{(n-r)!}=\frac{n!/r!}{(n-r)!}=\frac{n!}{1!(n-1)!}=\binom{n}{r}\)
    种这样的情况, 即合并前 \((x^{n-1}y^1)\) 出现 \(\binom{n}{r}\) 次,
    那么合并后它的系数便是 \(\binom{n}{r}\).
  \end{itemize}

  \begin{quote}
  这个 \(\frac{n(n-1)...(n-r)}{(n-r)!}\) 可以这么看待, 选取的这 \(r\) 个
  \(y\) 可以出自这 \(n\) 项 \((x+y)\) 中的任意 \(r\) 个, 第一个 \(y\) 有
  \(n\) 种选法, 第二个因为第一个''占用''了一个 \((x+y)\), 因此它只有
  \((n-1)\) 种选法, 第三个于是只有 \((n-2)\) 种\ldots{} 综上便有了
  \(n(n-1)...(n-r)\); 然后 \(r\) 个 \(y\) 的顺序是无所谓的, \(r\) 个
  \(y\) 本身先后的排序, 第一个 \(y\) 顺序可能是 \(1\) 至 \(r\), 有 \(r\)
  种选择, 第二个只有 \((r-1)\)\ldots{} 于是会额外引入一个倍数
  \(r(r-1)...1=r!\), 于是除掉.
  \end{quote}
\item
  可见某一项 \((x^{n-r}y^r)\), 系数应为 \(\binom{n}{r}\), \(r=0\) 至
  \(r=n\) 的项都是允许的, 于是利用求和符号表示, 便有了最开始的结论.
\end{itemize}

这样的思路也可以推出杨辉三角 (Pascal's Triangle):

\begin{Shaded}
\begin{Highlighting}[]
    \FloatTok{1}
   \FloatTok{1}  \FloatTok{1}
  \FloatTok{1}  \FloatTok{2}  \FloatTok{1}
 \FloatTok{1}  \FloatTok{3}  \FloatTok{3}  \FloatTok{1}
\FloatTok{1}  \FloatTok{4}  \FloatTok{6}  \FloatTok{4}  \FloatTok{1}
\end{Highlighting}
\end{Shaded}

三角的左右两边由 \(1\) 填满, 中间的某个数字是左上和右上两个数字之和.
不难发现, 从第二行开始, 每一行的数字是都是二项式展开的系数.

上述的推导, 和类似【抛 \(n\) 次公平的硬币, 得到 \(r\) 次正面和 \((n-r)\)
次反面】的场景有着非常深的联系, 这里暂时不做展开.

\textbf{数学归纳法}

这个方法一般只能用于证明, 不能用于推导.

\begin{quote}
\textbf{数学归纳法} (proof by induction) 思路如下

\begin{enumerate}
\def\labelenumi{\arabic{enumi}.}
\tightlist
\item
  \textbf{归纳奠基} (base case), 证明第一个情况是对的;
\item
  归纳递推, 假设第 \(n\) 个情况正确, 以此推出第 \((n+1)\) 个情况正确,
  便有所有情况都成立.
\end{enumerate}

已知若情况n成立便有情况(n+1)也成立; 因为有情况1成立, 于是代入n=1,
便有情况2也成立; 现在知道情况2也成立了, 继续代入n=2,
便有情况3也成立\ldots{}
\end{quote}

思路已经给到, 具体证明留作练习. 一点提示是
\(\binom{r}{n+1}=\binom{r}{n}+\binom{r-1}{n}\).

\textbf{应用}

除了常规的 \(n\) 是整数的一些应用, 在保证展开的形式是\textbf{收敛}
(coverge) 的情况下 (即求和的形式不会趋向于正/负无穷),
二项式展开的负整数, 甚至分数形式也是成立的.

例如狭义相对论 (special relativity) 中, 随着物体运动速度变化,
物体的相对论性质量 (relativitic mass)\footnote{静止质量是物体静止时的质量,
  或者说某个观察者发现某物体处于静止状态下时这个物体的质量;
  相对论性质量则是物体相对观察者具有一定速度时, 观察者观察到的质量.}会变大,
它和静止质量 \(m_0\) 符合关系式

\(m=\frac{m_0}{\sqrt{1-v^2/c^2}}.\)

上式中 \(v\) 时速度, \(c\) 是光速. 根据幂运算的规律 (复习【002】),
上式可以改写成

\(m=m_0(1-v^2/c^2)^{-1/2}.\)

在估算例如速度在 \(0.01c\) 或更小时, 相对论性质量与静止质量之差,
直接计算 \((m-m_0)\) 通常看不出 \(m\) 与 \(m_0\) 的区别\footnote{计算机保存的并不是准确值,
  而是浮点数 (暂不展开), 可以暂且不太正确但道理就这么个道理地理解为:
  它保存的答案是一个写成科学计数法的数值, 并且位数有限,
  超过一定位数的部分就被切掉了;
  于是两个很接近的数字在计算机看来有可能是相等的, 进而计算不出差值.};
事实上, 我们可以利用二项式展开, 因为

\((1+x)^n=1+nx+\frac{n(n-1)}{2!}x^2+...,\)

代入 \(x=(v^2/c^2)\) 与 \(n=\frac{1}{2}\) 便有

\(m=m_0\left(1+(-1/2)\left(-\frac{v^2}{c^2}\right)+\frac{(-1/2)(-1/2-1)}{2!}\left(-\frac{v^2}{c^2}\right)^2\right)=m\left(1+\frac{v^2}{2c^2}+\frac{3}{8}\frac{v^4}{c^4}+...\right).\)

这个形式下, 相对论性质量与静止质量之差就很明显

\(m\left(\frac{v^2}{2c^2}+\frac{3}{8}\frac{v^4}{c^4}+...\right),\)

利用前几项便可以得到很好的近似.
