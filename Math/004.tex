\begin{quote}
無名, 天地之始; 有名, 萬物之母. 常無, 欲以觀其妙; 常有, 欲以觀其徼. -
苏辙 『老子解』
\end{quote}

\hypertarget{ux5355ux5c04-ux6ee1ux5c04-ux53ccux5c04-injection-surjection-bijection}{%
\subsubsection{单射, 满射, 双射 (injection, surjection,
bijection)}\label{ux5355ux5c04-ux6ee1ux5c04-ux53ccux5c04-injection-surjection-bijection}}

一个函数 \(f:X\rightarrow Y\) 若满足, 如果 \(a\neq b\) 则
\(f(a)\neq f(b)\) 对于任何属于 \(X\) 的 \(a\) 和 \(b\),
那么它便是\textbf{单射}的 (injection, one-to-one)\footnote{One-to-one
  是更''纯正''英语的说法, 比较通俗, injection 是来自法语的舶来词,
  更具高级感; 后面的 onto 和 surjection 同.}.

一个函数 \(f:X\rightarrow Y\) , 若它的值域 (range) 和陪域 (codomain)
一致, 即对于任意 \(y\in Y\), 都存在至少一个 \(x\in X\) 满足
\(f(x)=y\)\footnote{介绍一下符号语言: \(\exist\) - 存在; \(\forall\) -
  对于所有. 于是这句话可以这么表述:
  \(\forall y\in Y, \exist x\in X \text{ s.t. } f(x)=y\) (s.t.=such that
  可以译为''使得''). 但是通常情况下,
  还是尽量避免符号语言而使用自然语言来描述.}, 那么它便是\textbf{满射}的
(surjection, onto).

一个同时单射又满射的函数是\textbf{双射}的 (bijection, one-to-one
correspondance).

还是以Captial这个函数为例子, 下面给出了单射, 满射,
双射三种情况分别的图示

\begin{quote}
左: 单射但不满射; 中: 满射但不单射; 右: 双射.
\end{quote}

\hypertarget{ux5947ux5076ux6027-parity}{%
\subsubsection{\texorpdfstring{奇偶性 (parity
\doge)}{奇偶性 (parity )}}\label{ux5947ux5076ux6027-parity}}

若一个函数满足 \(f(-x)=-f(x)\), 即改变输入值 (自变量) 的正负号, 输出值
(因变量) 的正负号也改变, 这个函数便是\textbf{奇函数} (odd function).
图像上它是关于原点对称的.

若一个函数满足 \(f(-x)=f(x)\), 即改变输入值 (自变量) 的正负号,
不影响输出值 (因变量) , 这个函数便是\textbf{偶函数} (even function).
图像上它是关于 \(y\) 轴对称的.

当然, 奇函数和偶函数事实上是很特殊的两类函数,
更多的函数既不是奇函数又不是偶函数.

一些运算规律:

\begin{itemize}
\tightlist
\item
  奇函数 + 奇函数 = 偶函数 证明: 假设存在两个奇函数 \(f(x)\) 和
  \(g(x)\), 令 \((f+g)(x) := f(x) + g(x)\), 即 \((f+g)(x)\)
  这个函数是原本两函数之和. 根据奇函数的定义, \(f(-x)=-f(x)\) 且
  \(g(-x)=-g(x)\), 将两式相加得 \(f(-x)+g(-x)=-f(x)-g(x)\), 即有
  \((f+x)(-x)=-(f+g)(x)\), 可见 \((f+g)(x)\) 是偶函数.
\item
  偶函数 + 偶函数 = 偶函数 本条及接下来的证明与上一条类似, 可以当作练习.
\item
  奇函数 ×/÷ 奇函数 = 偶函数
\item
  偶函数 ×/÷ 偶函数 = 偶函数
\item
  奇函数 ×/÷ 偶函数 = 奇函数
\item
  偶函数 ×/÷ 奇函数 = 奇函数
\end{itemize}

\hypertarget{ux53cdux51fdux6570-inverse-function}{%
\subsubsection{反函数 (inverse
function)}\label{ux53cdux51fdux6570-inverse-function}}

浅浅地非专业地叙述一下反函数. 设函数 \(y=f(x)\ (x\in X)\) 的值域是
\(Y\), 若存在一个函数 \(g(y)\) 使得 \(x= g(y)\ (y\in C)\), \(g(x)\)
便叫做 \(f(x)\) 的\textbf{反函数} (inverse function), 可以记作
\(x=f^{-1}(y)\), 它的定义域和值域分别是原函数的值域和定义域。

图像上, 反函数和原函数关于 \(y=x\) 对称.

在求反函数时要特别注意反函数与原函数的定义域和值域. 例如 \(y=f(x)=x^2\),
因为 \((\pm x)^2=y\), 反函数可能是 \(x=f^{-1}(y)=\sqrt{y}\) 也可能是
\(x=f^{-1}(y)=-\sqrt{y}\) , 但是不能是 \(x=f^{-1}(y)=\pm\sqrt{y}\),
因为这样便不符合函数定义了, 一个输入值不可以有多个输出值,
或则说一个自变量不能对应多个因变量
(但是多个因变量对应一个自变量是允许的, 可以参考满射但不单射的图例).
这里反函数取正或负取决于原函数的定义域, 若 \(y=f(x)=x^2, x\ge 0\), 则
\(x=f^{-1}(y)=\sqrt{y}\); 若 \(y=f(x)=x^2, x\le 0\), 则
\(x=f^{-1}(y)=-\sqrt{y}\).

\hypertarget{ux9690ux51fdux6570-implicit-function}{%
\subsubsection{隐函数 (implicit
function)}\label{ux9690ux51fdux6570-implicit-function}}

有的时候可能需要用函数来表达一个比较复杂的图像, 举一个简单一点的例子,
一个圆心位于原点的单位圆, 圆上任意一点到圆心距离都是 \(1\), 于是有
\(x^2+y^2=1\), 用前面学习的函数的形式表达这个关系, 有

\(y=\begin{cases}\sqrt{1-x^2}\\-\sqrt{1-x^2}\end{cases}.\)

这样似乎还没有起先的 \(x^2+y^2=1\) 这个形式美观, 因此不妨还是用
\(x^2+y^2-1=0\) 来表述单位圆上的 \(x\) 与 \(y\) 的关系. 类似这样,
利用一个【同时关于 \(x\) 与 \(y\) 的表达式 \(F(x,y)=0\)】来确定【 \(y\)
关于 \(x\) 的函数】的表达式, 我们称之为\textbf{隐函数} (implicit
function); 为表区分, 前面介绍的类似 \(y=f(x)\) 的函数,
称为\textbf{显函数} (explicit function).

\hypertarget{ux7ebfux6027-linearity}{%
\subsubsection{线性 (linearity)}\label{ux7ebfux6027-linearity}}

这是一个很好的特性, 并不局限于函数, 仅对于函数来说的话,
若一个函数是线性的, 便有

\(f(a+b)=f(a)+f(b),\ f(ax)=af(x).\)
