\documentclass[openany]{book}
\usepackage[UTF8]{ctex}
\title{从零开始的数学}
\newcommand{\booksubtitle}{从零开始系列}
\newcommand{\booklicense}{Creative Commons}

\author{Zibo (AHBETE) Wang}
% Author subtitle could be a university or a geographical location, for example
\newcommand{\authorsubtitle}{}

% Create convenient commands \booktitle and \bookauthor
\makeatletter
\newcommand{\booktitle}{\@title}
\newcommand{\bookauthor}{\@author}
\makeatother

% This utf8 declaration is not needed for versions of latex > 2018 but may
% be helpful for older software. Eventually it may not be worth keeping.
\usepackage[utf8]{inputenc}  
\usepackage{fix-cm}  % this package allows large \fontsize
\usepackage{tikz}    % this is for graphics. e.g. rectangle on title page
\usepackage{amsmath,amssymb} % Used by equations
\usepackage{physics,bookmark,tcolorbox}
\tcbuselibrary{breakable}
\setlength\parindent{0pt}

% The following dimensions specify 4.75" X 7.5" content on 6 3/8" by 9 1/4"
% paper. The paper width and height can be tweaked as required and the content
% should size to fit within the margins accordingly.
%
% The (inside) bindingoffset should be larger for books with more pages. Some
% standard recommended sizes are .375in minimum up to 1in for 600+ page books.
% Sizes .75in and .875in are also recommended roughly at 150 and 400 pages.
\usepackage[bindingoffset=0.625in,
            left=.5in, right=.5in,
            top=.8125in, bottom=.9375in,
            paperwidth=6.375in, paperheight=9.25in]{geometry}
% Here is an alternative geometry for reading on letter size paper:
% \usepackage[margin=.75in, paperwidth=8.5in, paperheight=11in]{geometry}

\renewcommand{\contentsname}{Table of Contents} % default is {Contents}
\usepackage{makeidx}
\makeindex % Initialize an index so we can add entries with \index

% Content Starts Here
\begin{document}
\frontmatter

% ---- Title Page ----
% current geometry will be restored after title page
\newgeometry{top=1.75in,bottom=.5in}
\begin{titlepage}
\begin{flushleft}

% Title
\begin{center}
\textbf{\fontfamily{qcs}{\fontsize{42}{42}\selectfont  \}从零开始\{}\\{\LARGE\selectfont  的数学}}
\end{center}

% Draw a line 4pt high
\par\noindent\rule{\textwidth}{4pt}\\

% Shaded box from left to right with Subtitle
% The text node is midway (centered).
\begin{tikzpicture}
\shade[bottom color=lightgray,top color=white]
    (0,0) rectangle (\textwidth, 1.5)
    node[midway] {\textbf{\large {\booksubtitle}}};
\end{tikzpicture}

% Edition Number
\begin{flushright}
\large Zeroth Edition
\end{flushright}

\vspace{\fill}

% Author and Location
\textbf{\large Author: }\textbf{\large \bookauthor}\\[3.5pt]
\textbf{\large \textit{\authorsubtitle}}\\
% \textbf{\large Editor: Jiyuan (Maki) Lu}\\
% \textbf{\large }

\vspace{\fill}

\end{flushleft}
\end{titlepage}
\restoregeometry
% ---- End of Title Page ----

% Do not show page numbers on colophon page
\thispagestyle{empty}

\begin{flushleft}
\vspace*{\fill}
% AHBETE's physics notes series\\
% A member of Maki's Lab (https://www.maki-math.com)\\
% Contact info: wangz57@rpi.edu\\
\vspace{\fill}
Copyright \textcopyright{} \the\year{}  \bookauthor\\
License: \booklicense
\end{flushleft}

% A title page resets the page # to 1, but the second title page
% was actually page 3. So add two to page counter.
\addtocounter{page}{2}

% The asterisk excludes chapter from the table of contents.
\chapter*{Preface}


% Three-level Table of Contents
\setcounter{tocdepth}{1}
\tableofcontents

\mainmatter

\part{}

\chapter{预备微积分 (precalculus)}
\hypertarget{ux52a0ux6cd5-addition}{%
\subsubsection{加法 (addition)}\label{ux52a0ux6cd5-addition}}

\(1+1=2\) , 好了, 这一节结束 (玩笑). 加法的运算, 九九加法表,
这里就不赘述.

我们首先考虑\textbf{整数} (integer), 记作 \(\mathbb{Z}\),
当我们考虑一类数字或者一类符合某种性质的对象时,
我们称这一类对象组成的东西叫做\textbf{集合} (set),
集合中的东西便叫做\textbf{元素} (element).

\begin{itemize}
\tightlist
\item
  不难发现, 从整数里取出两个元素, 例如 \(1\) 和 \(2\) , 将他们相加,
  得到的结果 \(1+2=3\) 依旧是整数.
  这样的性质叫做\textbf{封闭性}或者\textbf{闭包性} (closed, closure).
\item
  再考虑三个元素在加法下的运算. 三个元素的运算可以看作,
  两个元素的运算结果与第三个元素再次运算, 还是不难发现,
  任意从整数取出三个元素, 例如 \(1\) , \(2\) 和 \(3\) , 有
  \((1+2)+3=6=1+(2+3)\) ,
  即前两个元素先进行运算和后两个元素先进行运算的结论时一致的.
  这样的性质叫做\textbf{结合律} (assosiative).
\item
  整数里存在一个特殊的元素,
  使得加法这个运算不对其他任何元素''产生效果'', 这个特殊的元素是 \(0\) ,
  \(0+n=n+0=n\) , 这里 \(n\) 是任意整数, 我们可以这么标记:
  \(n\in\mathbb{Z}\) , 中间这个符号表示属于 (belong to).
  这个特殊的元素叫做\textbf{单位元}或\textbf{幺元} (identity
  element)\footnote{单位和幺都有一的含义, 因为在乘法中单位元是1,
    这可能是名字来源.}.
\item
  整数的加法中, 对于任何一个元素, 都能找到另一个元素,
  使得它们运算结果为单位元- \(0\) , 比如 \(1+(−1)=(−1)+1=0\) , 我们便叫
  \((−1)\) 是 \(1\) 的\textbf{逆元} (inverse element).
\end{itemize}

一个集合, 再附加一个二元运算(像加法这样输入两个元素输出一个元素的运算),
并且拥有上述性质和元素的, 我们便把它叫做\textbf{群} (group), 整数和加法,
便是这样构成了一个群 \((\mathbb{Z},+)\) .

好的,我们在学习加法的过程中顺便体验了以下群论. 要注意的是,
上文并没有强调\textbf{交换性} (commutative), 因为往后看我们会发现,
很多运算其实并不满足交换律,
满足交换律的群我们可以称它为\textbf{交换群}或\textbf{阿贝尔群} (Abelian
group).

\begin{quote}
有人问一个小朋友, ``3+4 等于几啊?'' 小朋友说: ``不知道, 但我知道 3+4
等于 4+3.'' 那人接着问: ``为什么呀?''
答曰:``因为整数与整数加法构成了阿贝尔群.''
这个笑话讽刺了某次法国一场幼儿园从抽象数学教起的实验,
不过最后实验的结果是以失败告终.
\end{quote}

\hypertarget{ux51cfux6cd5-subtraction}{%
\subsubsection{减法 (subtraction)}\label{ux51cfux6cd5-subtraction}}

思路要打开, 减法可以看作是加法的逆运算; 又或者, 减去一个数,
可看作加上这个数字的逆元.

减法的一些性质:

\begin{itemize}
\tightlist
\item
  \textbf{反交换律} (anti-commutativity), 例如 \(4−3=−(3−4)\) ,
  交换两个元素的顺序会导致结果变为之前结果的逆.
\item
  \textbf{非结合律} (non-associativity), 例如 \((6−3)−2\neq 6−(3−2)\) .
\end{itemize}

因为整数减法不满足结合律, 所以整数和减法不构成群, 只构成\textbf{拟群}
(quasi-group), 字面上可以理解成, 像群, 但是不是群, 这里不做展开.

\hypertarget{ux4e58ux6cd5-multiplication}{%
\subsubsection{乘法
(multiplication)}\label{ux4e58ux6cd5-multiplication}}

还是九九乘法表, 结束了 (玩笑). 乘法可以视作是多个同样加法的标记, 例如:
\(2×3=2+2+2=3+3=3×2\) .

来看看乘法的一些性质:

\begin{itemize}
\tightlist
\item
  易见\textbf{封闭性}或者\textbf{闭包性}是满足的.
\item
  也不难看出乘法具有\textbf{结合律}.
\item
  乘法的\textbf{幺元}是 1 .
\item
  再\textbf{逆元}上似乎出了问题, 到目前为止, 我们讨论的都还是整数
  \(\mathbb{Z}\) , 这个范围内, 似乎找不到逆元, 但是没有关系,
  我们把范围扩大到非零\textbf{有理数} (rational number)\footnote{有理数其实是谬译,
    rational在这里其实意为可约的, 而不是有理的.}, 记作
  \(\{\mathbb{Q}/\{0\}\}\) , 这样每一个元素 \(n\in\{\mathbb{Q}/\{0\}\}\)
  都有逆元 \(\frac{1}{n}\) . 将 \(0\) 剔除是因为它没有逆元, 我们应该知道
  0 不能作为分母.
\end{itemize}

以上性质已经决定了非零有理数和乘法构成群, \((\mathbb{Q}/\{0\}\,×)\) .
乘法另外还有特性:

\begin{itemize}
\tightlist
\item
  乘法与加法的混合运算, 会有\textbf{分配律} distributive property, 例如
  \(2×(3+4)=2×3+2×4\) .
\item
  任何数乘上 \(0\) 得到 \(0\) , \(0\) 可以称作乘法的\textbf{零元} (zero
  element), 零元没有逆.
\end{itemize}

\hypertarget{ux9664ux6cd5-division}{%
\subsubsection{除法 (division)}\label{ux9664ux6cd5-division}}

乘法和除法的关系类似加法和减法的关系. 除以零在大多数场景下是不被定义的.

\begin{quote}
轻清者上浮而为天 重浊者下凝而为地
\end{quote}

\hypertarget{ux5e42ux8fd0ux7b97-exponentiation}{%
\subsubsection{幂运算
(exponentiation)}\label{ux5e42ux8fd0ux7b97-exponentiation}}

幂运算可以视作重复的乘法, 即
\(a^n=\underbrace{a\times ...\times a}_{n}\), 这里 \(a\) 称为底数 (base)
, \(n\) 称为指数 (exponent)\footnote{从这一篇开始,
  文章的叙述讲逐渐从''具体→抽象''过渡到''抽象→具体'',
  即由之前先给一个具体数字运算的例子推广到用字母表示的通常情况,
  变为反过来的顺序; 阅读过程中如果觉得不适应, 抽象的点读不懂时,
  可以先接着往下看, 若之后有一个具体的例子, 可能对理解会有帮助.},
\(a^n\) 读作 \(a\) 的 \(n\) 次幂,或 \(a\) 的 \(n\) 次方.

先考虑正整数次幂, 一些运算规律:

\begin{itemize}
\tightlist
\item
  \(\begin{aligned}a^m\times a^n=\underbrace{a\times ...\times a}_{m}\times\underbrace{a\times ...\times a}_{n}&&\\  =\underbrace{a\times ...\times a}_{n+m}&&=a^{n+m}\end{aligned}\)
\item
  \(\begin{aligned}a^m\div a^n=\underbrace{a\times ...\times a}_{m}\div\underbrace{(a\times ...\times a)}_{n}&&\\  =\underbrace{a\times ...\times a}_{n-m}&&=a^{n-m}\end{aligned}\)
\end{itemize}

再来考虑 \(0\) 次幂, 因为上述运算规律 \(a^n\times a^0=a^{n+0}=a^n\),
因此应该有 \(a^0=1\) ; 要注意, 当底数为 \(0\) 时, \(0^0\)
是不被定义的\footnote{一说理由和 \(0\) 不能作为除数类似;
  另一说要从函数的角度出发, 这里稍稍剧透, 即,
  构建不同的函数极限试图求这个''值''会有不同的结果,
  所以这个''值''没有一个很好的公认的定义.}.

\begin{itemize}
\tightlist
\item
  \(a^0=1\)对于非零的 \(a\).
\end{itemize}

现在来看负整数为指数的幂, 参考第二条规律, 不难看出 \(a^{-n}\)
可以理解为除掉了 \(n\) 个 \(a\), 因此有

\begin{itemize}
\tightlist
\item
  \(a^{-n}=\frac{1}{a^n}\).
\end{itemize}

因为在前面一节我们已经把我们研究的范围扩充到了所有有理数,
所以不妨来看看分数作为指数的情况. 考虑 \(a^{\frac{1}{2}}\), 这里 \(a\)
是有理数, 令 \(b:=a^{\frac{1}{2}}\), 平方可得
\(b^2=a^{\frac{1}{2}}\times a^{\frac{1}{2}}=a^{\frac{1}{2}+\frac{1}{2}}=a\);
事实上我们知道, 要求 \(b\) 的话, 只需进行''开方''这个操作, 记作
\(b=\sqrt{a}\), 因此有 \(a^{\frac{1}{2}}=b=\sqrt{a}\); 然而,
这个操作其实是有一点''小问题''的.

\begin{quote}
这个''小问题''便是, 目前为止, 我们讨论的范围还限于有理数,
然而上述操作得到的 \(\sqrt{a}\) 并不一定是有理数;
这个问题在历史上也困扰了人们很久.

起初人们认为数轴上所有的数都应该可以用整数之比 (也就是有理数) 来表示,
但有人发现, 例如边长为1的正方形, 其对角线的平方利用\textbf{勾股定律}
(Pythagorean theorem - 毕达哥拉斯定律) 应该是 \(2\),
找不出一个有理数使得其平方正好为 \(2\).

然后为了解决问题, 提出问题的人就被解决掉了, 悲伤的故事.

现在, 平方正好为 \(2\) 的数字被记作了 \(\sqrt{2}\),
它不是有理数的证明可以留作证明, 一点提示就是可以利用反证法,
首先假设它是一个有理数, 并可以表示为例如 \(\frac{p}{q}\), 且 \(p\) 和
\(q\) 都是正整数, 然后证明这样的 \(\frac{p}{q}\) 不可能存在.
\end{quote}

解决这个''小问题''的方法, 是要再次扩展我们研究的范围; 这次我们将有理数,
即整数和分数, 以及数轴上''剩余''的那些不能表示成分数形式的无理数,
统称为\textbf{实数} (real number), 记作 \(\mathbb{R}\).
这样一来我们便不必担忧开方的结果''掉到''范围外了, 上面的结论也不难推广为

\begin{itemize}
\tightlist
\item
  \(a^{\frac{n}{m}}=\sqrt[m]{a^n}=(\sqrt[m]{b})^n\).
\end{itemize}

\hypertarget{ux5bf9ux6570-logarithm}{%
\subsubsection{对数 (logarithm)}\label{ux5bf9ux6570-logarithm}}

对数是幂运算的逆运算. 若 \(y=a^x\), 定义对数运算为 \(x=\log_a(y)\),
\(a\) 叫做底数 (base) , \(y\) 叫做真数.

对数有以下运算规律:

\begin{itemize}
\tightlist
\item
  \$ \log\_a(XY)= \log\_a(X)+ \log\_a(Y)\$. 证明如下: 令
  \(x=\log_a(X)\), \(y=\log_a(Y)\), 根据对数定义则有 \(a^x=X\),
  \(a^y=Y\);
  \(\log_a(XY)=\log_a(a^x\times a^y)=\log_a(a^{x+y})=x+y=\log_a(X)+ \log_a(Y)\).
\item
  \$ \log\_a\left(\frac{X}{Y}\right)= \log\_a(X)- \log\_a(Y)\$.
  证明和上一条类似.
\item
  \(\log_a(x^n)=n\log_a(x)\). 由第一条规律可得
  \(\log_a(x^n)=\underbrace{\log_a(x)\times...\times\log_a(x)}_{n}=n\log_a(x)\).
\item
  \(\log_a(x)=\frac{\log_b(x)}{\log_b(a)}\). 令 \(\log_a(x)=t\), 则有
  \(x=a^t\), 对两边同时取以 \(b\) 为底数的对数,
  \(\log_b(x)=\log_b(a^t)=t\log_b(a)=\log_a(x)\log_b(a)\),
  整理便可得上述规律.
\end{itemize}

以上四条为最基本最常用得运算规律,
还有一些运算规律可以从上面几条推到而来, 证明留作练习.

\begin{itemize}
\tightlist
\item
  \(\log_{a^n}x=\frac{1}{n}\log_{a}x\).
\item
  \(a^{\log_a(x)}=\log_a(a^x)=x\).
\item
  \(x^{\log_a(y)}=y^{\log_a(x)}\).
\item
  \(\log_a(x)=\frac{1}{\log_x(a)}\).
\item
  \(\log_a(b)\log_b(x)=\log_a(x)\).
\end{itemize}

\input{003}
\input{004}
\hypertarget{ux4e09ux89d2ux51fdux6570-trigonometry}{%
\subsubsection{三角函数
(Trigonometry)}\label{ux4e09ux89d2ux51fdux6570-trigonometry}}

三角函数最基本的使用应该是表示直角三角形的变长比. 如下图所示, 三角形
\[ABC\] 为直角三角形, 将 \[\angle BAC\] 记作 \[\theta\], 对于两条直角边
\[AB\] 和 \[BC\], 边 \[AB\] 在 \[\theta\] 边上, 称它为\textbf{邻边}
(adjacent), 边 \[BC\] 在 \[\theta\] 对面, 称它为\textbf{对边}
(opposite), 剩余的边 \[AC\] 被称为\textbf{斜边} (hypotenuse)。

易见, 各变长比仅和 \[\theta\] 相关\footnote{当然也可以说和除了直角外的另一个角
  \[(90^\circ-\theta)\] 相关; 边长比可以通过一个除直角外的角确定是因为,
  除直角外另一角相等的直角三角形都相似, 它们的边长比是一致的。},
三角函数便是用来表示各个比例的, 常用的三角函数有

\[\begin{align*}\cos\theta&=\frac{邻边}{斜边}=\frac{AB}{AC},\\
\sin\theta&=\frac{对边}{斜边}=\frac{BC}{AC},\\
\tan\theta&=\frac{对边}{邻边}=\frac{BC}{AB}.\end{align*}\]

不难看出\[\tan\theta=\frac{\sin\theta}{\cos\theta}\].

另外还有

\[\begin{align*}\sec\theta&\equiv\frac{1}{\sin\theta},\\
\csc\theta&\equiv\frac{1}{\cos\theta},\\
\cot\theta&\equiv\frac{1}{\tan\theta}.\end{align*}\]

\[\csc\] 很多时候也记作 \[\text{cosec}\].

一个非常实用的关系, 直角三角形中有\textbf{勾股定理} (Pythagorean
theorem): 斜边边长平方等于两直角边边长的平方之和, 即 \[AC^2=AB^2+BC^2\];
两边同时除以 \[AC^2\] 便有

\begin{itemize}
\tightlist
\item
  \[\boxed{1=\cos^2\theta+\sin^2\theta}\].\footnote{三角函数的平方:
    cos(x)\textsuperscript{2} 通常理解为 cos((x)\textsuperscript{2});
    cos\textsuperscript{2}x 约定俗成表示 (cos(x))\textsuperscript{2}.}
\end{itemize}

\textbf{正弦定律 Law of sine}

将三角形三个角分别记作 \[\alpha\], \[\beta\], 和 \[\gamma\],
将它们的对边分别记作 \[A\], \[B\], 和 \[C\]. 先是结论:

\begin{itemize}
\tightlist
\item
  \[\boxed{\frac{A}{\sin\alpha}=\frac{B}{\sin\beta}=\frac{C}{\sin\gamma}}\].
\end{itemize}

推导如下:

如上图所示, 以 \[C\] 为底做高, 将原本的三角形分为左右两个直角三角形,
这条高利用左边的直角三角形可以表示为 \[A\sin\beta\],
利用右边的直角三角形则是 \[B\sin\alpha\], 于是有
\[A\sin\beta=B\sin\alpha\], 整理可得
\[\frac{A}{\sin\alpha}=\frac{B}{\sin\beta}\];
再做另一条高重复前面的操作, 便可得到完整的结论.

\textbf{余弦定律 Law of cosine}

还是先上结论:

\begin{itemize}
\tightlist
\item
  \[\boxed{B^2=A^2+C^2-2AC\cos\beta}\],
\end{itemize}

即, 【一条边的边长平方】等于【另两条边的边长平方之】和加上【两倍的
(另两条边边长的乘积) 乘以 (另两条边的夹角的余弦)】.

推导如下:

如下图所示, 依旧利用底边 \[C\] 上的高将其分为左右两个直角三角形;
左边的直角三角形, 利用斜边 \[A\] 和角 \[\beta\], 两直角边分别可以表示为
\[A\cos\beta\] 和 \[A\sin\beta\], 于是右边的直角三角形边长便可表述为
\[A\sin\beta\] 和 \[(C-A\cos\beta)\]; 对右边的直角三角形使用勾股定理

\[\begin{align*}B^2&=A^2\sin^2\beta+(C-A\cos\beta)^2\\
&=A^2\sin^2\beta+C^2+A^2\cos^2\beta-2AC\cos\beta\\
&=A^2+C^2-2AC\cos\beta.\end{align*}\]

其中等式的后两行用到了之前得出的 \[1=\cos^2\theta+\sin^2\theta\].

\hypertarget{ux4efbux610fux89d2ux5ea6ux7684ux4e09ux89d2ux51fdux6570}{%
\subsubsection{任意角度的三角函数}\label{ux4efbux610fux89d2ux5ea6ux7684ux4e09ux89d2ux51fdux6570}}

不难发现, 前面讨论的情况似乎都是锐角的情况 (主要是因为插图\ldots),
钝角的三角函数似乎没那么直观了, 因为做不成一个含有钝角的直角三角形,
没法简单地用边长比来表示 \[\sin\] 和 \[\cos\] 等. 于是,
我们需要想办法将前面的情形推广.

如下左图所示, 建立直角坐标系, 做一圆心位于原点的单位圆, 即半径为 \[1\]
的圆, 考虑在第一象限的圆上的一点, 将其与原点做连线, 将从
\[x\]-轴正方向与这条连线\textbf{顺时针}方向形成的夹角记作 \[\theta\],
不难看出这个点的坐标 \[(x,y)\] 满足

\[\begin{cases}x=\cos\theta,\\y=\sin\theta.\end{cases}\]

于是不妨将其他象限的情况也按此定义,
于是如上右图所示的钝角甚至更大角度的三角函数便可以被定义了.

\hypertarget{ux5f27ux5ea6ux5236-radian}{%
\subsubsection{弧度制 (Radian)}\label{ux5f27ux5ea6ux5236-radian}}

为什么一个周角是 \[360^\circ\] 呢, 听说过一个不可考的说法: \[360\]
是一个有很多因数的数字 (1, 2, 3, 4, 5, 6, 8, 9, 10, 12\ldots),
等分起来的时候数字会比较友好, 所以 \[360^\circ\] 其实是非常随意地规定的.
那么有没有更好的用来描述角度方法呢? 答案是弧度.

一个半径为 \[r\] 的圆的周长是 \[2\pi r\], 一个圆心角为 \[n^\circ\]
的扇形的弧长是 \[2\pi r\frac{n}{360}\]. 可见圆心角越大弧越长,
且圆心角和弧长成正比. 既然如此,
不如重新将角度定义为圆心角与弧长的比值以方便计算, 于是便有了,
在新的这套单位系统中, 若圆心角大小为 \[\theta\], 其对应弧长应为
\[r\theta\]; 当圆心角是一个周角时, 对应弧长便成了圆的周长 \[r(2\pi)\].
所以角度和这个新的单位的换算有 \[360^\circ\equiv 2\pi\ \text{rad}\],
因为这个单位把圆心角和对应的弧长联系起来了, 因此称之为\textbf{弧度}
(radian).

扇形面积在这套单位制, 即弧度制下, 便也成了 \[\frac{1}{2}r^2\theta\].

\hypertarget{ux4e09ux89d2ux51fdux6570ux7684ux56feux50cf}{%
\subsubsection{三角函数的图像}\label{ux4e09ux89d2ux51fdux6570ux7684ux56feux50cf}}

现在这个时代, 大家都或多或少能接触到科学计算器,
再不济在bing.com上搜索''solver''用微软的 Microsoft Solver
也可以计算某个特定角度的三角函数值, 自然也可以绘制函数图像.
下图分别展示了 \[\sin(x)\] 和 \[\cos(x)\] 的图像,

一些值得关注的点是它们都是\textbf{周期函数} (periodic function),
随着自变量-角度的变化, 因变量-函数值的变化是周期性的, 它们的周期都是
\[2\pi\], 这一点从上文的单位圆里便可看出些许原因,
当角度变化超过一个周角时, 和角度刚从 \[0\] 开始的情况是一样的.

一个个人很喜欢的可视化如下:

右下显示的是角度 \[\theta\] 不断增加, 左下的图可以看作右下的点的 \[y\]
坐标也就是 \[\sin\theta\] 的值的变化, 左上则是 \[x\] 坐标也就是
\[\cos\theta\] 的值的变化.

\input{006}
\input{007}
\input{008}
\backmatter
\addcontentsline{toc}{chapter}{Index}
\printindex
\end{document}

\documentclass{article}
\usepackage[utf8]{inputenc}
\usepackage{CJKutf8, indentfirst}
\usepackage{graphicx} % Required for inserting images

\title{0}
\author{Zibo Wang}
\date{April 2023}

\begin{document}
\begin{CJK*}{UTF8}{gbsn}
\maketitle

\section{Introduction}
\hypertarget{ux52a0ux6cd5-addition}{%
\subsubsection{加法 (addition)}\label{ux52a0ux6cd5-addition}}

\(1+1=2\) , 好了, 这一节结束 (玩笑). 加法的运算, 九九加法表,
这里就不赘述.

我们首先考虑\textbf{整数} (integer), 记作 \(\mathbb{Z}\),
当我们考虑一类数字或者一类符合某种性质的对象时,
我们称这一类对象组成的东西叫做\textbf{集合} (set),
集合中的东西便叫做\textbf{元素} (element).

\begin{itemize}
\tightlist
\item
  不难发现, 从整数里取出两个元素, 例如 \(1\) 和 \(2\) , 将他们相加,
  得到的结果 \(1+2=3\) 依旧是整数.
  这样的性质叫做\textbf{封闭性}或者\textbf{闭包性} (closed, closure).
\item
  再考虑三个元素在加法下的运算. 三个元素的运算可以看作,
  两个元素的运算结果与第三个元素再次运算, 还是不难发现,
  任意从整数取出三个元素, 例如 \(1\) , \(2\) 和 \(3\) , 有
  \((1+2)+3=6=1+(2+3)\) ,
  即前两个元素先进行运算和后两个元素先进行运算的结论时一致的.
  这样的性质叫做\textbf{结合律} (assosiative).
\item
  整数里存在一个特殊的元素,
  使得加法这个运算不对其他任何元素''产生效果'', 这个特殊的元素是 \(0\) ,
  \(0+n=n+0=n\) , 这里 \(n\) 是任意整数, 我们可以这么标记:
  \(n\in\mathbb{Z}\) , 中间这个符号表示属于 (belong to).
  这个特殊的元素叫做\textbf{单位元}或\textbf{幺元} (identity
  element)\footnote{单位和幺都有一的含义, 因为在乘法中单位元是1,
    这可能是名字来源.}.
\item
  整数的加法中, 对于任何一个元素, 都能找到另一个元素,
  使得它们运算结果为单位元- \(0\) , 比如 \(1+(−1)=(−1)+1=0\) , 我们便叫
  \((−1)\) 是 \(1\) 的\textbf{逆元} (inverse element).
\end{itemize}

一个集合, 再附加一个二元运算(像加法这样输入两个元素输出一个元素的运算),
并且拥有上述性质和元素的, 我们便把它叫做\textbf{群} (group), 整数和加法,
便是这样构成了一个群 \((\mathbb{Z},+)\) .

好的,我们在学习加法的过程中顺便体验了以下群论. 要注意的是,
上文并没有强调\textbf{交换性} (commutative), 因为往后看我们会发现,
很多运算其实并不满足交换律,
满足交换律的群我们可以称它为\textbf{交换群}或\textbf{阿贝尔群} (Abelian
group).

\begin{quote}
有人问一个小朋友, ``3+4 等于几啊?'' 小朋友说: ``不知道, 但我知道 3+4
等于 4+3.'' 那人接着问: ``为什么呀?''
答曰:``因为整数与整数加法构成了阿贝尔群.''
这个笑话讽刺了某次法国一场幼儿园从抽象数学教起的实验,
不过最后实验的结果是以失败告终.
\end{quote}

\hypertarget{ux51cfux6cd5-subtraction}{%
\subsubsection{减法 (subtraction)}\label{ux51cfux6cd5-subtraction}}

思路要打开, 减法可以看作是加法的逆运算; 又或者, 减去一个数,
可看作加上这个数字的逆元.

减法的一些性质:

\begin{itemize}
\tightlist
\item
  \textbf{反交换律} (anti-commutativity), 例如 \(4−3=−(3−4)\) ,
  交换两个元素的顺序会导致结果变为之前结果的逆.
\item
  \textbf{非结合律} (non-associativity), 例如 \((6−3)−2\neq 6−(3−2)\) .
\end{itemize}

因为整数减法不满足结合律, 所以整数和减法不构成群, 只构成\textbf{拟群}
(quasi-group), 字面上可以理解成, 像群, 但是不是群, 这里不做展开.

\hypertarget{ux4e58ux6cd5-multiplication}{%
\subsubsection{乘法
(multiplication)}\label{ux4e58ux6cd5-multiplication}}

还是九九乘法表, 结束了 (玩笑). 乘法可以视作是多个同样加法的标记, 例如:
\(2×3=2+2+2=3+3=3×2\) .

来看看乘法的一些性质:

\begin{itemize}
\tightlist
\item
  易见\textbf{封闭性}或者\textbf{闭包性}是满足的.
\item
  也不难看出乘法具有\textbf{结合律}.
\item
  乘法的\textbf{幺元}是 1 .
\item
  再\textbf{逆元}上似乎出了问题, 到目前为止, 我们讨论的都还是整数
  \(\mathbb{Z}\) , 这个范围内, 似乎找不到逆元, 但是没有关系,
  我们把范围扩大到非零\textbf{有理数} (rational number)\footnote{有理数其实是谬译,
    rational在这里其实意为可约的, 而不是有理的.}, 记作
  \(\{\mathbb{Q}/\{0\}\}\) , 这样每一个元素 \(n\in\{\mathbb{Q}/\{0\}\}\)
  都有逆元 \(\frac{1}{n}\) . 将 \(0\) 剔除是因为它没有逆元, 我们应该知道
  0 不能作为分母.
\end{itemize}

以上性质已经决定了非零有理数和乘法构成群, \((\mathbb{Q}/\{0\}\,×)\) .
乘法另外还有特性:

\begin{itemize}
\tightlist
\item
  乘法与加法的混合运算, 会有\textbf{分配律} distributive property, 例如
  \(2×(3+4)=2×3+2×4\) .
\item
  任何数乘上 \(0\) 得到 \(0\) , \(0\) 可以称作乘法的\textbf{零元} (zero
  element), 零元没有逆.
\end{itemize}

\hypertarget{ux9664ux6cd5-division}{%
\subsubsection{除法 (division)}\label{ux9664ux6cd5-division}}

乘法和除法的关系类似加法和减法的关系. 除以零在大多数场景下是不被定义的.

\begin{quote}
轻清者上浮而为天 重浊者下凝而为地
\end{quote}

\hypertarget{ux5e42ux8fd0ux7b97-exponentiation}{%
\subsubsection{幂运算
(exponentiation)}\label{ux5e42ux8fd0ux7b97-exponentiation}}

幂运算可以视作重复的乘法, 即
\(a^n=\underbrace{a\times ...\times a}_{n}\), 这里 \(a\) 称为底数 (base)
, \(n\) 称为指数 (exponent)\footnote{从这一篇开始,
  文章的叙述讲逐渐从''具体→抽象''过渡到''抽象→具体'',
  即由之前先给一个具体数字运算的例子推广到用字母表示的通常情况,
  变为反过来的顺序; 阅读过程中如果觉得不适应, 抽象的点读不懂时,
  可以先接着往下看, 若之后有一个具体的例子, 可能对理解会有帮助.},
\(a^n\) 读作 \(a\) 的 \(n\) 次幂,或 \(a\) 的 \(n\) 次方.

先考虑正整数次幂, 一些运算规律:

\begin{itemize}
\tightlist
\item
  \(\begin{aligned}a^m\times a^n=\underbrace{a\times ...\times a}_{m}\times\underbrace{a\times ...\times a}_{n}&&\\  =\underbrace{a\times ...\times a}_{n+m}&&=a^{n+m}\end{aligned}\)
\item
  \(\begin{aligned}a^m\div a^n=\underbrace{a\times ...\times a}_{m}\div\underbrace{(a\times ...\times a)}_{n}&&\\  =\underbrace{a\times ...\times a}_{n-m}&&=a^{n-m}\end{aligned}\)
\end{itemize}

再来考虑 \(0\) 次幂, 因为上述运算规律 \(a^n\times a^0=a^{n+0}=a^n\),
因此应该有 \(a^0=1\) ; 要注意, 当底数为 \(0\) 时, \(0^0\)
是不被定义的\footnote{一说理由和 \(0\) 不能作为除数类似;
  另一说要从函数的角度出发, 这里稍稍剧透, 即,
  构建不同的函数极限试图求这个''值''会有不同的结果,
  所以这个''值''没有一个很好的公认的定义.}.

\begin{itemize}
\tightlist
\item
  \(a^0=1\)对于非零的 \(a\).
\end{itemize}

现在来看负整数为指数的幂, 参考第二条规律, 不难看出 \(a^{-n}\)
可以理解为除掉了 \(n\) 个 \(a\), 因此有

\begin{itemize}
\tightlist
\item
  \(a^{-n}=\frac{1}{a^n}\).
\end{itemize}

因为在前面一节我们已经把我们研究的范围扩充到了所有有理数,
所以不妨来看看分数作为指数的情况. 考虑 \(a^{\frac{1}{2}}\), 这里 \(a\)
是有理数, 令 \(b:=a^{\frac{1}{2}}\), 平方可得
\(b^2=a^{\frac{1}{2}}\times a^{\frac{1}{2}}=a^{\frac{1}{2}+\frac{1}{2}}=a\);
事实上我们知道, 要求 \(b\) 的话, 只需进行''开方''这个操作, 记作
\(b=\sqrt{a}\), 因此有 \(a^{\frac{1}{2}}=b=\sqrt{a}\); 然而,
这个操作其实是有一点''小问题''的.

\begin{quote}
这个''小问题''便是, 目前为止, 我们讨论的范围还限于有理数,
然而上述操作得到的 \(\sqrt{a}\) 并不一定是有理数;
这个问题在历史上也困扰了人们很久.

起初人们认为数轴上所有的数都应该可以用整数之比 (也就是有理数) 来表示,
但有人发现, 例如边长为1的正方形, 其对角线的平方利用\textbf{勾股定律}
(Pythagorean theorem - 毕达哥拉斯定律) 应该是 \(2\),
找不出一个有理数使得其平方正好为 \(2\).

然后为了解决问题, 提出问题的人就被解决掉了, 悲伤的故事.

现在, 平方正好为 \(2\) 的数字被记作了 \(\sqrt{2}\),
它不是有理数的证明可以留作证明, 一点提示就是可以利用反证法,
首先假设它是一个有理数, 并可以表示为例如 \(\frac{p}{q}\), 且 \(p\) 和
\(q\) 都是正整数, 然后证明这样的 \(\frac{p}{q}\) 不可能存在.
\end{quote}

解决这个''小问题''的方法, 是要再次扩展我们研究的范围; 这次我们将有理数,
即整数和分数, 以及数轴上''剩余''的那些不能表示成分数形式的无理数,
统称为\textbf{实数} (real number), 记作 \(\mathbb{R}\).
这样一来我们便不必担忧开方的结果''掉到''范围外了, 上面的结论也不难推广为

\begin{itemize}
\tightlist
\item
  \(a^{\frac{n}{m}}=\sqrt[m]{a^n}=(\sqrt[m]{b})^n\).
\end{itemize}

\hypertarget{ux5bf9ux6570-logarithm}{%
\subsubsection{对数 (logarithm)}\label{ux5bf9ux6570-logarithm}}

对数是幂运算的逆运算. 若 \(y=a^x\), 定义对数运算为 \(x=\log_a(y)\),
\(a\) 叫做底数 (base) , \(y\) 叫做真数.

对数有以下运算规律:

\begin{itemize}
\tightlist
\item
  \$ \log\_a(XY)= \log\_a(X)+ \log\_a(Y)\$. 证明如下: 令
  \(x=\log_a(X)\), \(y=\log_a(Y)\), 根据对数定义则有 \(a^x=X\),
  \(a^y=Y\);
  \(\log_a(XY)=\log_a(a^x\times a^y)=\log_a(a^{x+y})=x+y=\log_a(X)+ \log_a(Y)\).
\item
  \$ \log\_a\left(\frac{X}{Y}\right)= \log\_a(X)- \log\_a(Y)\$.
  证明和上一条类似.
\item
  \(\log_a(x^n)=n\log_a(x)\). 由第一条规律可得
  \(\log_a(x^n)=\underbrace{\log_a(x)\times...\times\log_a(x)}_{n}=n\log_a(x)\).
\item
  \(\log_a(x)=\frac{\log_b(x)}{\log_b(a)}\). 令 \(\log_a(x)=t\), 则有
  \(x=a^t\), 对两边同时取以 \(b\) 为底数的对数,
  \(\log_b(x)=\log_b(a^t)=t\log_b(a)=\log_a(x)\log_b(a)\),
  整理便可得上述规律.
\end{itemize}

以上四条为最基本最常用得运算规律,
还有一些运算规律可以从上面几条推到而来, 证明留作练习.

\begin{itemize}
\tightlist
\item
  \(\log_{a^n}x=\frac{1}{n}\log_{a}x\).
\item
  \(a^{\log_a(x)}=\log_a(a^x)=x\).
\item
  \(x^{\log_a(y)}=y^{\log_a(x)}\).
\item
  \(\log_a(x)=\frac{1}{\log_x(a)}\).
\item
  \(\log_a(b)\log_b(x)=\log_a(x)\).
\end{itemize}

\input{003}
\input{004}
\hypertarget{ux4e09ux89d2ux51fdux6570-trigonometry}{%
\subsubsection{三角函数
(Trigonometry)}\label{ux4e09ux89d2ux51fdux6570-trigonometry}}

三角函数最基本的使用应该是表示直角三角形的变长比. 如下图所示, 三角形
\[ABC\] 为直角三角形, 将 \[\angle BAC\] 记作 \[\theta\], 对于两条直角边
\[AB\] 和 \[BC\], 边 \[AB\] 在 \[\theta\] 边上, 称它为\textbf{邻边}
(adjacent), 边 \[BC\] 在 \[\theta\] 对面, 称它为\textbf{对边}
(opposite), 剩余的边 \[AC\] 被称为\textbf{斜边} (hypotenuse)。

易见, 各变长比仅和 \[\theta\] 相关\footnote{当然也可以说和除了直角外的另一个角
  \[(90^\circ-\theta)\] 相关; 边长比可以通过一个除直角外的角确定是因为,
  除直角外另一角相等的直角三角形都相似, 它们的边长比是一致的。},
三角函数便是用来表示各个比例的, 常用的三角函数有

\[\begin{align*}\cos\theta&=\frac{邻边}{斜边}=\frac{AB}{AC},\\
\sin\theta&=\frac{对边}{斜边}=\frac{BC}{AC},\\
\tan\theta&=\frac{对边}{邻边}=\frac{BC}{AB}.\end{align*}\]

不难看出\[\tan\theta=\frac{\sin\theta}{\cos\theta}\].

另外还有

\[\begin{align*}\sec\theta&\equiv\frac{1}{\sin\theta},\\
\csc\theta&\equiv\frac{1}{\cos\theta},\\
\cot\theta&\equiv\frac{1}{\tan\theta}.\end{align*}\]

\[\csc\] 很多时候也记作 \[\text{cosec}\].

一个非常实用的关系, 直角三角形中有\textbf{勾股定理} (Pythagorean
theorem): 斜边边长平方等于两直角边边长的平方之和, 即 \[AC^2=AB^2+BC^2\];
两边同时除以 \[AC^2\] 便有

\begin{itemize}
\tightlist
\item
  \[\boxed{1=\cos^2\theta+\sin^2\theta}\].\footnote{三角函数的平方:
    cos(x)\textsuperscript{2} 通常理解为 cos((x)\textsuperscript{2});
    cos\textsuperscript{2}x 约定俗成表示 (cos(x))\textsuperscript{2}.}
\end{itemize}

\textbf{正弦定律 Law of sine}

将三角形三个角分别记作 \[\alpha\], \[\beta\], 和 \[\gamma\],
将它们的对边分别记作 \[A\], \[B\], 和 \[C\]. 先是结论:

\begin{itemize}
\tightlist
\item
  \[\boxed{\frac{A}{\sin\alpha}=\frac{B}{\sin\beta}=\frac{C}{\sin\gamma}}\].
\end{itemize}

推导如下:

如上图所示, 以 \[C\] 为底做高, 将原本的三角形分为左右两个直角三角形,
这条高利用左边的直角三角形可以表示为 \[A\sin\beta\],
利用右边的直角三角形则是 \[B\sin\alpha\], 于是有
\[A\sin\beta=B\sin\alpha\], 整理可得
\[\frac{A}{\sin\alpha}=\frac{B}{\sin\beta}\];
再做另一条高重复前面的操作, 便可得到完整的结论.

\textbf{余弦定律 Law of cosine}

还是先上结论:

\begin{itemize}
\tightlist
\item
  \[\boxed{B^2=A^2+C^2-2AC\cos\beta}\],
\end{itemize}

即, 【一条边的边长平方】等于【另两条边的边长平方之】和加上【两倍的
(另两条边边长的乘积) 乘以 (另两条边的夹角的余弦)】.

推导如下:

如下图所示, 依旧利用底边 \[C\] 上的高将其分为左右两个直角三角形;
左边的直角三角形, 利用斜边 \[A\] 和角 \[\beta\], 两直角边分别可以表示为
\[A\cos\beta\] 和 \[A\sin\beta\], 于是右边的直角三角形边长便可表述为
\[A\sin\beta\] 和 \[(C-A\cos\beta)\]; 对右边的直角三角形使用勾股定理

\[\begin{align*}B^2&=A^2\sin^2\beta+(C-A\cos\beta)^2\\
&=A^2\sin^2\beta+C^2+A^2\cos^2\beta-2AC\cos\beta\\
&=A^2+C^2-2AC\cos\beta.\end{align*}\]

其中等式的后两行用到了之前得出的 \[1=\cos^2\theta+\sin^2\theta\].

\hypertarget{ux4efbux610fux89d2ux5ea6ux7684ux4e09ux89d2ux51fdux6570}{%
\subsubsection{任意角度的三角函数}\label{ux4efbux610fux89d2ux5ea6ux7684ux4e09ux89d2ux51fdux6570}}

不难发现, 前面讨论的情况似乎都是锐角的情况 (主要是因为插图\ldots),
钝角的三角函数似乎没那么直观了, 因为做不成一个含有钝角的直角三角形,
没法简单地用边长比来表示 \[\sin\] 和 \[\cos\] 等. 于是,
我们需要想办法将前面的情形推广.

如下左图所示, 建立直角坐标系, 做一圆心位于原点的单位圆, 即半径为 \[1\]
的圆, 考虑在第一象限的圆上的一点, 将其与原点做连线, 将从
\[x\]-轴正方向与这条连线\textbf{顺时针}方向形成的夹角记作 \[\theta\],
不难看出这个点的坐标 \[(x,y)\] 满足

\[\begin{cases}x=\cos\theta,\\y=\sin\theta.\end{cases}\]

于是不妨将其他象限的情况也按此定义,
于是如上右图所示的钝角甚至更大角度的三角函数便可以被定义了.

\hypertarget{ux5f27ux5ea6ux5236-radian}{%
\subsubsection{弧度制 (Radian)}\label{ux5f27ux5ea6ux5236-radian}}

为什么一个周角是 \[360^\circ\] 呢, 听说过一个不可考的说法: \[360\]
是一个有很多因数的数字 (1, 2, 3, 4, 5, 6, 8, 9, 10, 12\ldots),
等分起来的时候数字会比较友好, 所以 \[360^\circ\] 其实是非常随意地规定的.
那么有没有更好的用来描述角度方法呢? 答案是弧度.

一个半径为 \[r\] 的圆的周长是 \[2\pi r\], 一个圆心角为 \[n^\circ\]
的扇形的弧长是 \[2\pi r\frac{n}{360}\]. 可见圆心角越大弧越长,
且圆心角和弧长成正比. 既然如此,
不如重新将角度定义为圆心角与弧长的比值以方便计算, 于是便有了,
在新的这套单位系统中, 若圆心角大小为 \[\theta\], 其对应弧长应为
\[r\theta\]; 当圆心角是一个周角时, 对应弧长便成了圆的周长 \[r(2\pi)\].
所以角度和这个新的单位的换算有 \[360^\circ\equiv 2\pi\ \text{rad}\],
因为这个单位把圆心角和对应的弧长联系起来了, 因此称之为\textbf{弧度}
(radian).

扇形面积在这套单位制, 即弧度制下, 便也成了 \[\frac{1}{2}r^2\theta\].

\hypertarget{ux4e09ux89d2ux51fdux6570ux7684ux56feux50cf}{%
\subsubsection{三角函数的图像}\label{ux4e09ux89d2ux51fdux6570ux7684ux56feux50cf}}

现在这个时代, 大家都或多或少能接触到科学计算器,
再不济在bing.com上搜索''solver''用微软的 Microsoft Solver
也可以计算某个特定角度的三角函数值, 自然也可以绘制函数图像.
下图分别展示了 \[\sin(x)\] 和 \[\cos(x)\] 的图像,

一些值得关注的点是它们都是\textbf{周期函数} (periodic function),
随着自变量-角度的变化, 因变量-函数值的变化是周期性的, 它们的周期都是
\[2\pi\], 这一点从上文的单位圆里便可看出些许原因,
当角度变化超过一个周角时, 和角度刚从 \[0\] 开始的情况是一样的.

一个个人很喜欢的可视化如下:

右下显示的是角度 \[\theta\] 不断增加, 左下的图可以看作右下的点的 \[y\]
坐标也就是 \[\sin\theta\] 的值的变化, 左上则是 \[x\] 坐标也就是
\[\cos\theta\] 的值的变化.

\input{006}
\input{007}
\input{008}

\end{CJK*}
\end{document}
