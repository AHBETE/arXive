绕行似乎要结束了, 之前的铺垫使得接下来的道路逐渐明朗\ldots{}

\hypertarget{ux81eaux7136ux5e38ux6570-natural-constant}{%
\subsubsection{自然常数 (natural
constant)}\label{ux81eaux7136ux5e38ux6570-natural-constant}}

通常自然常数会以下面两个例子引出:

\textbf{复利}

考虑一个奇怪的银行, 年利率是 \[100\%\], 也就是说, 存入 \[1\] 个货币,
到了年底便有

\[1\times(1+100\%)=2.\]

更奇怪的一点, 这家银行的单位时间利率不会因为存款周期改变, 也就是说, 存
\[0.5\] 年的利率是 \[0.5\times100\%=50\%\], 那么存半年连本带利取出,
再重新存入, 到了年底会怎么样?

\[1\times\left(1+\frac{100\%}{2}\right)^2=2.25.\]

年底的存款变多了! 那么如果存入更短的周期, 连本带息取出, 然后再存入,
重复这个操作到年末, 会怎么样呢? 考虑存取三次:

\[1\times\left(1+\frac{100\%}{3}\right)^3\approx2.37\].

可以发现年底存款变得更多了. 那如果这样存取的操作足够频繁,
到年底有可能赚取无限多的货币吗? 很可惜, 答案是否定的. 先上结论

\[\lim_{n\rightarrow\infty}\left(1+\frac{1}{n}\right)^n\approx2.71828.\]

虽然还没正式的介绍过''\textbf{极限}'' (limit), ``\textbf{收敛}''
(converge) 这些感念, 但是上式表的的意思是: 左边的 \[\lim\] 是取极限 -
limit - 的意思, 取当 \[n\] 趋向于无穷 (infinity) \[\infty\];
右边则是被取极限的形式, 在上式中, 右边的 \[n\] 便需要趋向于无穷.

计算上的话, 我们可以代入尽可能大的 \[n\],
大多数科学计算器是可以胜任这个估算的; 或者我们可以使用二项式展开
(参见【007】), 省略一些步骤, 不难得到

\[\lim_{n\rightarrow\infty}\left(1+\frac{1}{n}\right)^n=\frac{1}{0!}+\frac{1}{1!}+\frac{1}{2!}+...=\sum_{n=0}^\infty\frac{1}{n!}.\]

可以看到求和的形式, 加的项是逐渐变小的, 这个''变小''是足够快得,
使得整个求和是收敛的 (即有限的),
当然这个求和收敛的严格证明还算留到之后再细说.

既然复利的极限趋向于一个具体的数, 于是我们便规定这个数字叫自然常数:

\[\boxed{\lim_{n\rightarrow\infty}\left(1+\frac{1}{n}\right)^n\equiv\mathrm{e}=2.71828...}\]

\textbf{抽卡}

这也是一个经典的例子, 例如抽卡出货的概率是 \[\frac{1}{x}\],
那么不出货的概率便是 \[100\%-\frac{1}{x}\]; 日常我们会觉得,
比如抛硬币正面概率是 \[\frac{1}{2}\], 那么抛 \[2\]
次大概率上应该能出一个正面, 而事实上, 抛两次还是有挺大概率不出正面的,

从上图可见, 有 \[\frac{1}{4}\] 的概率抛出两次反面; 即, 反面的概率是
\[1-\frac{1}{2}=\frac{1}{2}\], 两次抛硬币互为独立事件,
因此两次都是反面的概率直接是这两个独立事件概率的乘积

\[\left (1-\frac{1}{2}\right)^2=0.25.\]

掷骰子也是一样, 我们总会觉得, 掷 \[6\] 次总''应该''出一个六点吧,
而事实上, 不出六点的概率是 \[1-\frac{1}{6}=\frac{5}{6}\], 于是掷 \[6\]
还是有可能不出六点的, 概率是

\[\left (1-\frac{1}{6}\right)^6\approx0.33.\]

回到抽卡的例子, 如果出货的概率非常非常小: 卡池里有茫茫多的 n 卡, r 卡,
\ldots{} , 只有那么一张 ssr, 要从几乎无限多的卡里抽出一张 ssr 来
(抽完要放回) , 但是相应的, 卡池越大抽卡次数也越多, 于是,
经历了无限次的抽卡后, 依旧有不出货的概率

\[\lim_{n\rightarrow\infty}\left (1-\frac{1}{n}\right)^n=0.367879...\]

同样可以利用二项式展开来估算

\[\lim_{n\rightarrow\infty}\left (1-\frac{1}{n}\right)^n=\frac{1}{0!}-\frac{1}{1!}+\frac{1}{2!}-\frac{1}{3!}...=\sum_{n=0}^\infty\frac{1^{n}}{n!}.\]

这个值事实上是 \[\frac{1}{\mathrm{e}}\]. 证明如下:

\[\begin{align}\left(\lim_{n\rightarrow\infty}\left (1-\frac{1}{n}\right)^n\right)^{-1}&=\lim_{n\rightarrow\infty}\left (1-\frac{1}{n}\right)^{-n}\\\text{(Let }m&=-n\text{ )}\\&=\lim_{n\rightarrow\infty}\left (1+\frac{1}{m}\right)^m\\&\equiv\mathrm{e,}\end{align}\]

可见
\[\boxed{\lim_{n\rightarrow\infty}\left (1-\frac{1}{n}\right)^n=\frac{1}{\mathrm{e}}}\].

\textbf{衰变}

这是笔者个人的一些经历, 本人高中阶段接触的物理教材是比较简单的那种,
于是衰变, 半衰期之类的讲得浅显, 大致有以下结论:

\begin{itemize}
\tightlist
\item
  单独一个原子衰变是一个完全随机的过程, 即我们不可知它具体的衰变时间;
  然而一堆同一种放射性原子, 经过一段时间, 未衰变的原子数量是之前的一半,
  这一段时间便叫做\textbf{半衰期} (half-life), 记作 \[t_{1/2}\];
\item
  没经过一个半衰期, 未衰变的原子数量是在这个半衰期前的数量的半; 即,
  假设某种元素的某个放射性同位素半衰期为一分钟, 一开始有 \[1000\]
  个这样的原子, 经过一分钟后, 大约还有 \[500\] 个没有衰变,
  再经过一分钟后, 还有大约 \[250\] 个没有衰变\ldots{}
\end{itemize}

于是, 用一个关于时间的函数来表示还未衰变的原子的数量便是

\[N(t)=N_0\times\left(\frac{1}{2}\right)^{t/t_{1/2}}.\]

但是用 \[\frac{1}{2}\] 作为底数看起来就很随意, 为什么不能是其他的比例呢?
应该也是可以的, 既然可以用变成原来 \[\frac{1}{3}\] , \[\frac{1}{4}\] ,
\ldots{} 的''三分之一衰期'', ``四分之一衰期'', \ldots{}
来表述某个时间还剩下多少未衰变的, 有没有更''自然''而不那么任意的底数呢?
于是便有\textbf{衰变常数} (decay constant)

\[\lambda:=\frac{\ln(2)}{T},\]

使得还未衰变的原子的数量可以表述为

\[N(t)=N_0\mathrm{e}^{-\lambda t}.\]

这其实是一个换底数的操作 (参见【002】), 不具体推导. 自然常数来了,
于是上面这个式子便''自然''起来了 (其实还没有). 当时稍稍初见这个公式时,
稍稍满意了一些, 但也不知道自然常数作为底数的深意; 直到很后来,
才知道当一个东西的变化率和它本身的大小成正比时 (比如这个衰变这个例子,
单位时间衰变的原子数量和当前未衰变的原子的数量是成正比的),
自然函数总会''自然地''出现.
